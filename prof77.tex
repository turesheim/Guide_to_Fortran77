%this is prof77.tex   originally dated              1995-FEB-22
%added GNU Free Documentation License               2001-Feb-27
%made compatible with Latex and tth                 2001-Oct-23
%fixed bug: recursive WRITE in 2nd version of AREA3 2001-Dec-21
%fixed bug: format I10->I9 in section 2.5           2002-Oct-4
%fixed bug: verbatime=>verbatim                     2002-Jun-20
%fixed typo: than=>then in s13.6                    2002-Jun-23
%fixed omission of NAMED in INQUIRE statement       2002-Jly-29
%two lines merged in example in s6.1 fixed          2002-Jly-31
%changed ==> to use verbatim, bad HTML conversion   2003-Jly-31
%fixed errors in s9.3,9.4, remove pi,pounds chars   2003-Sep-03
%produced better .pdf using pdflatex (thanks to Rory Yorke)
%fixed bad error in MEANSD routine                  2004-Feb-27
%changed S => FACTOR in s2.7  (tks to Jan Wennstrom) 2004-Oct-1
%fixed typos reported by Jan Wennstrom              2004-Nov-22 
%improved Latex, suggestions from David Simpson     2005-Jan-18
%few typos reported by David Forfar                 2005-Jun-07

\documentclass[11pt,a4paper]{article} 
\def\tthdump#1{#1}
\setlength{\textheight}{230mm} 
\setlength{\textwidth}{160mm} 
\setlength{\oddsidemargin}{0mm} 
\setlength{\evensidemargin}{-5mm} 
\setlength{\topmargin}{0pt}
\setlength{\parindent}{5mm}
\setlength{\parskip}{1mm}

\usepackage{times}
\usepackage[pdfborder=0, pdfstartview=FitV]{hyperref}
%\usepackage[colorlinks=true, pdfstartview=FitV, linkcolor=dgreen, citecolor=blue, urlcolor=blue]{hyperref}

\pagestyle{headings}
\title{Professional Programmer's Guide to Fortran77}
\author{Clive G. Page, University of Leicester, UK} 
\date{7th June 2005} 

\begin{document} 
\maketitle 
\vspace*{3mm}
\begin{center}
Copyright \copyright 1988 - 2005 Clive G. Page

Last update/bug fix: 2005 June 5.
\end{center}
\vspace*{3mm}

Permission is granted to copy, distribute and/or modify this document
under the terms of the GNU Free Documentation License, Version 1.1 or any
later version published by the Free Software Foundation; with no Invariant
Sections, with no Front-Cover Texts, and with no Back-Cover Texts.  A copy
of the license is included in the section entitled ``GNU Free Documentation
License''.

This file contains the text of {\em Professional Programmer's Guide to 
Fortran77} published by Pitman in 1988.  The book is now long out  of
print, so it seemed sensible to make the text freely available  over the
Internet.  The ISO Standard for Fortran77 is, of course, now  obsolete,
since Fortran90 and Fortran95 have replaced it.   I strongly recommend
using Fortran95 as a multitude of features have been added to Fortran
which make programming easier  and programs more reliable.  

One of the attractions of Fortran77 is that a good free compiler exists in the
form of GNU Fortran, g77.  At present I don't know of any free compilers
for full Fortran95, but you can download a compiler for a subset language
called F, which seems an excellent way to learn modern Fortran.   Unfortunately
this book will not be much help with F or Fortran95.  Perhaps some day I
may get time to revise it completely.

For more information on Fortran (and F) see these web-sites, which have links
to many others:
\begin{tabular}{p{0.5\textwidth}p{0.4\textwidth}}
\verb+http://www.star.le.ac.uk/~cgp+ & My home page\\
\verb+http://www.fortran.com/+ & Fortran Market and F home page\\
\texttt{http://www.ifremer.fr/ditigo/\newline\hspace*{1em}molagnon/fortran90/engfaq.html}
&Excellent FAQ\\
\texttt{http://dsm.dsm.fordham.edu/\newline\hspace*{1em}\~{}ftnchek/}&FTNCHEK static analyzer
\end{tabular}\\
Whether you write your own programs in Fortran77, or merely use code
written by others, I strongly urge you to use FTNCHEK syntax checker to
find mistakes.  You can download versions for many platforms from the
web-site listed above.

I wrote the book originally using WordPerfect, but later translated it into
\LaTeX\  to make it easier to produce on-line versions in HTML and
Postscript.  The text here is very similar to the published version but
I took the opportunity to correct a few mistakes and make some
very minor updates.  If you find more errors, please let me know (email to
\texttt{c} (at-sign) \texttt{page.demon.co.uk}).

The book was intentionally kept as short as possible so it could be sold 
at a modest price, but I managed to cover the entire Fortran77 language as 
defined in the ANSI and ISO Standards, including several topics 
which are often omitted from much larger textbooks because they are 
deemed to be too ``advanced''. 

In order to encourage the writing of clear, reliable, portable, robust,
and well structured code, short sections appear throughout the book
offering specific guidance on the the practical use of Fortran. Obsolete
or superfluous features of the language, mainly those which have been
retained for compatibility with earlier versions of Fortran, are omitted
from the main text but are covered in the section 13.  This is provided
solely for the assistance of those who have to cope with existing
poorly-written programs or ones which pre-date the Fortran77 standard.

\newpage 
\tableofcontents 
\newpage 

\section{What Is Fortran?} 
%\markboth{What Is Fortran} 

Fortran is the most widely used programming language in the 
world for numerical applications.  It has achieved this position 
partly by being on the scene earlier than any of the other major 
languages and partly because it seems gradually to have evolved 
the features which its users, especially scientists and engineers, 
found most useful.  In order to retain compatibility with old 
programs, Fortran has advanced mainly by adding new features 
rather than by removing old ones.  The net result is, of course, that 
some parts of the language are, by present standards, rather 
archaic: some of these can be avoided easily, others can still be a 
nuisance. 

This section gives a brief history of the language, outlines its 
future prospects, and summarises its strengths and weaknesses. 

\subsection{Early Development} 

Fortran was invented by a team of programmers working for IBM 
in the early nineteen-fifties.  This group, led by John Backus, 
produced the first compiler, for an IBM 704 computer, in 1957. 
They used the name Fortran because one of their principal aims 
was ``formula translation''.  But Fortran was in fact one of the very 
first high-level language: it came complete with control structures 
and facilities for input/output.  Fortran became popular quite 
rapidly and compilers were soon produced for other IBM 
machines.  Before long other manufacturers were forced to design 
Fortran compilers for their own hardware.  By 1963 all the major 
manufacturers had joined in and there were dozens of different 
Fortran compilers in existence, many of them rather more powerful 
than the original.  

All this resulted in a chaos of incompatible dialects.  Some order 
was restored in 1966 when an American national standard was 
defined for Fortran.  This was the first time that a standard had 
ever been produced for a computer programming language.  
Although it was very valuable, it hardly checked the growth of the 
language.  Quite deliberately the Fortran66 standard only 
specified a set of language features which had to be present: it did 
not prevent other features being added.  As time went on these 
extensions proliferated and the need for a further standardization 
exercise became apparent.  This eventually resulted in the current 
version of the language: Fortran77.  

\subsection{Standardization} 

One of the most important features of Fortran programs is their 
portability, that is the ease with which they can be moved from one 
computer system to another.  Now that each generation of 
hardware succeeds the previous one every few years, while good 
software often lasts for much longer, more and more programs 
need to be portable.  The growth in computer networks is also 
encouraging the development of portable programs.  

The first step in achieving portability is to ensure that a standard 
form of programming language is acceptable everywhere.  This 
need is now widely recognised and has resulted in the development 
of standards for all the major programming languages.  In practice, 
however, many of the new standards have been ignored and 
standard-conforming systems for languages like Basic and Pascal 
are still very rare.  

Fortunately Fortran is in much better shape: almost all current 
Fortran systems are designed to conform to the standard usually 
called Fortran77.  This was produced in 1977 by a committee of 
the American National Standards Institute (ANSI) and was 
subsequently adopted by the International Standards Organisation 
(ISO).  The definition was published as ANSI X3.9-1978 and ISO 
1539-1980.  The term ``Standard Fortran'' will be used in the rest 
of this book to refer to mean Fortran77 according to this 
definition.  

Fortran is now one of the most widely used computer languages in 
the world with compilers available for almost every type of 
computer on the market.  Since Fortran77 is quite good at 
handling character strings as well as numbers and also has 
powerful file-handling and input/output facilities, it is suitable for 
a much wider range of applications than before.  

\subsubsection*{Full and Subset Fortran} 

The ANSI Standard actually defines two different levels for 
Fortran77.  The simpler form, subset Fortran, was intended for use 
on computers which were too small to handle the full language.  
Now that even personal computers are powerful enough to handle 
full Fortran77, subset Fortran is practically obsolete.  This book, 
therefore, only describes full Fortran77.   

\subsubsection*{Fortran90} 

The ISO Standard for Fortran90 has, officially, replaced that for 
Fortran77.  It introduces a wealth of new features many of them already 
in use in other high-level languages, which will make programming 
easier, and facilitate the construction of portable and robust programs. 
The whole of the Fortran77 Standard is included as a proper subset, so 
existing (standard-conforming) Fortran programs will automatically 
conform also to the new Standard.  Until well-tested compilers for 
Fortran90 are widespread, however, most programmers are still using 
Fortran77, with perhaps a few minor extensions. 

\subsection{Strengths and Weaknesses} 

Fortran has become popular and widespread because of its unique 
combination of properties.  Its numerical and input/output facilities 
are almost unrivalled while those for logic and character handling 
are as good as most other languages.  Fortran is simple enough that 
you do not need to be a computer specialist to become familiar 
with it fairly quickly, yet it has features, such as the independent 
compilation of program units, which allow it to be used on very 
large applications.  Programs written in Fortran are also more 
portable than those in other major languages.  The efficiency of 
compiled code also tends to be quite high because the language is 
straight-forward to compile and techniques for handling Fortran 
have reached a considerable degree of refinement.  Finally, the 
ease with which existing procedures can be incorporated into new 
software makes it especially easy to develop new programs out of 
old ones. 

It cannot be denied, however, that Fortran has more than its fair 
share of weaknesses and drawbacks.  Many of these have existed 
in Fortran since it was first invented and ought to have been 
eliminated long ago: examples include the 6-character limit on 
symbolic names, the fixed statement layout, and the need to use 
statement labels.   

Fortran also has rather liberal rules and an extensive system of 
default values: while this reduces programming effort it also makes 
it harder for the system to detect the programmer's mistakes.  In 
many other programming languages, for example, the data type of 
every variable has to be declared in advance.  Fortran does not 
insist on this but, in consequence, if you make a spelling mistake 
in a variable name the compiler is likely to use two variables when 
you only intended to use one.  Such errors can be serious but are 
not always easy to detect.  

Fortran also lacks various control and data structures which simplify 
programming languages with a more modern design. These limitations, and 
others, are all eliminated with the advent of Fortran90. 

\subsection{Precautions} 

\subsubsection*{Extensions and Portability} 

Computer manufacturers have a natural tendency to compete with each other 
by providing Fortran systems which are ``better'' than before, usually by 
providing extensions to the language.  This does not conflict with the 
Fortran Standard, provided that standard-conforming programs are still 
processed correctly.  Indeed in the long term languages advance by the 
absorbtion of such extensions. In the short term, however, their use is 
more problematical, since they necessarily makes programs less portable. 

When the latest Fortran Standard was issued in 1977 there was fairly 
widespread disappointment that it did not go just a little further in 
eliminating some of the tiresome restrictions that had persisted since 
the early days.  The US Department of Defense issued a short list of 
extensions which manufacturers were encouraged to add to their Fortran77 
systems.  The most important of these were the following: 
\begin{itemize} 
\item the {\tt END DO} statement 
\item the {\tt DO WHILE} loop 
\item the {\tt INCLUDE} statement 
\item the {\tt IMPLICIT NONE} facility 
\item intrinsic functions for bit-wise operations on integers. 
\end{itemize} 

Many Fortran systems, especially those produced in the United States, now 
support these extensions but they are by no means universal and should 
not be used in portable programs. 

One of the most irksome restrictions of Fortran77 is that symbolic names 
cannot be more than six characters long.  This forces programmers to 
devise all manner of contractions, abbreviations, and acronyms in place 
of meaningful symbolic names.  It is very tempting to take advantage of 
systems which relax this rule but this can have serious repercussions. 
Consider a program which makes use of variables called TEMPERATURE and 
TEMPERED.  Many compilers will be quite happy with these, though a few 
will reject both names on grounds of length.  Unfortunately there are 
also one or two compilers in existence which will simply ignore all 
letters after the sixth so that both names will be taken as references to 
the same variable, TEMPER.  Such behaviour, while deplorable, is quite in 
accordance with the Standard which only requires systems to compile 
programs correctly if they conform to its rules. 

The only way to be certain of avoiding problems like this is to ignore 
such temptations entirely and just use Standard Fortran. Many compilers 
provide a switch or option which can be set to cause all non-standard 
syntax to be flagged.  Everything covered in this book is part of 
Standard Fortran unless clearly marked to the contrary. 

\subsubsection*{Guidelines} 

Computer programming always requires a very high standard of 
care and accuracy if it is to be successful.  This is even more vital 
when using Fortran than with some other languages, because, as 
explained above, the liberal rules of Fortran make it harder for the 
system to detect mistakes.  To program successfully it is not 
enough just to conform to the rules of the language, it is also 
important to defend yourself against known pitfalls.   

There is a useful lesson to be learned from the failure of one of the 
earliest planetary probes launched by NASA.  The cause of the 
failure was eventually traced to a statement in its control software 
similar to this:\\ 
\verb+      DO 15 I = 1.100+\\ 
when what should have been written was:\\ 
\verb+      DO 15 I = 1,100+\\ 
but somehow a dot had replaced the comma.  Because Fortran 
ignores spaces, this was seen by the compiler as:\\ 
\verb+      DO15I = 1.100+\\ 
which is a perfectly valid assignment to a variable called DO15I 
and not at all what was intended. 

Fortran77 permits an additional comma to be inserted after the 
label in a {\tt DO} statement, so it could now be written as:\\ 
\verb+      DO 15,I = 1,100+\\ 
which has the great advantage that it is no longer as vulnerable to a 
single-point failure. 

There are many hazards of this sort in Fortran, but the risk of 
falling victim to them can be minimised by adopting the 
programming practices of more experienced users.  To help you, 
various recommendations and guidelines are given throughout this 
book.  Some of the most outdated and unsatisfactory features of 
Fortran are not described in the main part of the book at all but 
have been relegated to section 13. 

There is not room in a book of this size to go further into the 
techniques of program design and software engineering.  As far as 
possible everything recommended here is consistent with the 
methods of modular design and structured programming, but you 
should study these topics in more detail before embarking on any 
large-scale programming projects. 

\section{Basic Fortran Concepts} 
%\markboth{Basic Fortran Concepts} 

This section presents some of the basic ideas of Fortran by 
showing some complete examples.  In the interests of simplicity, 
the problems which these solve are hardly beyond the range of a 
good pocket calculator, and the programs shown here do not 
include various refinements that would usually be present in 
professional software.  They are, however, complete working 
programs which you can try out for yourself if you have access to 
a Fortran system.  If not, it is still worth reading through them to 
see how the basic elements of Fortran can be put together into 
complete programs. 

\subsection{Statements} 

To start with, here is one of the simplest program that can be 
devised:  
\begin{verbatim}
       PROGRAM TINY 
       WRITE(UNIT=*, FMT=*) 'Hello, world' 
       END 
\end{verbatim}
As you can probably guess, all this program does is to send a 
rather trite message ``Hello, world'' to your terminal.  Even so its 
layout and structure deserve some explanation. 

The program consists of three lines, each containing one statement.  
Each Fortran statement must have a line to itself (or more than one 
line if necessary), but the first six character positions on each line 
are reserved for statement labels and continuation markers.  Since 
the statements in this example need neither of these features, the 
first six columns of each line have been left blank.   

The {\tt PROGRAM} statement gives a name to the program unit and 
declares that it is a main program unit.  Other types of program 
unit will be covered later on.  The program can be called anything 
you like provided the name conforms to the Fortran rules; the first 
character of a Fortran symbolic name must be a letter but, 
unfortunately, they cannot be more than six characters long in 
total.  It is generally sensible to give the same name to the program 
and to the file which holds the Fortran source code (the original 
text).  

The {\tt WRITE} statement produces output: the parentheses enclose a 
list of {\em control} items which determine where and in what form 
the output appears.  \verb?UNIT=*? selects the standard output file which 
is normally your own terminal; \verb?FMT=*? selects a default output 
layout (technically known as list-directed format).  Asterisks are 
used here, as in many places in Fortran, to select a default or 
standard option.  This program could, in fact, have been made 
slightly shorter by using an abbreviated form of the WRITE 
statements: 
\begin{verbatim}
      WRITE(*,*) 'Hello, world' 
\end{verbatim}
Although the keywords {\tt UNIT=} and {\tt FMT=} are optional, they help 
to make the program more readable.  The items in the control list, 
like those in all lists in Fortran, are separated by commas.   

The control information in the {\tt WRITE} statement is followed by a 
list of the data items to be output: here there is just one item, a 
character constant which is enclosed in a pair of apostrophe (single 
quote) characters. 

An {\tt END} statement is required at the end of every program unit.  
When the program is {\em compiled} (translated into machine 
code) it tells the compiler that the program unit is complete; when 
encountered at run-time the {\tt END} statement stops the program 
running and returns control to the operating system. 

The Standard Fortran character set does not contain any lower-case 
letters so statements generally have to be written all in upper case.  
But Fortran programs can process as data any characters supported 
by the machine; character constants (such as the message in the last 
example) are not subject to this constraint.  

\subsection{Expressions and Assignments} 

The next example solves a somewhat more realistic problem: it computes 
the repayments on a fixed-term loan (such as a home mortgage loan).  The 
fixed payments cover the interest and repay part of the capital sum; the 
annual payment can be calculated by the following formula: 

\[ payment = \frac{rate\cdot amount}{(1 - (1+rate)^{-nyears})} \] 

In this formula, rate is the annual interest rate expressed as a 
fraction; since it is more conventional to quote interest rates as a 
percentage the program does this conversion for us.  
\begin{verbatim}
       PROGRAM LOAN 
       WRITE(UNIT=*, FMT=*)'Enter amount, % rate, years' 
       READ(UNIT=*, FMT=*) AMOUNT, PCRATE, NYEARS 
       RATE = PCRATE / 100.0 
       REPAY = RATE * AMOUNT / (1.0 - (1.0+RATE)**(-NYEARS)) 
       WRITE(UNIT=*, FMT=*)'Annual repayments are ', REPAY 
       END 
\end{verbatim}
This example introduces two new forms of statement: the {\tt READ} 
and assignment statements, both of which can be used to assign 
new values to variables. 

The {\tt READ} statement has a similar form to {\tt WRITE}: here it 
reads in three numbers entered on the terminal in response to the 
prompt and assigns their values to the three named variables.  
{\tt FMT=*} again selects list-directed (or free-format) input which 
allows the numbers to be given in any convenient form: they can 
be separated by spaces or commas or even given one on each line. 

The fourth statement is an assignment statement which divides 
{\tt PCRATE} by 100 and assigns the result to another variable called 
{\tt RATE}.  The next assignment statement evaluates the loan 
repayment formula and assigns the result to a variable called 
{\tt REPAY}.   

Several arithmetic operators are used in these expressions: as in most 
programming languages ``{\tt /}'' represents division and ``{\tt *}'' 
represents multiplication; in Fortran ``{\tt **}'' is used for 
exponentiation, i.e.\ raising one number to the power of another. Note 
that two operators cannot appear in succession as this could be 
ambiguous, so that instead of ``{\tt **-N}'' the form ``{\tt **(-N)}'' 
has to be used. 

Another general point concerning program layout: spaces (blanks) are not 
significant in Fortran statements so they can be inserted freely to 
improve the legibility of the program. 

When the program is run, the terminal dialogue will look 
something like this:  
\begin{verbatim}
Enter amount, % rate, years 
20000, 9.5, 15 
Annual repayments are    2554.873 
\end{verbatim}
The answer given by your system may not be exactly the same as 
this because the number of digits provided by list-directed 
formatting depends on the accuracy of the arithmetic, which varies 
from one computer to another.  

\subsection{Integer and Real Data Types} 

The LOAN program would have been more complicated if it had 
not taken advantage of some implicit rules of Fortran concerning 
data types: this requires a little more explanation.  

Computers can store numbers in several different ways: the most 
common numerical data types are those called integer and real.  
Integer variables store numbers exactly and are mainly used to 
count discrete objects.  Real variables are useful many other 
circumstances as they store numbers using a floating-point 
representation which can handle numbers with a fractional part as 
well as whole numbers.  The disadvantage of the real data type is 
that floating-point numbers are not stored exactly: typically only 
the first six or seven decimal digits will be correct.  It is important 
to select the correct type for every data item in the program.  In the 
last example, the number of years was an integer, but all of the 
other variables were of real type.  

The data type of a constant is always evident from its form: character 
constants, for example, are enclosed in a pair of apostrophes.  In 
numerical constants the presence of a decimal point indicates that they 
are real and not integer constants: this is why the value one was 
represented as ``{\tt 1.0}'' and not just ``{\tt 1}''. 

There are several ways to specify the data type of a variable.  One 
is to use explicit type statements at the beginning of the program.  
For example, the previous program could have begun like this: 
\begin{verbatim}
       PROGRAM LOAN 
       INTEGER NYEARS 
       REAL AMOUNT, PCRATE, RATE, REPAY 
\end{verbatim} 

Although many programming languages require declarations of 
this sort for every symbolic name used in the program, Fortran 
does not.  Depending on your point of view, this makes Fortran 
programs easier to write, or allows Fortran programmers to become 
lazy.  The reason that these declarations can often be omitted in 
Fortran is that, in the absence of an explicit declaration, the data 
type of any item is determined by the first letter of its name.  The 
general rule is: 

\begin{tabular}{ll} 
initial letters I-N &         integer type \\ 
initial letters A-H and O-Z & real type. \\ 
\end{tabular} 

In the preceding program, because the period of the loan was 
called {\tt NYEARS} (and not simply {\tt YEARS}) it automatically became 
an integer, while all the other variables were of real type. 

\subsection{\texttt{DO} Loops} 

Although the annual repayments on a home loan are usually fixed, 
the outstanding balance does not decline linearly with time.  The 
next program demonstrates this with the aid of a {\tt DO}-loop.  
\begin{verbatim}
      PROGRAM REDUCE 
      WRITE(UNIT=*, FMT=*)'Enter amount, % rate, years' 
      READ(UNIT=*, FMT=*) AMOUNT, PCRATE, NYEARS 
      RATE = PCRATE / 100.0 
      REPAY = RATE * AMOUNT / (1.0 - (1.0+RATE)**(-NYEARS)) 
      WRITE(UNIT=*, FMT=*)'Annual repayments are ', REPAY 
      WRITE(UNIT=*, FMT=*)'End of Year  Balance' 
      DO 15,IYEAR = 1,NYEARS 
          AMOUNT = AMOUNT + (AMOUNT * RATE) - REPAY 
          WRITE(UNIT=*, FMT=*) IYEAR, AMOUNT  
15    CONTINUE 
      END 
\end{verbatim} 
The first part of the program is similar to the earlier one.  It 
continues with another {\tt WRITE} statement which produces headings 
for the two columns of output which will be produced later on.   

The {\tt DO} statement then defines the start of a loop: the statements 
in the loop are executed repeatedly with the loop-control variable 
{\tt IYEAR} taking successive values from 1 to {\tt NYEARS}.  The first 
statement in the loop updates the value of {\tt AMOUNT} by adding the 
annual interest to it and subtracting the actual repayment.  This 
results in {\tt AMOUNT} storing the amount of the loan still owing at 
the end of the year.  The next statement outputs the year number 
and the latest value of {\tt AMOUNT}.  After this there is a 
{\tt CONTINUE} statement which actually does nothing but act as a 
place-marker.  The loop ends at the {\tt CONTINUE} statement because 
it is attached to the label, {\tt 15}, that was specified in the {\tt DO} 
statement at the start of the loop.   

The active statements in the loop have been indented a little to the 
right of those outside it: this is not required but is very common 
practice among Fortran programmers because it makes the 
structure of the program more conspicuous. 

The program REDUCE produces a table of values which, while 
mathematically correct, is not very easy to read: 
\begin{verbatim}
Enter amount, % rate, years 
2000, 9.5, 5 
Annual repayments are    520.8728     
End of Year  Balance 
      1   1669.127     
      2   1306.822     
      3   910.0968     
      4   475.6832     
      5  2.9800416E-04 
\end{verbatim} 

\subsection{Formatted Output} 

The table of values would have a better appearance if the decimal 
points were properly aligned and if there were only two digits after 
them.  The last figure in the table is actually less than a thirtieth of 
a penny, which is effectively zero to within the accuracy of the 
machine.  A better layout can be produced easily enough by using 
an explicit format specification instead of the list-directed output 
used up to now.  To do this, the last {\tt WRITE} statement in the 
program should be replaced with one like this:\\ 
\verb+      WRITE(UNIT=*, FMT='(1X,I9,F11.2)') IYEAR, AMOUNT+\\ 
The amended program will then produce a neater tabulation:  

\begin{verbatim}
Enter amount, % rate, years 
2000, 9.5, 5 
Annual repayments are    520.8728     
End of Year  Balance 
        1    1669.13 
        2    1306.82 
        3     910.10 
        4     475.68 
        5        .00 
\end{verbatim} 
The format specification has to be enclosed in parentheses and, as 
it is actually a character constant, in a pair of apostrophes as well.  
The first item in the format list, {\tt 1X}, is needed to cope with the 
carriage-control convention: it provides an additional blank at the 
start of each line which is later removed by the Fortran system.  
There is no logical explanation for this: it is there for compatibility 
with very early Fortran system.  The remaining items specify the 
layout of each number: {\tt I9} specifies that the first number, an 
integer, should be occupy a field 9 columns wide; similarly {\tt F11.2} 
puts the second number, a real (floating-point) value, into a field 
11 characters wide with exactly 2 digits after the decimal point.  
Numbers are always right-justified in each field.  The field widths 
in this example have been chosen so that the columns of figures 
line up satisfactorily with the headings. 

\subsection{Functions} 

Fortran provides a useful selection of intrinsic functions to carry 
out various mathematical operations such as square root, maximum 
and minimum, sine, cosine, etc., as well as various data type 
conversions.  You can also write your own functions.  The next 
example, which computes the area of a triangle, shows both forms 
of function in action. 

The formulae for the area of a triangle with sides of length a, b, and 
c is: 
\[ s = (a + b + c)/2 \] 
%%tth: \[ area = sqrt({s.(s-a).(s-b).(s-c)}) \] 
\tthdump{\[ area = \sqrt{[s\cdot (s-a)\cdot (s-b)\cdot (s-c)]} \]}
\begin{verbatim}
      PROGRAM TRIANG 
      WRITE(UNIT=*,FMT=*)'Enter lengths of three sides:' 
      READ(UNIT=*,FMT=*) SIDEA, SIDEB, SIDEC 
      WRITE(UNIT=*,FMT=*)'Area is ', AREA3(SIDEA,SIDEB,SIDEC) 
      END 

      FUNCTION AREA3(A, B, C) 
*Computes the area of a triangle from lengths of sides 
      S = (A + B + C)/2.0 
      AREA3 = SQRT(S * (S-A) * (S-B) * (S-C)) 
      END 
\end{verbatim} 
This program consists of two program units.  The first is the main 
program, and it has as similar form to those seen earlier.  The only 
novel feature is that the list of items output by the {\tt WRITE} 
statement includes a call to a function called {\tt AREA3}.  This 
computes the area of the triangle.  It is an external function which 
is specified by means of a separate program unit technically known 
as a function subprogram.   

The external function starts with a {\tt FUNCTION} statement which 
names the function and specifies its set of dummy arguments.  This 
function has three dummy arguments called {\tt A, B,} and {\tt C}.  The 
values of the actual arguments, {\tt SIDEA, SIDEB,} and {\tt SIDEC}, are 
transferred to the corresponding dummy arguments when the 
function is called.  Variable names used in the external function 
have no connection with those of the main program: the actual and 
dummy argument values are connected only by their relative 
position in each list.  Thus {\tt SIDEA} transfers its value to {\tt A},
and so 
on.  The name of the function can be used as a variable within the 
subprogram unit; this variable must be assigned a value before the 
function returns control, as this is the value returned to the calling 
program. 

Within the function the dummy arguments can also be used as 
variables.  The first assignment statement computes the sum, 
divides it by two, and assigns it to a local variable, {\tt S}; the second 
assignment statement uses the intrinsic function {\tt SQRT} which 
computes the square-root of its argument.  The result is returned to 
the calling program by assigning it to the variable which has the 
same name as the function. 

The {\tt END} statement in a procedure does not cause the program to 
stop but just returns control to the calling program unit. 

There is one other novelty: a comment line describing the action 
of the function.  Any line of text can be inserted as a comment 
anywhere except after an {\tt END} statement.  Comment lines have an 
asterisk in the first column. 

These two program units could be held on separate source files and 
even compiled separately.  An additional stage, usually called 
linking, is needed to construct the complete executable program 
out of these separately compiled object modules.  This seems an 
unnecessary overhead for such simple programs but, as described 
in the next section, it has advantages when building large 
programs.  

In this very simple example it was not really necessary to separate 
the calculation from the input/output operations but in more 
complicated cases this is usually a sensible practice.  For one thing 
it allows the same calculation to be executed anywhere else that it 
is required.  For another, it reduces the complexity of the program 
by dividing the work up into small independent units which are 
easier to manage.   

 
\subsection{IF-blocks} 

Another important control structure in Fortran is the {\tt IF} statement 
which allows a block of statements to be executed conditionally, 
or allows a choice to be made between different courses of action.  

One obvious defect of the function {\tt AREA3} is that has no 
protection against incorrect input.  Many sets of three real numbers 
could not possibly form the sides of a triangle, for example 1.0, 
2.0, and 7.0.  A little analysis shows that in all such impossible 
cases the argument of the square root function will be negative, 
which is illegal.  Fortran systems should detect errors like this at 
run-time but will vary in their response.  Even so, a message like 
``negative argument for square-root'' may not be enough to suggest 
to the user what is wrong.  The next version of the function is 
slightly more user-friendly; unfortunately because one cannot use a WRITE
statement inside a function which is itself being used in a WRITE
statement, the error message has to come from a STOP statement.
\begin{verbatim}
      REAL FUNCTION AREA3(A, B, C) 
*Computes the area of a triangle from lengths of its sides. 
*If arguments are invalid issues error message and returns zero. 
      REAL A, B, C 
      S = (A + B + C)/2.0 
      FACTOR = S * (S-A) * (S-B) * (S-C) 
      IF(FACTOR .LE. 0.0) THEN 
          STOP 'Impossible triangle'
      ELSE 
          AREA3 = SQRT(FACTOR) 
      END IF 
      END 
\end{verbatim} 
The {\tt IF} statement works with the {\tt ELSE} and {\tt END IF}
statements to  enclose two blocks of code.  The statements in the first
block are  only executed if the expression in the {\tt IF} statement is
true, those in  the second block only if it is false.  The statements in
each block  are indented for visibility, but this is, again, just a
sensible  programming practice. 

With this modification, the value of {\tt FACTOR} is tested and if it is 
negative or zero then an error message is produced; {\tt AREA}3 is also 
set to an impossible value (zero) to flag the mistake.  Note that the 
form ``{\tt .LE.}'' is used because the less-than-or-equals character, 
``{\tt <}'', is not present in the Fortran character set.  If {\tt FACTOR} is 
positive the calculation proceeds as before. 

\subsection{Arrays} 

Fortran has good facilities for handling arrays.  They can have up to 
seven dimensions.  The program STATS reads a set of real numbers from a 
data file and puts them into a one-dimensional array.  It then computes 
their mean and standard deviation. Given an array of values $x_{1}, 
x_{2}, x_{3}, ... x_{N}$, the mean M and standard deviation S are 
given by: 
%%tth: \[ M =  \frac{sum (x_{i})}{N} \] 
%%tth: \[ S^{2}  =  \frac{sum((x_{i} - M)^{2}}{(N-1))} \] 

\tthdump{
\[ M =  \frac{\sum x_{i}}{N} \] }

\tthdump{\[ S^{2}  =  \frac{(\sum (x_{i} - M)^{2})}{(N-1)} \]
}

To simplify this program, it will be assumed that the first number 
in the file is an integer which tells the program how many real data 
points follow. 
\begin{verbatim}
       PROGRAM STATS 
       CHARACTER FNAME*50 
       REAL X(1000) 
       WRITE(UNIT=*, FMT=*) 'Enter data file name:' 
       READ(UNIT=*, FMT='(A)') FNAME 
       OPEN(UNIT=1, FILE=FNAME, STATUS='OLD') 
*Read number of data points NPTS 
       READ(UNIT=1, FMT=*) NPTS 
       WRITE(UNIT=*, FMT=*) NPTS, ' data points' 
       IF(NPTS .GT. 1000) STOP 'Too many data points' 
       READ(UNIT=1, FMT=*) (X(I), I = 1,NPTS) 
       CALL MEANSD(X, NPTS, AVG, SD) 
       WRITE(UNIT=*, FMT=*) 'Mean =', AVG, ' Std Deviation =', SD 
       END 

       SUBROUTINE MEANSD(X, NPTS, AVG, SD) 
       INTEGER NPTS 
       REAL X(NPTS), AVG, SD 
       SUM   = 0.0 
       DO 15, I = 1,NPTS 
           SUM   = SUM   + X(I) 
15     CONTINUE 
       AVG = SUM / NPTS 
       SUMSQ = 0.0
       DO 25, I = 1,NPTS
           SUMSQ = SUMSQ + (X(I) - AVG)**2
25     CONTINUE
       SD  = SQRT(SUMSQ /(NPTS-1)) 
       END 
\end{verbatim} 
{\bf NOTE: the original form of the routine MEANSD produced the wrong
result for the standard deviation; thanks to Robert Williams for pointing this out}.

This program has several new statement forms. 

The \verb?CHARACTER? statement declares that the variable {\tt FNAME} is
to hold a 
string of 50 characters: this should be long enough for the file-names 
used by most operating systems. 

The \verb?REAL? statement declares an array X with 1000 elements 
numbered from {\tt X(1)} to {\tt X(1000)}. 

The \verb?READ? statement uses a format item {\tt A} which is needed to
read 
in a character string: A originally stood for ``alpha-numeric''. 

The \verb?OPEN? statement then assigns I/O unit number one (any small 
integer could have been used) to the file.  This unit number is 
needed in subsequent input/output statements.  The item {\tt STATUS='OLD'} 
is used to specify that the file already exists. 

The \verb?IF? statement is a special form which can replace an IF-block 
where it would only contain one statement: its effect is to stop the 
program running if the array would not be large enough. 

The \verb?READ? statement which follows it has a special form known 
as an implied-{\tt DO}-loop: this reads all the numbers from the file in 
to successive elements of the array {\tt X} in one operation.  

The \verb?CALL? statement corresponds to the {\tt SUBROUTINE} statement 
in the same way that a function reference corresponded to a 
{\tt FUNCTION} statement.  The difference is that the arguments {\tt X}
and 
{\tt NPTS} transfer information into the subroutine, whereas {\tt AVG} and 
{\tt SD} return information from it.  The direction of transfer is 
determined only by the way the dummy arguments are used within 
the subroutine.  An argument can be used to pass information in 
either direction, or both. 

The \verb?INTEGER? statement is, as before, not really essential but it is 
good practice to indicate clearly the data type of every procedure 
argument.  

The \verb?REAL? statement declares that {\tt X} is an array but uses a
special 
option available only to dummy arguments: it uses another 
argument, {\tt NPTS}, to specify its size and makes it an adjustable 
array.  Normally in Fortran array bounds must be specified by 
constants, but the rules are relaxed for arrays passed into 
procedures because the actual storage space is already allocated in 
the calling program unit; the {\tt REAL} statement here merely specifies 
how many of the 1000 elements already allocated are actually to be 
used within the subroutine.  

The rest of the subroutine uses a loop to accumulate the sum of the 
elements in SUM, and the sum of their squares in SUMSQ.  It then 
computes the mean and standard deviation using the usual 
formulae, and returns these values to the main program, where they 
are printed out. 

\section{Fortran in Practice} 

This section describes the steps required to turn a Fortran program 
from a piece of text into executable form.  The main operation is 
that of translating the original Fortran source code into the 
appropriate machine code.  On a typical Fortran system this is 
carried out in two separate stages.  This section explains how this 
works in more detail. 

These descriptions differ from those in the rest of the book in two 
ways.  Firstly, it is not essential to understand how a Fortran 
system works in order to use it, just as you do not have to know 
how an internal combustion engine works in order to drive a car. 
But, in both cases, those who have some basic understanding of 
the way in which the machine works find it easier to get the best 
results.  This is especially true when things start to go wrong -- and 
most people find that things go wrong all too easily when they start 
to use a new programming language.  

Secondly the contents of this section are much more system-dependent than
all the others in the book.  The Fortran Standard only specifies what a
Fortran program should do when it is executed, it has nothing directly to
say about the translation process.  In practice, however, nearly all
Fortran systems work in much the same way, so there should not be too
many differences between the ``typical'' system described here and the one
that you are actually using.  Regrettably the underlying similarities are
sometimes obscured by differences in the terminology that different
manufacturers use.

\subsection{The Fortran System} 

The two main ways of translating a program into machine code are 
to use an interpreter or a compiler. 

An interpreter is a program which stays in control all the while the
program is running.  It translates the source code into machine code one
line at a time and then executes that line immediately.  It then goes on
to translate the next, and so on.  If an error occurs it is usually
possible to correct the mistake and continue running the program from the
point at which it left off.  This can speed up program development
considerably.  The main snag is that all non-trivial programs involve
forms of repetition, such as loops or procedure calls.  In all these
cases the same lines of source code are translated into machine code over
and over again.  Some interpreters are clever enough to avoid doing all
the work again but the overhead cannot be eliminated entirely.

The compiler works in an entirely different way.  It is an 
independent program which translates the entire source code into 
machine code at once.  The machine code is usually saved on a 
file, often called an executable image, which can then be run 
whenever it is needed.  Because each statement is only translated 
once, but can be executed as many times as you like, the time take 
by the translation process is less important.  Many systems provide 
what is called an optimising compiler which takes even more 
trouble and generates highly efficient machine code; optimised 
code will try to make the best possible use of fast internal registers 
and the compiler will analyse the source program in blocks rather 
than one line at a time.  As a result, compiled programs usually run 
an order of magnitude faster than interpreted ones.  The main 
disadvantage is that if the program fails in any way, it is necessary 
to edit the source code and recompile the whole thing before 
starting again from the beginning.  The error messages from a 
compiled program may also be less informative than those from an 
interpreter because the original symbolic names and line numbers 
may not be retained by the compiler.  

Interpreters, being more ``user-friendly'', are especially suitable for 
highly interactive use and for running small programs produced by 
beginners.  Thus languages like APL, Basic, and Logo are usually 
handled by an interpreter.  Fortran, on the other hand, is often used 
for jobs which consume significant amounts of computer time: in 
some applications, such as weather forecasting, the results would 
simply be of no use if they were produced more slowly.  The speed 
advantage of compilers is therefore of great importance and in 
practice almost all Fortran systems use a compiler to carry out the 
translation. 

\subsubsection*{Separate Compilation} 

The principal disadvantage of a compiler is the necessity of re-compiling
the whole program after making any alteration to it, no matter how small.
Fortran has partly overcome this limitation by allowing program units to
be compiled separately; these compiled units or modules are linked
together afterwards into an executable program.

A Fortran compiler turns the source code into what is usually 
called object code: this contains the appropriate machine-code 
instructions but with relative memory addresses rather than 
absolute ones.  All the program units can be compiled together, or 
each one can be compiled separately.  Either way a set of object 
modules is produced, one from each program unit.  The second 
stage, which joins all the object modules together, is usually 
known as linking, but other terms such as loading, link-editing, 
and task-building are also in use.  The job of the linker is to collect 
up all these object modules, allocate absolute addresses to each 
one, and produce a complete executable program, also called an 
executable image. 

The advantage of this two-stage system is that if changes are made 
to just one program unit then only that one has to be re-compiled.  
It is, of course, necessary to re-link the whole program.  The 
operations which the linker performs are relatively simple so that 
linkers ought to be fast.  Unfortunately this is not always so, and 
on some systems it can take longer to link a small program than to 
compile it.  

 
\subsection{Creating the Source Code} 

The first step after writing a program is to enter it into the 
computer: these files are known as the source code.  Fortran 
systems do not usually come with an editor of their own: the 
source files can be generated using any convenient text editor or 
word processor. 

Many text editors have options which ease the drudgery of entering
Fortran statements.  On some you can define a single key-stroke to skip
directly to the start of the statement field at column 7 (but if the
source files are to conform to the standard this should work by inserting
a suitable number of spaces and not a tab character).  An even more
useful feature is a warning when you cross the right-margin of the
statement field at column 72.  Most text editors make it easy to delete
and insert whole words, where a word is anything delimited by spaces.  It
helps with later editing, therefore, to put spaces between items in
Fortran statements.  This also makes the program more readable.

Most programs will consist of several program units: these may go 
on separate files, all on one file, or any combination.  On most 
systems it is not necessary for the main program unit to come first.  
When first keying in the program it may seem simpler to put the 
whole program on one file, but during program development it is 
usually more convenient to have each program unit on a separate 
file so that they can be edited and compiled independently.  It 
minimises confusion if each source file has the same name as the 
(first) program unit that it contains. 

\subsubsection*{{\tt INCLUDE} Statements} 

Many systems provide a pseudo-statement called {\tt INCLUDE} (or 
sometimes {\tt INSERT}) which inserts the entire contents of a separate 
text file into the source code in place of the {\tt INCLUDE} statement.  
This feature can be particularly useful when the same set of 
statements, usually specification statements, has to be used in 
several different program units.  Such is often the case when 
defining a set of constants using {\tt PARAMETER} statements, or 
when declaring common blocks with a set of {\tt COMMON} 
statements.  {\tt INCLUDE} statements reduce the key-punching effort 
and the risk of error.  Although non-standard, {\tt INCLUDE} 
statements do not seriously compromise portability because they 
merely manipulate the source files and do not alter the source code 
which the compiler translates. 

\subsection{Compiling} 

The main function of a Fortran compiler is to read a set of source 
files and write the corresponding set of object modules to the 
object file.   

Most compilers have a number of switches or options which can 
be set to control how the compiler works and what additional 
output it produces.  Some of the more useful ones, found on many 
systems, are described below. 
\begin{itemize} 
\item Almost all compilers can produce a listing file: a text file 
containing a copy of the source code, with the lines numbered, and 
with error messages and other useful information attached.  A list 
of all the symbolic names and labels used in the program unit is 
often provided: this should be checked for unexpected entries as 
they may be the result of spelling mistakes.   
\item An even more useful addition to the listing is a cross-reference 
table: this lists every place that each symbolic name has 
been used.  Good compilers indicate which names have only been 
used once as these often indicate a programming mistake.  
\item Another widely available option is the detection of syntax 
which does not conform to the Fortran Standard: this helps to 
ensure program portability.  
\item Often it is possible to choose the optimization level. During 
program development a low level of optimization should be 
selected if this makes the compiler run faster; it may improve the 
error detection.  Highly optimised machine code may execute 
faster but if the source code lines are rearranged error messages 
may be less helpful.  
\item Many systems allow additional code to be included which 
check for errors at run-time.  Errors such as over-running the 
bounds of an array or a character string, or arithmetic over-flow 
can usually be trapped.  Such errors are not uncommon, so this 
assistance is very valuable.  Some programming manuals suggest 
that these options should only be selected during program 
development and switched-off thereafter in the interests of speed.  
This is rather like wearing seat-belts in the car only while you are 
learning to drive and ignoring them as soon as you are allowed out 
on the motorway.  Run-time checks do not usually reduce the 
execution speed noticeably.  
\end{itemize} 

 
\subsection{Linking} 

At its simplest, the linker just takes the set of object modules 
produced by the compiler and links them all together into an 
executable image.  One of these modules must correspond to the 
main program unit, the other modules will correspond to 
procedures and to block data subprogram units. 

It often happens that a number of different programs require some 
of the same computations to be carried out.  If these calculations 
can be turned into procedures and linked into each program it can 
save a great deal of programming effort, especially in the long run.  
This ``building block'' approach is particularly beneficial for large 
programs.  Many organisations gradually build up collections of 
procedures which become an important software resource.  
Procedures collected in this way tend to be fairly reliable and free 
from bugs, if only because they have been extensively tested and 
de-bugged in earlier applications.   

\subsubsection*{Object Libraries} 

It obviously saves on compilation time if these commonly-used procedures
can be kept in compiled form as object modules. Almost all operating
systems allow a collection of object modules to be stored in an object
library (sometimes known as a pre-compiled or relocatable-code library).
This is a file containing a collection of object modules together with an
index which allows them to be extracted easily.  Object libraries are not
only more efficient but also easier to use as there is only one file-name
to specify to the linker.  The linker can then work out for itself which
modules are needed to satisfy the various {\tt CALL} statements and
function references encountered in the preceding object modules. Object
libraries also simplify the management of a procedure collection and may
reduce the amount of disc space needed.  There are usually simple ways of
listing the contents of an object library, deleting modules from it, and
replacing modules with new versions.

All Fortran systems come with a system library which contains the 
object modules for various intrinsic functions such as {\tt SIN,} {\tt
COS}, 
and {\tt SQRT}.  This is automatically scanned by the linker and does 
not have to be specified explicitly. 

Software is often available commercially in the form of procedure 
libraries containing modules which may be linked into any Fortran 
program.  Those commonly used cover fields such as statistics, 
signal processing, graphics, and numerical analysis.  

\subsubsection*{Linker Options} 

The order of the object modules supplied to the linker does not 
usually matter although some systems require the main program to 
be specified first.  The order in which the library files are searched 
may be important, however, so that some care has to be exercised 
when several different libraries are in use at the same time. 

The principal output of the linker is a single file usually called the 
executable image.  Most linkers can also produce a storage map 
showing the location of the various modules in memory.  
Sometimes other information is provided such as symbol tables 
which may be useful in debugging the program. 

 
\subsection{Program Development} 
The program development process consists of a number of stages 
some of which may have to be repeated several times until the end 
product is correct: 
\begin{enumerate} 
\item Designing the program and writing the source-code text. 
\item Keying in the text to produce a set of Fortran source files. 
\item Compiling the source code to produce a set of object 
modules. 
\item Linking the object modules and any object libraries into a 
complete executable image. 
\item Running the executable program on some test data and 
checking the results. 
\end{enumerate} 

 
The main parts of the process are shown in the diagram below. 
\newpage
\begin{verbatim}
                  Source program
                        |
                        V
                  FORTRAN COMPILER --> (optional) Compiler listing
                        |
                        V
                  Object code
                        |
                        V
Object libraries --> LINKER --> (optional) linker map
(optional)              |
                        V
                  Executable program
\end{verbatim}

\subsubsection*{Handling Errors } 

Things can go wrong at almost every stage of the program 
development process for a variety of reasons, most of them the 
fault of the programmer.  Naturally the Fortran system cannot 
possibly detect all the mistakes that it is possible for human 
programmers to make.  Errors in the syntax of Fortran statements 
can usually be detected by the compiler, which will issue error 
messages indicating what is wrong and, if possible, where.   

Other mistakes will only come to light at the linking stage.  If, 
for example, you misspell the name of a subroutine or function the 
compiler will not be able to detect this as it only works on one 
program unit at a time, but the linker will say something like 
``unsatisfied external reference''.  This sort of message will 
sometimes appear if you misspell the name of an array since array 
and function references can have the same form.  

Most errors that occur at run-time are the result of programmer 
error, or at least failure to anticipate some failure mode.  Even 
things like division by zero or attempting to access an array 
element which is beyond its declared bounds can be prevented by 
sufficiently careful programming.   

There is, however, a second category of run-time error which no 
amount of forethought can avoid: these nearly all involve the 
input/output system.  Examples include trying to open a file which 
no longer exists, or finding corrupted data on an input file.  For 
this reason most input/output errors can be trapped, using the 
{\tt IOSTAT=} or {\tt ERR=} keywords in any I/O statement.  There is no 
way of trapping run-time errors in any other types of statement in 
Standard Fortran. 

But, just because a program compiles, links, and runs without 
apparent error, it is not safe to assume that all bugs have been 
eliminated.  There are some types of mistake which will simply 
give you the wrong answer.  The only way to become confident 
that a program is correct is to give it some test data, preferably for 
a case where the results can be calculated independently.  When a 
program is too elaborate for its results to be predictable it should 
be split into sections which can be checked separately. 

\section{Program Structure and Layout} 

This section explains the rules for program construction and text 
layout.  A complete Fortran program is composed of a number of 
separate program units.  Each of these can contain both statements 
and comment lines.  Statements are formed from items such as 
keywords and symbolic names.  These in turn consist of characters.  

 
\subsection{The Fortran Character Set} 

The only characters needed to write Fortran programs, and the only 
ones that should be used in portable software, are those in the 
Fortran character set:  

\begin{tabular}{ll} 
the 26 upper-case letters  &     {\tt A B C} ... {\tt X Y Z} \\ 
the 10 digits &                  {\tt 0 1 2 3 4 5 6 7 8 9} \\ 
and 13 special characters: & \\ 
\end{tabular} 
\begin{center} 
\begin{tabular}{llll} 
{\tt +} & plus                 &  {\tt -} & minus      \\ 
{\tt *} & asterisk             &  {\tt /} & slash \\ 
\verb+ +& blank                &  {\tt =} & equals \\ 
{\tt (} & left parenthesis     &  {\tt )} & right parenthesis \\ 
{\tt .} & decimal point        &  {\tt ,} & comma \\ 
{\tt '} & apostrophe           &  {\tt :} & colon \\ 
{\tt \$} & currency symbol \\ 
\end{tabular} 
\end{center} 
Although this character set is somewhat limited, it is at least 
universally available, which helps to make programs portable. What 
suffers is program legibility: lower-case letters are absent and it is 
necessary to resort to ugly constructions like {\tt .LT.} and {\tt .GT.} 
to represent operators like \verb+<+ and \verb+>+.  Some of the special 
characters, such as the asterisk and parentheses, are also rather 
overloaded with duties. 

\subsubsection*{Blanks} 

The blank, or space, character is ignored everywhere in Fortran 
statements (except within character constants, which are enclosed 
in a pair of apostrophes).  Although you do not need to separate 
items in Fortran statements with blanks, it is good practice to 
include a liberal helping of them since they improve legibility and 
often simplify editing.  The only limitation (as explained below) is 
that statement lines must not extend beyond column 72. 

\subsubsection*{Currency Symbol} 

The currency symbol has no fixed graphic representation: it appears on 
most systems as the dollar ``{\tt \$}'', but other forms such as 
``$\pounds$'' equally valid.  This variability does not matter much 
because the currency symbol is not actually needed in Standard Fortran 
syntax. 

\subsubsection*{Other Characters} 

Most computers have a character set which includes many other 
printable characters, for example lower-case letters, square 
brackets, ampersands and per-cent signs.  Any printable characters 
supported by the machine may be used in comment lines and 
within character constants.  

The Fortran character set does not include any carriage-control 
characters such as tab, carriage-return, or form-feed, but formatted 
{\tt WRITE} statements can be used to produce paginated and tabulated 
output files. 

Fortran programs can process as data any characters supported by 
the local hardware.  The Fortran Standard is not based on the use 
of any particular character code but it requires its character 
comparison functions to use the collating sequence of the 
American Standard Code for Information Interchange (ASCII).  
Further details are given in section 7.6. 

\subsection{Statements and Lines} 

The statement is the smallest unit of a Fortran program, 
corresponding to what is called an instruction or command in some 
programming languages.  Most types of statement start with a 
keyword which consists of one (or sometimes two) English words 
describing the main action of that statement, for example: {\tt READ}, 
{\tt DO,} {\tt ELSE IF}, {\tt GO TO}.  Since blanks are ignored, compound 
keywords can be written either as one word or two: {\tt ELSEIF} or 
{\tt ELSE IF} (but the latter seems easier to read). 

The rules for statement layout are an unfortunate relic of punched-card
days.  Every statement must start on a new line and each line is divided
into three fixed fields:
\begin{itemize}
\item  columns 1 to 5 form the label field, 
\item  column 6 forms the continuation marker field, 
\item  columns 7 to 72 form the statement field. 
\end{itemize} 

Since labels and continuation markers are only needed on a few 
statements, the first six columns of most lines are left blank.  

Any characters in column 73 or beyond are likely to be ignored 
(columns 73 to 80 were once used to hold card sequence numbers).  
This invisible boundary after column 72 demands careful attention 
as it can have very pernicious effects: it is possible for a statement 
to be truncated at the boundary but still be syntactically correct, so 
that the compiler will not detect anything wrong. 

\subsubsection*{Continuation Lines} 

Statements do not have to fit on a single line.  The initial line of 
each statement should have a blank in column 6, and all 
subsequent lines, called continuation lines, must have some 
character other than blank (or the digit zero) in column 6.  Up to 
19 continuation lines are allowed, i.e.\ 20 in total.  The column 
layout needed with continuation lines is illustrated here:  
\begin{verbatim}
columns 
123456789... 
      IF(REPLY .EQ. 'Y' .OR. REPLY .EQ. 'y' .OR. 
     $   REPLY .EQ. 'T' .OR. REPLY .EQ. 't') THEN 
\end{verbatim} 
The currency symbol makes a good continuation marker since if 
accidentally misplaced into an adjacent column it would be almost 
certain to produce an error during compilation. 

The {\tt END} statement is an exception to the continuation rule: it may 
not be followed by continuation lines and no other statement may 
have an initial line which just contains the letters ``END''.  Neither 
rule causes many problems in practice. 

Programs which make excessive use of continuation lines can be 
hard to read and to modify: it is generally better, if possible, to 
divide a long statement into several shorter ones.  

\subsubsection*{Comment Lines} 

Comments form an important part of any computer program even 
though they are completely ignored by the compiler: their purpose 
is to help any human who has to read and understand the program 
(such as the original programmer six months later). 

Comments in Fortran always occupy a separate line of text; they 
are marked by an asterisk in the first column.   For example: 
\begin{verbatim}
*Calculate the atmospheric refraction at PRESS mbar. 
       REF = PRESS * (0.1594 + 1.96E-2 * A + 2E-5 * A**2) 
*Correct for the temperature T (Celsius) 
       TCOR = (273.0 + T) * (1.0 + 0.505 * A + 8.45E-2 * A**2) 
\end{verbatim} 
A comment may appear at any point in a program unit except after 
the {\tt END} statement (unless another program unit follows, in which 
case it will form the first line of the next unit).  A completely blank 
line is also allowed and is treated as a blank comment.  This means 
that a blank line is not actually permitted after the last {\tt END} 
statement of a program. 

There is no limit to the number of consecutive comment lines 
which may be used; comments may also appear in the middle of a 
sequence of continuation lines.  To conform to the Fortran 
Standard, comment lines should not be over 72 characters long, 
but this rule is rarely enforced. 

Comments may include characters which are not in the Fortran 
character set.  It helps to distinguish comments from code if they 
are mainly written in lower-case letters (where available).  It is also 
good practice for comments to precede the statements they 
describe rather than follow them.   

Some systems allow end-of-line comments, usually prefaced by an 
exclamation mark: this is not permitted by the Fortran standard.  
For compatibility with Fortran66 comments can also be denoted 
by the letter C in column 1.  
  
\subsubsection*{Statement Labels} 

A label can be attached to any statement.  There are three reasons 
for using labels:  
\begin{itemize} 
\item  the end of each {\tt DO}-loop is specified by a label given in 
the {\tt DO} statement;  
\item  every {\tt FORMAT} statement must have a label attached as 
that is how {\tt READ} and {\tt WRITE} statements refer to it; 
\item  any executable statement may have a label attached so 
that control may be transferred to it, for example by a 
{\tt GO TO} statement. 
\end{itemize} 

Example: 
\begin{verbatim}
*Read numbers from input file until it ends, add them up. 
      SUM = 0.0 
100   READ(UNIT=IN, FMT=200, END=9999) VALUE 
200   FORMAT(F20.0) 
      SUM = SUM + VALUE 
      GO TO 100 
9999  WRITE(UNIT=*, FMT=*)'SUM of values is', SUM 
\end{verbatim} 
Each label has the form of an unsigned integer in the range 1 to 
99999.  Blanks and leading zeros are ignored.  The numerical 
value is irrelevant and cannot be used in a calculation at all.  The 
label must appear in columns 1 to 5 of the initial line of the 
statement.  In continuation lines the label field must be blank. 

A label must be unique within a program unit but labels in 
different program units are quite independent.  Although any 
statement may be labelled, it only makes sense to attach a label to 
a {\tt FORMAT} statement or an executable statement, since there is no 
way of using a label on any other type of statement.  

Statement labels are unsatisfactory because nearly all of them mark a
point to which control could be transferred from elsewhere in the program
unit.  This makes it much harder to understand a program with many
labelled statements.  Unfortunately at present one cannot avoid using
labels altogether in Fortran.  If labels are used at all they should
appear in ascending order and preferably in steps of 10 or 100 to allow
for changes.  Labels do not have to be right-justified in the label
field.

\subsection{Program Units} 

A complete executable program consists of one or more program 
units.  There is always one (and only one) main program unit: this 
starts with a {\tt PROGRAM} statement.  There may also be any 
number of subprogram units of any of the three varieties:  
\begin{itemize} 
\item  subroutine subprograms: these start with a 
{\tt SUBROUTINE} statement  
\item  function subprograms, also known as external functions: 
these start with a {\tt FUNCTION} statement  
\item  block data subprograms: these start with a {\tt BLOCK 
DATA} statement.   
\end{itemize} 

Subroutines and external functions are known collectively as 
external procedures; block data subprograms are not procedures 
and are used only for the special purpose of initialising the 
contents of named common blocks. 

Every program unit must end with an {\tt END} statement. 

\subsubsection*{Procedures} 

Subroutines and external functions are collectively known as 
external procedures: they are described in full in section 9.  A 
procedure is a self-contained sequence of operations which can be 
called into action on demand from elsewhere in the program.  
Fortran supplies a number of intrinsic functions such as {\tt SIN, COS, 
TAN, MIN, MAX,} etc.  These are procedures which are 
automatically available when you need to use them in expressions.  
External functions can be used in similar ways: there may be any 
number of arguments but only one value is returned via the 
function name.  

The subroutine is a procedure of more general form: it can have 
any number of input and output arguments but it is executed only 
in response to an explicit {\tt CALL} statement.   

Procedures may call other procedures and so on, but a procedure 
may not call itself directly or indirectly; Fortran does not support 
recursive procedure calls.  

Most Fortran systems allow procedures to be written in other 
languages and linked with Fortran modules into an executable 
program.  If the procedure interface is similar to that of a Fortran 
subroutine or function this presents no problem.  

The normal way to transfer information from one program unit to 
another is to use the argument list of the procedure as described in 
section 9, but it is also possible to use a common block: a shared 
area of memory.  This facility, which is less modular, is described 
in section 12.  

\subsection{Statement Types and Order} 

Fortran statements are either executable or non-executable.  The 
compiler translates executable statements directly into a set of 
machine code instructions.  Non-executable statements are mainly 
used to tell the compiler about the program; they are not directly 
translated into machine code.  The {\tt END} statement is executable 
and so are all those in the lowest right-hand box of the table below; 
all other statements are non-executable. 

The general order of statements in a program unit is: 
\begin{itemize} 

\item Program unit header ({\tt PROGRAM, SUBROUTINE, FUNCTION}, or  
{\tt BLOCK DATA} statement) 
\item Specification statements 
\item Executable statements  
\item {\tt END} statement. 
\end{itemize} 

The table below shows shows the complete statement ordering rules: the 
statements listed in each box can be intermixed with those in boxes 
on the same horizontal level (thus {\tt PARAMETER} statements can 
be intermixed with {\tt IMPLICIT} statements) but those in boxes 
separated vertically must appear in the proper order in each 
program unit (thus all statement functions must precede all 
executable statements). 

\begin{center} 
\begin{tabular}{|l|l|l|} 
\hline 
\multicolumn{3}{|c|}{{\tt PROGRAM, FUNCTION, SUBROUTINE, BLOCK DATA}} \\ 
\hline 
 & & {\tt IMPLICIT} \\ 
\cline{3-3}           
 & {\tt PARAMETER} & {\em Type statements:} \\ 
 &           & {\tt INTEGER, REAL, DOUBLE PRECISION,} \\ 
 &           & {\tt COMPLEX, LOGICAL, CHARACTER}  \\ 
 &           & {\em Other specification statements:}    \\ 
 &           & {\tt COMMON, DIMENSION, EQUIVALENCE,}    \\ 
 &           & {\tt EXTERNAL, INTRINSIC, SAVE}           \\ 
\cline{2-3} 
{\tt FORMAT} &     & {\em Statement function statements} \\ 
\cline{3-3} 
 & {\tt DATA}      & {\em Executable statements:} \\ 
 &           & {\tt BACKSPACE, CALL, CLOSE, CONTINUE, DO,} \\ 
 &           & {\tt ELSE, ELSE IF, END IF, GO TO, IF,} \\ 
 &           & {\tt INQUIRE, OPEN, READ, RETURN, REWIND,} \\ 
 &           & {\tt STOP, WRITE,} {\em assignment statements}. \\ 
\hline 
\multicolumn{3}{|c|}{{\tt END}} \\ 
\hline 
\end{tabular} 
\end{center} 

\subsubsection*{Execution Sequence} 

A program starts by executing the first executable statement of the 
main program unit.  Execution continues sequentially unless 
control is transferred elsewhere: an {\tt IF} or {\tt GO TO} statement,
for 
example, may transfer control to another part of the same program 
unit, whereas a {\tt CALL} statement or function reference will transfer 
control temporarily to a procedure. 

A program continues executing until it reaches a {\tt STOP} statement 
in any program unit, or the {\tt END} statement of the main program 
unit, or until a fatal error occurs.  When a program terminates 
normally (at {\tt STOP} or {\tt END}) the Fortran system closes any files 
still open before returning control to the operating system.  But 
when a program is terminated prematurely files, especially output 
files, may be left with incomplete or corrupted records.   

\subsection{Symbolic Names} 

Symbolic names can be given to items such as variables, arrays, 
constants, functions, subroutines, and common blocks.  All 
symbolic names must conform to the following simple rule: the 
first character of each name must be a letter, this may be followed 
by up to five more letters or digits.  Here are some examples of 
valid symbolic names: 
\begin{verbatim}
          I  MATRIX  VOLTS  PIBY4  OLDCHI  TWOX  R2D2  OUTPUT 
\end{verbatim} 
And here are some names which do not conform to the rules: 
\begin{center} 
\begin{tabular}{ll} 
{\tt COMPLEX} & (too many letters) \\ 
{\tt MAX\_EL}  & (underscore is not allowed) \\ 
{\tt 2PI}     & (starts with a digit) \\ 
{\tt Height}  & (lower-case letters are not allowed).\\ 
\end{tabular} 
\end{center} 
It is best to avoid using digits in names unless the meaning is clear, 
because they are often misread.  The digit {\tt 1} is easily confused 
with the letter {\tt I}, similarly {\tt 0} looks much like the letter 
{\tt O} on many devices. 

The six-character limit on the length of a symbolic name is one of the 
most unsatisfactory features of Fortran: programs are much harder to 
understand if the names are cryptic acronyms or abbreviations, but with 
only six characters there is little choice. Although many systems do not 
enforce the limit (and Fortran90 allows names up to 31 characters long), 
at present the only way to ensure software portability is to keep to it 
strictly. There is a further problem with items which have an associated 
data type (constants, variables, arrays, and functions).  Unless the data 
type is declared explicitly in a type statement, it is determined by the 
initial letter of the name.  This may further restrict the choice. 

\subsubsection*{Scope of Symbolic Names} 

Symbolic names which identify common blocks and program units of all 
types are global in scope, i.e.\ their name must be unique in the entire 
executable program.  Names identifying all other items (variables, 
arrays, constants, statement functions, intrinsic functions, and all 
types of dummy argument) are local to the program unit in which they are 
used so that the same name may be used independently in other program 
units. 

To see the effect of these rules here is a simple example.  Suppose your 
program contains a subroutine called SUMMIT.  This is a global name so it 
cannot be used as the name of global item (such as an external procedure 
or a common block) in the same executable program.  In the SUMMIT 
subroutine and in any other program unit which calls it the name cannot 
be used for a local item such as a variable or array.  In all other 
program units, however, including those which call SUMMIT indirectly, the 
name SUMMIT can be used freely e.g.\ for a constant, variable, or array. 

The names of global items need to be chosen more carefully because it is 
harder to alter them at a later stage; it can be difficult to avoid name 
clashes when writing a large program or building a library of procedures 
unless program unit names are allocated systematically.  It seems 
appropriate for procedures to have names which are verb-like.  If you 
find it difficult to devise sensible procedure names remember that the 
English language is well stocked with three and four-letter verbs which 
form a good basis, for example: DO, ASK, GET, PUT, TRY, EDIT, FORM, LIST, 
LOAD, SAVE, PLOT.  By combining a word like one of these with one or two 
additional letters it is possible to make up a whole range of procedure 
names. 

\subsubsection*{Reserved Words} 

In most computer languages there is a long list of words which are 
reserved by the system and cannot be used as symbolic names: Cobol 
programmers, for example, have to try to remember nearly 500 of them.  In 
Fortran there are no reserved words.  Some Fortran keywords (for instance 
{\tt DATA, END}, and {\tt OPEN}) are short enough to be perfectly valid 
symbolic names.  Although it is not against the rules to do this, it can 
be somewhat confusing. 

The names of the intrinsic functions (such as {\tt SQRT, MIN, CHAR}) are, 
technically, local names and there is nothing to prevent you using them 
for your own purposes, but this is not generally a good idea either.  For 
example, if you choose to use the name {\tt SQRT} for a local variable 
you will have more difficulty in computing square-roots in that program 
unit.  It is even more unwise to use the name of an intrinsic function as 
that of an external procedure because in this case the name has to be 
declared in an {\tt EXTERNAL} statement in every program unit in which it 
is used in this way. 

\subsection{\texttt{PROGRAM} Statement} 

The {\tt PROGRAM} statement can only appear at the start of the main 
program unit.  Its only function is to indicate what type of program unit 
it is and to give it symbolic name.  Although this name cannot be used 
anywhere else in the program, it may be used by the Fortran system to 
identify error messages etc.  The general form is simply:\\ 
\verb+      PROGRAM+ {\em name}\\ 
Where {\em name} is a symbolic name.  This name is global in scope and 
may not be used elsewhere in the main program nor as a global name in any 
other program unit.  For compatibility with Fortran66 the {\tt PROGRAM} 
statement is optional.  This can have unexpected effects: if you forget 
use a {\tt SUBROUTINE} or {\tt FUNCTION} statement at the start of a 
procedure the compiler will assume it to be a (nameless) main program 
unit.  Since this will normally result in two main program units, the 
linker is likely to detect the mistake. 

\subsection{\texttt{END} Statement} 

The {\tt END} statement must appear as the last statement of every 
program unit.  It simply consists of the word:\\ 
\verb+      END+\\ 
which may not be followed by any continuation lines (or comments).  The 
{\tt END} statement is executable and may have a label attached.  If an 
{\tt END} statement is executed in a subprogram unit, i.e.\ a procedure, 
it returns control to the calling unit; if an {\tt END} statement is 
executed in the main program it closes any files which are open, stops 
the program, and returns control to the operating system. 

\section{Constants, Variables, and Arrays} 

This section deals with the data-storage elements of Fortran: constants, 
variables, and arrays.  These all possess an important property called 
data type.  The data type of an item determines what sort of information 
it holds and the operations that can be performed on it. 

\subsection{Data Types} 

All the information processed by a digital computer is held internally in 
the form of binary digits or bits.  Suitable collections of bits can be 
used to represent many different types of data including numbers and 
strings of characters.  It is not necessary to know how the information 
is represented internally in order to write Fortran programs, only that 
there is a different representation for each type of data.  The data type 
of each item also determines what operations can be carried out on it: 
thus arithmetic operations can be carried out on numbers, whereas 
character strings can be split up or joined together.  The data type of 
each item is fixed when the program is written. 

Fortran, with its emphasis on numerical operations, has four data types 
just for numbers.  These are collectively known as the arithmetic data 
types.  Arithmetic expressions can include mixtures of data types and, in 
most cases, automatic type conversions are provided.  In other 
circumstances, however, especially in procedure calls, there is no 
provision for automatic type conversion and it is essential for data 
types to match exactly. 

The range and precision of the arithmetic data types are not specified by 
the Standard: typical values are indicated below, but the only way to be 
sure is to check the manuals provided with your own Fortran system. 

Several intrinsic functions are available to convert from one data type 
to another.  Conversion from character strings to numbers and vice-versa 
can be complicated; these are best carried out with the internal file 
{\tt READ} and {\tt WRITE} statements (see section 10.3). 

There are, as yet, no user-defined or structured data types in Fortran. 

\subsubsection*{Standard Data Types} 

The table below summarises the properties of the six data types 
provided in Standard Fortran: 

 
\begin{tabular}{p{1.1in}p{4.3in}} 
\hline 
Data type & Characteristics  \\ 
\hline 
\\ 
Integer                  & Whole numbers stored exactly. \\ 

Real                     & Numbers, which may have fractional parts, 
                         stored using a floating-point representation 
                         with limited precision. \\ 

Double Precision         & Similar to real but with greater precision. \\ 

Complex                  & Complex numbers: stored as an ordered 
                              pair of real numbers. \\ 

Logical                  & A Boolean value, i.e.\ one which is either 
                              true or false. \\ 

Character                & A string of characters of fixed length. \\ 
\hline 
\end{tabular} 

The first four types (integer, real, double precision, and complex) 
all hold numerical information and are collectively known as 
arithmetic data types.   

\subsubsection*{Integer Type} 

The integer data type can only represent whole numbers but they 
are stored exactly in all circumstances.  Integers are often used to 
count discrete objects such as elements of an array, characters in 
a string, or iterations of a loop. 

The range of numbers covered by the integer type is system-dependent.
The majority of computers use 32 bits for their integer arithmetic (1 bit
for the sign and 31 for the magnitude) giving a number range of
$-2,147,483,648$ to $+2,147,483,647$.  Some systems have an even larger
integer range but a few very small systems only allow 16-bit integer
arithmetic so that their integer range is only $-32,768$ to $+32,767$.

\subsubsection*{Real Type} 

Most scientific applications use the real data type more than anything
else.  Real values are stored internally using a floating-point
representation which gives a larger range than the integer type but the
values are not, in general, stored exactly.  Both the range and precision
are machine dependent.

In practice most machines use at least 32 bits to store real numbers. 
Many systems now use the IEEE Standard representation: for 32-bit 
numbers this gives a precision of just over 7 decimal digits and allows a 
number range from around $10^{-38}$ to just over $10^{+38}$.  This can be 
something of a limitation because there are many types of calculation, 
especially in physics and astronomy, which lead to numbers in excess of 
$10^{40}$.  Some computers designed expressly for scientific work, 
sometimes called ``super-computers'', allocate 64 bits for real numbers so 
that the numerical precision is much larger; the range is often larger as 
well.  On such machines it is rarely necessary to use the double 
precision type. 

\subsubsection*{Double Precision Type} 

Double precision is an alternative floating-point type.  The Fortran 
Standard only specifies that it should have greater precision than the 
real type but in practice, since the double precision storage unit is 
twice the size, it is safe to assume that the precision is at least 
doubled.  The number range may, however, be the same as that for real 
type. 

Although double precision values occupy twice as much memory 
as real (or integer) values, computations on them do not 
necessarily take twice as long. 

\subsubsection*{Complex Type} 

The complex data type stores two real values as a single entity.  
There is no double precision complex type in Standard Fortran.   

Complex numbers arise naturally when extracting the roots of 
negative numbers and are used in many branches of mathematics, 
physics, and engineering.  A complex number is often represented 
as $(A + iB)$, where $A$ and $B$ are the real and imaginary parts 
respectively and $i^{2} = -1$.  Electrical engineers, having used the 
letter $i$ to represent current, use the notation $(A + jB)$ instead.  

Although the rules for manipulating complex numbers are 
straight-forward, it is convenient to have the Fortran system to do 
the work.  It is usually more efficient as well, because the computer 
can use its internal registers to store the intermediate products in 
complex arithmetic.  Exponentiation and the four regular 
arithmetic operators can be used on complex values, and various 
intrinsic functions are also provided such as square-root, 
logarithms, and the trigonometric functions. 

\subsubsection*{Logical Type} 

The logical data type is mainly used in conjunction with {\tt IF} 
statements which select a course of action according to whether 
some condition is true or false.  A logical variable (or array 
element) may be used to store such a condition value for future 
use.  Logical variables and arrays are also useful when dealing 
with two-valued data such as whether a person is male or female, 
a file open or closed, power on or off, etc.  

Some programmers seem reluctant to use logical variables and 
arrays because they feel that it must be inefficient to use an entire 
computer word of perhaps 32 bits to store just one bit of 
information.  In fact the extra code needed to implement a more 
efficient data packing scheme usually wastes more memory than 
the logical variables would have occupied.  

\subsubsection*{Character Type} 

The character type is unique in that each character item has a length
defined for it: this is the number of characters that it holds. In
general the length is fixed when the item is declared and cannot be
altered during execution.  The only exception to this is for dummy
arguments of procedures: here it is possible for the dummy argument to
acquire the length of the corresponding actual argument.  Using this
facility, general-purpose procedures can be written to operate on
character strings irrespective of their length. In addition, the rules
for character assignment take care of mismatched lengths by truncating or
padding with blanks on the right as necessary.  This means that the
Fortran character type has many of the properties of a genuine
variable-length character-handling system.

The maximum length of a character item is system-dependent: it is 
safe to assume that all systems will allow strings of up to 255 
characters, a length limit of 32767 (or even more) is quite 
common.  The minimum length of a character item is one 
character; empty or null strings are not permitted.  

\subsubsection*{Storage Units} 

Although the Fortran Standard does not specify the absolute 
amount of memory to be allocated to each data type, it does specify 
the relative amounts.  This is not important very often, only when 
constructing unformatted direct-access records or when using 
{\tt COMMON} and {\tt EQUIVALENCE} statements.  The rules are as 
follows: 

\begin{center} 
\begin{tabular}{ll} 
\hline 
Data types & Storage units\\ 
\hline 
integer, real, logical       & 1 {\em numerical} storage unit \\ 
complex, double precision   & 2 {\em numerical} storage units \\ 
character*(N)               & N {\em character} storage units \\ 
\hline 
\end{tabular} 
\end{center} 

In the case of an array the number of storage units must be 
multiplied by the total number of elements in the array. 

The relationship between the numeric and character storage units 
is deliberately undefined because it is entirely system-dependent. 

\subsubsection*{Guidelines} 

It is usually fairly clear which data type to choose for each item in 
a program, though there are some borderline cases among the 
various arithmetic data types.   

When processing data which are inherently integers, such as the number of
seeds which germinate in each plot, or the number of photons detected in
each time interval, it is not always clear whether to use integer or real
arrays to store them.  They both use the same memory space but on some
machines additions and subtractions are faster on integers than on
floating-point numbers. In practice, however, any savings can be
swallowed up in the data type conversions that are usually necessary in
subsequent processing.  The main snag with integers is the limited range;
on some machines integer overflow is not detected whereas floating-point
overflows nearly always produce error messages.

If your machine stores its real variables in 32-bit words then the 
precision of around 1 in $10^{7}$ is likely to be inadequate in some 
applications.  This imprecision is equivalent to an error of several 
pence in a million pounds, or around ten milliseconds in a day.  If 
errors of this order are significant you should consider using the 
double precision type instead.  This will normally reduce the errors 
by at least another factor of $10^{7}$.  Mixing data types increases the 
risks of making mistakes and it is often simpler and safer to use the 
double precision type instead of real throughout the program, even 
though this may use slightly more memory and processor time. 

Although automatic type conversions are provided for the 
arithmetic types in expressions, in other cases such as procedure 
calls it is essential for each actual argument to have the same data 
type as the corresponding dummy argument.  Since program units 
are compiled independently, it is difficult for either the compiler 
or the linker to detect type mismatches in calls to external 
procedures.   

\subsubsection*{Non-standard Data Types} 

Although Standard Fortran only provides the above six data types, many 
systems provide additional ones. You may come across data type names such 
as: {\tt LOGICAL*1}, {\tt INTEGER*2}, {\tt REAL*8}, {\tt COMPLEX*16}, 
etc.  The number after the asterisk indicates the number of bytes of 
storage used for each datum (a byte being a group of 8 bits).  This 
notation has a certain logic but is totally non-standard.  The use of a 
term like {\tt REAL*8} when it is simply a synonym for {\tt DOUBLE 
PRECISION} seems particularly pointless.  There are, of course, 
circumstances when types such as {\tt COMPLEX*16} are necessary but the 
price to be paid is the loss of portability. 

\subsection{Constants} 

A constant has a value which is fixed when the program is written.  
The data type of every constant is evident from its form.  
Arithmetic constants always use the decimal number base: 
Standard Fortran does not support other number bases such as 
octal or hexadecimal. 

Although arithmetic constants may in general have a leading sign 
(plus or minus) there are some circumstances in Fortran when an 
unsigned constant is required.  If the constant is zero then any sign 
is ignored. 

\subsubsection*{Integer Constants} 

The general form of an integer constant is a sign (plus or minus) 
followed by a string of one or more digits.  All other characters 
(except blanks) are prohibited.  If the number is positive the plus 
sign is optional.  Here are some examples of valid integer 
constants:\\ 
\verb?      -100        42       0     +1048576?\\ 
It is easier to read a large number if its digits are marked off in 
groups of three: traditionally the comma (or in some countries the 
dot) is used for this purpose.  The blank can be used in the same 
way in Fortran programs (but not in data files):\\ 
\verb+      -1 000 000 000 +\\ 
Note that this number, although conforming to the rules of Fortran, 
may be too large in magnitude to be stored as an integer on some 
systems.  

\subsubsection*{Real Constants} 

A real constant must contain a decimal point or an exponent (or 
both) to distinguish it from one of integer type.  The letter ``E'' is 
used in Fortran to represent ``times 10 to the power of''.  For 
example, the constant $1.234 \times 10^{-5}$ is written as ``1.234E-5''.  

The most general form of a real constant is: 
\begin{center} 
\begin{tabular}{ccccccc} 
{\em sign} & {\em digits} & {\tt .} & {\em digits} & {\tt E} & {\em sign} 
& {\em digits} \\ 
\multicolumn{2}{c}{{\em --integer-part--}} & 
 \multicolumn{2}{c}{{\em --decimal-part--}} & & 
        \multicolumn{2}{c}{{\em --exponent--}} \\ 
\multicolumn{4}{c}{{\em ---basic-real-constant---}} & 
  \multicolumn{3}{c}{{\em ---exponent-section---}} \\ 
\end{tabular} 
\end{center} 
Both signs are optional; a plus sign is assumed if no sign is 
present.  Leading zeros in the integer-part and in the exponent are 
ignored.  Either the integer part or the decimal part may be omitted 
if it is zero but one or the other must be present.  If the value of the 
exponent is zero the entire exponent section may be omitted 
provided a decimal point is present in the number. 

There is no harm in giving more decimal digits in a real (or double 
precision) constant than the computer can make use of: the value 
will be correctly rounded by the computer and the extra decimal 
places ignored. 

Here are a few examples of valid real constants:\\ 
\verb?   .5      -10.       1E3       +123.456E4   .000001?\\ 
Dangling decimal points, though permitted, are easily overlooked, 
and it is conventional to standardize constants in exponential 
notation so that there is only one digit before the decimal point.  
Using this convention, these values would look like this: \\ 
\verb?   0.5     -10.0     1000.0     1.23456E6    1.0E-6?

\subsubsection*{Double Precision Constants} 

A double precision constant has a similar form to a real constant 
but it must contain an exponent but using the letter ``D'' in place of 
``E'' even if the exponent is zero.  Some examples of double 
precision constants are:\\ 
\verb?    3.14159265358987D0   1.0D-12   -3.652564D+02 ?

\subsubsection*{Complex Constants} 

A complex constant has the form of two real or integer constants 
separated by a comma and enclosed in a pair of parentheses.  The 
first number is the real component and the second the imaginary 
component.  Some examples of valid complex constants are:\\ 
\verb? (3.14,-5.67)      (+1E5,0.125)      (0,0)      (-0.999,2.718E15)?

\subsubsection*{Logical Constants} 

There are only two possible logical constants, and they are 
expressed as: {\tt .TRUE.} and  {\tt .FALSE.} 
The dots at each end are needed to distinguish these special forms 
from the words TRUE and FALSE, which could be used as 
symbolic names. 

\subsubsection*{Character Constants} 

A character constant consists of a string of characters enclosed in 
a pair of apostrophes which act as quotation marks.  Within the 
quoted string any characters available in the character set of the 
machine are permitted; the blank (or space) character is significant 
within character constants and counts as a single character just like 
any other.  Examples of valid character constants are: 
\begin{verbatim}
      'X' 
      '$40 + 15%' 
      'This is a constant including spaces' 
\end{verbatim} 

The apostrophe character can be included in a character constant 
by representing it as two successive apostrophes (with no 
intervening blanks).  This pair of apostrophes only counts as a 
single character for the purposes of computing the length of the 
string.  For example: \verb?'DON''T'? is a constant of length 5. 

\subsection{Specifying Data Type} 

The preceding rules ensure that the data type of an literal constant 
is completely determined by its form.  Similarly the data type of an 
expression depends on the operands and operators involved.  The 
intrinsic functions are also a special case, since their properties, 
including their data types, are known to the compiler.  All other 
typed objects in a Fortran program are referred to by symbolic 
names.  The rules given here apply to all of these named objects: 
variables, arrays, named constants, statement functions, and 
external functions.  

In many programming languages, especially those in the Algol 
family, the data type of almost every item in the program has to be 
specified explicitly.  Many programmers regard it as a chore to 
have to provide all these type specifications, although their 
presence does make it rather easier for the compiler to detect 
mistakes.   

In Fortran you can specify data types explicitly in a similar way by 
using type statements, but Fortran also makes life easier by having 
certain default types.  The data type of any object which has not 
been declared in a type statement depends on the first letter of its 
name.  The default rules are: 

\begin{center} 
\begin{tabular}{ll} 
\hline 
First letter of the name  & Implicit type \\ 
\hline 
A to H   & REAL \\ 
I to N   & INTEGER \\ 
O to Z   & REAL \\ 
\hline 
\end{tabular} 
\end{center} 

Most programs make extensive use of integer and real objects, so 
these default values reduce the number of type statements that are 
required, provided suitable initial letters are chosen for the 
symbolic names. 

The first-letter rule can also be changed throughout a program unit 
by using an {\tt IMPLICIT} statement, described below. 

\subsubsection*{Type Statements} 

There are six different type statements, one for each data type.  In 
their simplest form they just consist of the appropriate data-type 
keyword followed by a list of symbolic names.  For example: 
\begin{verbatim}
      INTEGER  AGE, GRADE 
      LOGICAL  SUPER  
      REAL  RATE, HOURS, PAY, TAX, INSURE 
\end{verbatim} 

In this example the first four items declared to be real would have 
had that type anyway had the default rules been left to operate.  
Confirmatory type specification does no harm.  

There is no limit to the number of type statements that can be used 
but a name must not have its type specified explicitly more than 
once in a program unit.  Type statements must precede all 
executable statements in the unit; it is good practice, though not 
essential, for them to precede other specification statements 
referring to the same name.  Type statements can be used in a 
subprogram to specify the types of the dummy arguments and, in 
an external function, the type of the function as well.  Type 
statements by themselves have no effect on intrinsic function 
names but it is not a good idea to use them in this way. 

The {\tt CHARACTER} statement is slightly different from the others 
because it also specifies the length of each character item, i.e.\ the 
number of characters it holds.  The length can be given separately 
for each item, thus:\\ 
\verb?      CHARACTER NAME*15, STREET*30, TOWN*20, PCODE*7?\\ 
Alternatively, if several items are to have the same length, a default 
length for the statement can be given at the beginning:\\ 
\verb?      CHARACTER*20 STAR, GALAXY, COMET*4, PLANET ?\\ 
This declares the name {\tt COMET} to have a length of 4 characters, 
whereas {\tt STAR, GALAXY}, and {\tt PLANET} are all 20 characters 
long.  If the length is not specified at all it defaults to one.  The 
length can also be specified by means of a named integer constant 
or an integer constant expression enclosed in parentheses.  For 
example: 
\begin{verbatim}
       PARAMETER (NEXT=15, LAST=80) 
       CHARACTER TEXT*(NEXT+LAST) 
\end{verbatim} 

Note that the length of a character item is fixed at compilation 
time.  The special form:\\ 
\verb?      CHARACTER NAME*(*)?\\ 
is permitted in two cases: for named constants the length of the 
literal constant in the {\tt PARAMETER} statement is used (section 
5.4); for dummy arguments of procedures the length of the 
associated actual argument is used (section 9.5).  Type statements 
can also be used to declare the dimensions of arrays: this is 
described in section 5.6. 

\subsubsection*{{\tt IMPLICIT} Statement} 

The {\tt IMPLICIT} statement can be used to change the first-letter 
default rule throughout a program unit.  For example:\\ 
\verb?      IMPLICIT DOUBLE PRECISION (D,X-Z), INTEGER (N-P)?\\ 
would mean that all names starting with the letters D,X,Y, or Z 
would (unless declared otherwise in type statements) have the type 
double precision.  Similarly the letters I through P, instead of just 
I through N, will imply integer type.  The other letters (A-C,E-H, 
and Q-W) will still imply real type.  

{\tt IMPLICIT} can be used with character type to specify a default 
length as well, for example:\\ 
\verb?      IMPLICIT CHARACTER*100 (C,Z), CHARACTER*4 (S)?\\ 
But this is not usually of much practical value.  As with type 
statements, the default character length is one. 

More than one {\tt IMPLICIT} statement can be used in a program unit 
but the same letter must not have its implied type specified more 
than once.  The usual Fortran implied-type rules apply to all initial 
letters not listed in any {\tt IMPLICIT} statements.  The list of letters 
given after each type must appear in alphabetical order.  
{\tt IMPLICIT} statements normally precede all other specification 
statements in a program.  There is one exception to this: 
{\tt PARAMETER} statements may precede them provided that the 
constants named in them are not affected by the {\tt IMPLICIT} 
statement.  Note that dummy arguments and function names may 
be affected by a subsequent {\tt IMPLICIT} statement.  {\tt IMPLICIT} 
statements have no effect on intrinsic function names. 

\subsubsection*{Guidelines} 

There are two diametrically opposed schools of thought on type 
specification.  The first holds that all names should have their 
types specified explicitly.  This certainly helps programmers to 
avoid mistakes, because they have to think more carefully about 
each item.  It also helps the compiler to diagnose errors more 
easily, especially if the it knows that all names are going to be 
declared in advance.  Some Fortran compilers allow a statement of 
the form ``{\tt IMPLICIT NONE}'' which makes all names typeless by 
default and so requiring every name to be explicitly typed.  Others 
have a compile-time switch with the same effect.  If yours does not 
you may be able to produce a similar effect by using something 
like:\\ 
\verb?      IMPLICIT CHARACTER*1000000 (A-Z)?\\ 
near the beginning of each program unit which is likely to cause an 
error for anything not explicitly typed.  One disadvantage of the 
practice of declaring all names in advance is that the program may 
become so cluttered with specification statements that it may 
obscure its structure and algorithm.  

The alternative way of working is to make maximum use of 
implicit types to reduce the number of statements.  This means, of 
course, that the first letter of each name has to be chosen to suit the 
type, leaving no more than five to be chosen freely: this makes it 
harder than ever to devise meaningful symbolic names.  As a 
result, Fortran programs often include names like {\tt RIMAGE} or 
{\tt ISIZE} or {\tt KOUNT}. Clearly type statements are still needed for 
character type because it is usually necessary to use items of a 
number of different lengths.  

Experience suggests that either system can be satisfactory provided 
it is used consistently.  However the wholesale reassignment of 
initial letters with {\tt IMPLICIT} statements usually increases the 
chance of making a mistake.  {\tt IMPLICIT}, if used at all, should only 
reassign one or two rarely-used letters to the less common data 
types, for example: 
\begin{verbatim}
      IMPLICIT DOUBLE PRECISION (Z), LOGICAL (Q), 
      COMPLEX (X) 
\end{verbatim} 
It is also prudent to use an identical {\tt IMPLICIT} statement in each 
program unit, otherwise type mismatches are more likely to be 
made in procedure calls.  

\subsection{Named Constants} 

The {\tt PARAMETER} statement can be used to give a symbolic name 
to any constant.  This can be useful in several rather different 
circumstances. 

With constants of nature (such as $\pi$) and physical conversion 
factors (like the number of pounds in a kilogram) it can save typing 
effort and reduce the risk of error if the actual number is only 
given once in the program and the name used everywhere else: 
\begin{verbatim}
      REAL PI, TWOPI, HALFPI, RTOD 
      PARAMETER (PI = 3.14159265,  TWOPI = 2.0 * PI) 
      PARAMETER (HALFPI = PI / 2.0,  RTOD = 180.0 / PI) 
\end{verbatim} 
The names {\tt PI,} {\tt TWOPI,} etc. can then be used in place of the
literal 
constants elsewhere in the program unit.  It is much better to use 
named constants than variables in such cases as they are given 
better protection against inadvertent corruption: constants are often 
protected by hardware.  The use of symbolic names rather than 
numbers can also make the program a little more readable: it is 
probably harder to work out the significance of a number like 
1.570796325 than to deduce the meaning of {\tt HALFPI}. 

Another important application of named constants is for items 
which are not permanent constants but parameters of a program, 
i.e.\ items fixed for the present but subject to alteration at some 
later date.  Named constants are often used to specify array 
bounds, character-string lengths, and so on.  For example: 
\begin{verbatim}
      INTEGER MAXR, MAXC, NPTS 
      PARAMETER (MAXR = 100, MAXC = 500, NPTS = MAXR*MAXC) 
      REAL MATRIX(MAXR,MAXC), COLUMN(MAXR), ROW(MAXC) 
\end{verbatim} 
The constants such as {\tt MAXR} and {\tt MAXC} can also be used in the 
executable part of the program, for instance to check that the array 
subscripts are in range:  
\begin{verbatim}
      IF(NCOL .GT. MAXC .OR. NROW .GT. MAXR) THEN 
          STOP 'Matrix is too small'  
      ELSE 
          MATRIX(NROW,NCOL) = ROW(NCOL) 
      END IF 
\end{verbatim} 
If, at some point, the matrix turns out to be too small for your 
needs then you only have to alter this one {\tt PARAMETER} 
statement: everything else will change automatically when the 
program is recompiled. 

The rules for character assignment apply to {\tt PARAMETER} 
statements: see section 7.4.  In addition a special length 
specification of {\tt *(*)} is permitted which means that the length of 
item is set to that of the literal constant.  The type specification 
must precede the {\tt PARAMETER} statement. 
\begin{verbatim}
      CHARACTER*(*) LETTER, DIGIT, ALPNUM 
      PARAMETER (LETTER = 'ABCDEFGHIJKLMNOPQRSTUVWXYZ', 
     $  DIGIT = '0123456789', ALPNUM = LETTER // DIGIT) 
      CHARACTER WARN*(*) 
      PARAMETER (WARN = 'This matrix is nearly singular') 
\end{verbatim} 
The constant ALPNUM will be 36 characters long and contain all 
the alpha-numeric characters (letters and digits). 

Named logical constants also exist, but useful applications for 
them are somewhat harder to find: 
\begin{verbatim}
      PARAMETER (NX = 100, NY = 200, NZ = 300, NTOT = NX*NY*NZ) 
      LOGICAL LARGE 
      PARAMETER (LARGE = (NTOT .GT. 1000000) .OR. (NZ .GT. 1000)) 
\end{verbatim} 

\subsubsection*{{\tt PARAMETER} Statement} 

The general form of the {\tt PARAMETER} statement is:\\ 
\verb+      PARAMETER (+ {\em cname = cexp, cname = cexp,} ... {\tt )}\\ 
where each {\em cname} is a symbolic name which becomes the name of 
a constant, and each {\em cexp} is a constant expression of a suitable 
data type. 

The terms in a constant expression can only be literal constants or 
named constants defined earlier in the same program unit.  
Variables, array elements, and function references are not 
permitted at all.  Otherwise the usual rules for expressions apply: 
parentheses can be used around sub-expressions, and the 
arithmetic types can be intermixed. There is one restriction on 
exponentiation: it can only be used to raise a number to an integer 
power.  The normal rules for assignment statements apply: for 
arithmetic types suitable conversions will be applied if necessary; 
character strings will be truncated or padded to the required length.  
Note that substring references are not permitted in character 
constant expressions. 

{\tt PARAMETER} statements are specification statements and may 
precede or follow type statements.  But any type (or {\tt IMPLICIT}) 
statement which affects the data type or length of a named constant 
must precede it.  Subject to these rules, {\tt PARAMETER} statements 
are permitted to precede {\tt IMPLICIT} statements.  This makes it 
possible for a named constant to set the default length for the 
character type for certain ranges of initial letters.  For example: 
\begin{verbatim}
      PROGRAM CLEVER 
      PARAMETER (LENCD = 40, LENE = 2 * LENCD) 
      IMPLICIT CHARACTER*(LENCD)(C-D), CHARACTER*(LENE)(E) 
      PARAMETER (DEMO = 'This is exactly 40 chars long') 
\end{verbatim} 
Once defined, a named constant can be used in any expression, 
including a dimension-bound expression, or in a {\tt DATA} statement.  
A named constant cannot be used just as part of another constant 
(for example one component of a complex constant) and named 
constants are not permitted at all within format specifications. 

\subsubsection*{Guidelines} 

One of the limitations of Standard Fortran at present is that there 
is no way of allocating memory dynamically.  One of the best ways 
around this is to use named constants to specify array bounds; this 
makes it much easier to alter programs to suit new requirements. 

Names should also be given to all mathematical and physical 
constants that your programs require.  If the same constants are 
needed in several program units then it may be sensible to 
compose a suitable set of {\tt PARAMETER} statements for all of them 
and bring them in where ever necessary using {\tt INCLUDE} 
statements.  

If you define double precision constants in a {\tt PARAMETER} 
statement do not forget that each literal constant value must 
include an exponent using the letter D.  

There are no constant arrays in Fortran: the only way to overcome 
this limitation is to declare an ordinary array in a type statement 
and initialise its elements with a {\tt DATA} statement (described in 
section 11). 

\subsection{Variables} 

A variable is simply a named memory location with a fixed data 
type.  As explained earlier, variables do not have to be declared in 
advance if the data type implied by the first letter of the name is 
appropriate.  Otherwise a type statement is required. 

At the start of execution the value of each variable is undefined 
unless a suitable {\tt DATA} statement appears in the program unit (see 
section 11).  Undefined values must not be used in expressions.  
Local variables in procedures do not necessarily retain their values 
from one invocation of the procedure to another unless a suitable 
{\tt SAVE} statement is provided (section 9.11). 

\subsection{Arrays} 

An array is a group of memory locations given a single name.  The 
elements of the array all have the same data type.   

In mathematics the elements of an array a would be denoted by a1, 
a2, a3, and so on.  In Fortran a particular array element is identified 
by providing a subscript expression in parentheses after the array 
name: A(1), A(2), A(3), etc.  Subscripts must have integer type but 
they may be specified by expressions of arbitrary complexity, 
including function calls. 

An array element can be used in the same way as a variable in 
almost all executable statements.  Array elements are most often 
used within loops: typically an integer loop counter selects each 
element of the array in turn.  
\begin{verbatim}
*Add array OLD to array NEW making array TOTAL 
      PARAMETER (NDATA = 1024) 
      REAL OLD(NDATA), NEW(NDATA), TOTAL(NDATA) 
*...... 
      DO 100, I = 1,NDATA 
          TOTAL(I) = OLD(I) + NEW(I) 
100   CONTINUE 
\end{verbatim} 

\subsubsection*{Declaring Arrays} 

Arrays can have up to seven dimensions; the lower bound of each 
dimension is one unless declared otherwise.  There is no limit on 
the upper bound provided it is not less than the lower bound. 
Arrays which are dummy arguments of a procedure may have their 
dimension bounds specified by integer variables which are 
arguments of the procedure; in all other cases each dimension 
bound must be an integer constant expression.  This fixes the size 
of the array at compile-time. 

Type, {\tt DIMENSION,} and {\tt COMMON} statements may all be used 
to declare arrays, but {\tt COMMON} statements have a specialised use 
(described in section 12).  The {\tt DIMENSION} statement has a 
similar form to a type statement but only declares the bounds of an 
array without determining its data type.  It is usually simpler and 
neater to use a type statement which specifies both at once: \\ 
\verb?      CHARACTER COLUMN(5)*25, TITLE*80 ?\\ 
Note that when declaring character arrays the string length follows 
the list of array bounds.  The character array COLUMN has 5 
elements each of which is 25 characters long; TITLE is, of course, 
just a variable 80 characters long.  Although a default string length 
can be set for an entire type statement, it is not possible to set a 
default array size in a similar way. 

It is generally good practice to use named constants to specify 
array bounds as this facilitates later modifications: 
\begin{verbatim}
      PARAMETER (MAXIM = 15) 
      INTEGER POINTS(MAXIM) 
      COMPLEX  SERIES(2**MAXIM) 
\end{verbatim} 
These arrays all have a lower bound of one.  A different lower 
bound can be specified for any dimension as shown below.  The 
lower and upper bounds are separated by a colon: 
\begin{verbatim}
      REAL TAX(1985:1990), PAY(12,1985:1990) 
      LOGICAL TRIPLE(-1:1, -1:1, -1:1, -1:1) 
\end{verbatim} 
TAX has 6 elements from TAX(1985) to TAX(1990).\\ 
PAY has 72 elements from PAY(1,1985) to PAY(12,1990).\\ 
TRIPLE has 81 elements from BIN(-1,-1.-1.-1) to BIN(1,1,1,1).

Although Fortran itself sets no limits to the sizes of arrays that can 
be defined, the finite capacity of the hardware is likely to do so.  
In virtual memory operating systems it is possible to use arrays 
larger than physical memory: those parts of the array not in active 
use are held on backing store such as a disc file. 

\subsubsection*{Using Arrays} 

An array element reference must always use the same number of 
subscripts as the number of dimensions declared for the array.  
Each subscript can be an integer expression of any complexity, but 
there are restrictions on functions with side effects (see section 
9.3).  

An array element reference is only valid if all of the subscript 
expressions are defined and if each one is in the range declared for 
it.  An array element can only be used in an expression if a value 
for it has been defined.  A {\tt DATA} statement (section 12) can be 
used to define an initial value for an entire array or any set of 
elements.  

An array can be used without subscripts:  
\begin{itemize} 
\item  in a specification statement such as a type, 
{\tt DIMENSION,} or {\tt SAVE} statement; 
\item  in a function reference or {\tt CALL} statement: this transfers 
the whole of the array to the associated dummy 
argument (which must have a compatible array 
declaration); 
\item  in the data transfer list of a {\tt READ} or {\tt WRITE} statement: 
this causes the whole array to be input or output.  This 
is not permitted for an assumed size dummy argument 
array. 
\item  as a unit identifier in a {\tt READ} or {\tt WRITE} statement: a 
character array is then an internal file with one record 
per element. 
\item  as a format identifier in a {\tt READ} or {\tt WRITE} statement: 
the format specification is contained in the character 
array with its elements taken in sequence. 
\end{itemize} 

 
\subsubsection*{Storage Sequence} 

Arrays are always stored in a contiguous set of memory locations.  
In the case of multi-dimensional arrays, the order of the elements 
is that the first subscript varies most rapidly, then the second 
subscript, and so on.  For example in the following 2-dimensional 
array, X(2,3) (which for simplicity I have made one of only six elements):  

\tthdump{
\begin{displaymath}
           X(2,3) = \left[ \begin{array}{ccc}
                    x_{1,1} & x_{1,2} & x_{1,3} \\
                    x_{2,1} & x_{2,2} & x_{2,3}
                    \end{array}\right]
\end{displaymath}
}

The elements are stored in the following sequence: \\ 
\verb?    X(1,1), X(2,1), X(1,2), X(2,2), X(1,3), X(2,3) ?\\ 
i.e.\ the sequence moves down each column first, then across to the 
next row.  This column order is different from that used in some 
other programming languages. 

The storage order may be important if you use large multi-dimensional
arrays and wish to carry out some operation on all the elements of the
array.  It is then likely to be faster to access the array in storage
order, i.e.\ by columns rather than rows.  This means arranging loop
indices with the last subscript indexed by the outer loop, and so on
inwards.  For example:
\begin{verbatim}
      DOUBLE PRECISION ARRAY(100,100), SUM 
      SUM = 0.0D0 
      DO 250,L = 1,100 
          DO 150,K = 1,100 
               SUM = SUM + ARRAY(K,L) 
150       CONTINUE 
250   CONTINUE 
\end{verbatim} 
With the loops arranged this way around the memory locations are 
accessed in consecutive order, which minimises the processor 
overhead in subscript calculations.  

\section{Arithmetic} 

Fortran has good facilities for processing numbers.  Arithmetic 
expressions and assignment statements can include integer, real, 
double precision, or complex items.  Data type conversions are 
provided automatically when necessary; type conversions can also 
be performed explicitly using intrinsic functions.  Other intrinsic 
functions are available for trigonometry, logarithms, and other 
useful operations. 

For example, the well-known cosine formula for the third side of 
a triangle, given the other two sides and the angle between them is: 

\tthdump{\[     \sqrt{{b^{2} + c^{2} - 2\cdot b\cdot c\cdot \cos(A)}} \] }
%%tth: \[ sqrt( b^{2} + c^{2} - 2.b.c.cos(A)) \] 

Translated into a Fortran expression it looks like this:\\ 
\verb?      SQRT(B**2 + C**2 - 2.0 * B * C * COS(ANGLEA)) ?\\ 
which makes use of the intrinsic functions {\tt SQRT} and {\tt COS}.  
Although {\tt SQRT(X)} produces the same result as {\tt X**0.5}, the 
square-root function is simpler, faster, and probably more accurate 
than raising to the power of one half, which would actually be 
carried out using both the {\tt EXP} and {\tt LOG} functions.  

Assignment statements evaluate an expression and assign its value 
to a variable (or array element).  Unlike almost all other Fortran 
statements, they do not start with a keyword.  For example: 
\begin{verbatim}
      A = SQRT(B**2 + C**2 - 2.0 * B * C * COS(ANGLEA)) 
      TOTAL(N/2+1) = 0.0 
      FLUX = FLUX + 1.0 
\end{verbatim} 

\subsection{Arithmetic Expressions} 

An expression in its simplest form is just a single operand, such as 
a constant or variable.  More complicated expressions combine 
various operands with operators, which specify the computations 
to be performed.  For example:\\ 
\verb?      RATE * HOURS + BONUS?\\ 
The rules of Fortran have been designed to resemble those of 
mathematics as far as possible, especially in determining the order 
in which the expression is evaluated.  In this example the 
multiplication would always be carried out before the addition, not 
because if comes first, but because it has a higher precedence.  
When in doubt, or to over-ride the precedence rules, parentheses 
can be used:\\ 
\verb?      (ROOM + DINNER) * 1.15?

Sub-expressions enclosed in parentheses are always evaluated first; 
they can be nested to any reasonable depth.  If in doubt, there is no 
harm in adding parentheses to determine the order of evaluation or 
to make a complicated expression easier to understand.   

Arithmetic expressions can contain any of the five arithmetical 
operators \verb? +  -  *  /  ** ?. 
The double asterisk represents exponentiation, i.e.\ raising a 
number to a power.  Thus the mathematical expression: 
\[ (1 + RATE/100)^{years} \] 
could be represented in Fortran as:\\ 
\verb?      (1.0 + RATE/100.0)**YEARS ?\\ 
(note the explicit decimal points in the constants to make them real 
values).  

Arithmetic expressions can involve operands of different data 
types: the data type of the result is determined by some simple 
rules explained below. 

\subsubsection*{General Rules} 

Arithmetic expressions can contain arithmetic operands, arithmetic 
operators, and parentheses.  There must always be at least one 
operand.  The operands can belong to any of the four arithmetic 
data types (integer, real, double precision, or complex); the result 
also has an arithmetic data type.  Operands can be any of the 
following: 
\begin{itemize} 
\item   unsigned literal constants 
\item   named constants 
\item   variables 
\item   array elements 
\item   function references 
\item   complete expressions enclosed in parentheses. 
\end{itemize} 
The rules for forming more complicated arithmetic expressions are 
as follows.  An arithmetic expression can have any of the following 
forms: 
\begin{center} 
\begin{tabular}{l} 
      {\em operand} \\ 
      {\tt +}{\em operand} \\ 
      {\tt -}{\em operand} \\ 
      {\em arithmetic-expression} {\em ~~arith-op} {\em ~~operand} \\ 
\end{tabular} 
\end{center} 
where the arith-op can be any of these operators: 
\begin{center} 
\begin{tabular}{cl} 
            \verb?+?      & {\em addition} \\ 
            \verb?-?      & {\em subtraction} \\ 
            \verb?*?      & {\em multiplication} \\ 
            \verb?/?      & {\em division} \\ 
            \verb?**?     & {\em exponentiation}\\
\end{tabular} 
\end{center} 

The effect of these rules is that an expression consists of a string 
of operands separated by operators and, optionally, a plus or minus 
at the start.  A leading plus sign has no effect; a leading minus sign 
negates the value of the expression. 

All literal arithmetical constants used in expressions must be 
unsigned: this is to prevent the use of two consecutive operators 
which is confusing and possibly ambiguous:\\ 
\verb?      4 / -3.0**-1             ? (illegal).\\ 
The way around this is to use parentheses, for example:\\ 
\verb?      4 / (-3.0)**(-1)? \\ 
which makes the order of evaluation explicit. 
\\ 
The order of evaluation of an expression is: 
\begin{enumerate} 
\item sub-expressions in parentheses 
\item function references 
\item exponentiation, i.e.\ raising to a power 
\item multiplication and division 
\item addition, subtraction, or negation. 
\end{enumerate} 

 
Within each of these groups evaluation proceeds from left to right, 
except that exponentiations are evaluated from right to left.  Thus: 
{\tt A / B / C} is equivalent to {\tt (A / B) / C} 
whereas  {\tt X ** Y ** Z} is equivalent to  {\tt X ** (Y ** Z)}. 

An expression does not have to be evaluated fully if its value can 
be determined otherwise: for example the result of:\\ 
\verb?      X * FUNC(G)?\\ 
can be determined without calling the function FUNC if X happens 
to be zero.  This will not cause problems if you only use functions 
that have no side effects. 

\subsubsection*{Data Type Conversions} 

If an operator has two operands of the same data type then the 
result has the same type.  If the operands have different data types 
then an implicit type conversion is applied to one of them to bring 
it to the type of the other.  These conversions always go in the 
direction which minimises loss of information: \\
\centerline{
{\em integer } converts to {\em real }
converts to {\em complex} {\em or} {\em double precision}} 

Since there is no way of converting a complex number to double 
precision type, or vice-versa, without losing significant 
information, both these conversions are prohibited: an operator 
cannot have one complex operand and one of double precision 
type.  All other combinations are permitted.  These implicit type 
conversions have the same result as if the appropriate intrinsic 
function (REAL, DBLE, or CMPLX) had been used.  These are 
described in detail below.  Note that the data type of any operation 
just depends on the two operands involved; the rest of the 
expression has no influence on it whatever. 

Exponentiation is an exception to the type conversion rule: when 
the exponent is an integer it does not have to be converted to the 
type of the other operand and the result is evaluated as if by 
repeated multiplication.  But if the exponent has any other data 
type the calculation is performed by implicit use of the LOG and 
EXP functions, thus: 
\begin{verbatim}
     2.0**3   ===>  2.0 * 2.0 * 2.0      ===> 8.0

     2.0**3.0 ===>  EXP(3.0 * LOG(2.0))  ===> + 8.0 
\end{verbatim}
The first result will, of course, be computed more rapidly and 
accurately than the second.  If the exponent has a negative value 
the result is simply the reciprocal of the corresponding positive 
power, thus: 
\begin{verbatim}
     2.0**(-3)  ===>  1.0/2.0**3  ===>  1.0/8.0  ===>  +0.125
\end{verbatim}

Note that conversion from real to double precision cannot produce 
any information not present originally.  Thus with a real variable 
R and a double precision variable D: 
\begin{verbatim}
       R = 1.0 / 3.0 
       D = R 
\end{verbatim} 
D may end up with a value such as 0.3333333432674408... which 
is no closer to the value of one third than R was originally.   

\subsubsection*{Integer Division} 

Integer division always produces a result which is another integer 
value: any fractional part is truncated, i.e.\ rounded towards zero.  
This makes it especially important to provide a decimal point at the 
end of a real constant even if the fractional part is zero.  For 
example: 
\begin{verbatim}
      8 / 3  ===>  2

     -8 / 3  ===>  -2

     2**(-3) ===>  1/(2**3)  ===>  1/8  ===>  0
\end{verbatim}

The combination of the two preceding rules may have unexpected 
effects, for example:
\begin{verbatim}
      (-2)**3  ===> (-2) * (-2) * (-2)  ===>  -8
\end{verbatim}
whereas (-2)**3.0 is an invalid expression as the computer would 
try to evaluate the logarithm of -2.0, which does not exist.  
Similarly, the expression: 
\begin{verbatim}
     3 / 4 * 5.0  ===>  REAL(3/4) * 5.0  ===> 0.0 
\end{verbatim}
whereas 
\begin{verbatim}
     5.0 * 3 / 4  ===> 15.0 / REAL(4)  ===>  3.75
\end{verbatim}
 
\subsubsection*{Restrictions} 

Certain arithmetical operations are prohibited because their results 
are not mathematically defined.  For example dividing by zero, 
raising a negative value to a real power, and raising zero to a 
negative power.  The Fortran Standard does not specify exactly 
what is to happen if one of these errors occurs: most systems issue 
an error message and abort the program. 

Errors can also occur because numbers are stored on a computer 
with finite range and precision.  The results of adding or 
multiplying two very large numbers may be outside the number 
range: this is called overflow.  A similar effect on very large 
negative integers is called underflow.  Most systems will issue a 
warning message for overflow or underflow, and may abort the 
program, but some processors cannot detect errors of this sort 
involving integer arithmetic. 

Every operand (variable, array element, or function reference) used 
in an expression must have a defined value at the time the 
expression is evaluated.  Note that variables and arrays are initially 
undefined unless a suitable {\tt DATA} statement is used.   
Expressions must not include references to any external functions 
with side effects on other operands of the expression: see section 
9.3 for more details. 

\subsubsection*{Arithmetic Constant Expressions} 

Arithmetic constant expressions can be used in {\tt PARAMETER} 
statements and to specify implied-DO parameters in {\tt DATA} 
statements.  All the operands in a constant expression must be 
literal constants or previously defined named constants. Variables, 
array elements, and function references are all prohibited.  
Exponentiation is only allowed if the number is raised to an integer 
power.  

The same rules apply to integer constant expressions but in 
addition the operands must all be integer constants: such 
expressions can be used to specify array bounds in type, 
{\tt COMMON,} and {\tt DIMENSION} statements, and to specify string 
lengths in {\tt CHARACTER} statements.   

\subsubsection*{Bit-wise Logical Operations on Integers} 

When Fortran programs communicate directly with digital 
hardware it may be necessary to carry out bit-wise logical 
operations on bit-patterns.  Standard Fortran does not provide any 
direct way of doing this, since logical variables essentially only 
store one bit of information and integer variables can only be used 
for arithmetic.  Many systems provide, as an extension, intrinsic 
functions to perform bit-wise operations on integers.  The function 
names vary: typically they are {\tt IAND,} {\tt IOR,} {\tt ISHIFT}.  A few 
systems provide allow the normal logical operators such as {\tt .AND.} 
and {\tt .OR.} to be used with integer arguments: this is a much more 
radical extension and much less satisfactory, not only because it 
reduces portability, but also reduces the ability of the compiler to 
detect errors in normal arithmetic expressions.  

Many systems also provide format descriptors to transfer integers 
using octal and hexadecimal number bases: these are also non-standard.  

\subsubsection*{Guidelines} 

Expressions with mixed data types should be examined carefully 
to ensure that the type-conversion rules have the desired effect.  It 
does no harm to use the type conversion functions explicitly and 
it may make the working clearer.   

Particular care is needed with the data types of literal constants.  
It is bad practice to use an integer constant where you really need 
a real constant.  Although this will work in most expressions it is 
a serious mistake to use the wrong form of constant in the 
argument list of a procedure.  

Long and complicated expressions which spread over several lines 
can be rather trying to read and offer more scope for programming 
errors.  Sometimes it is better to split the computation into several 
shorter equations at the expense of one or two temporary variables.  

 
It is often tempting to try to write programs that are as efficient as 
possible.  With modern compilers there is little point in trying to 
rearrange expressions to optimise speed.  One of the few 
exceptions is that if an intrinsic function is provided it is always 
best to use it; thus {\tt SQRT(X)} is likely to be faster and more 
accurate than {\tt X**0.5}. 

You may find that your system actually sets the whole of memory 
to zero initially, except for items defined with {\tt DATA} statements, 
but it is very bad programming practice to rely on this. 

\subsection{Arithmetic Intrinsic Functions} 

Intrinsic functions are supplied automatically by the system and 
can be used in expressions in any program unit.  A description of 
their special properties appears in section 9.1. 

Many of the arithmetic intrinsic functions have generic names: that 
is they can be used with several different types of arguments.  The 
SQRT function, for example, can be used with a real, double 
precision, or complex argument.  The Fortran system automatically 
selects the correct specific function for the job: SQRT, DSQRT, or 
CSQRT.  These specific names can be ignored in almost all 
circumstances, and are listed only in the appendix. In most cases 
the data type of the function is the same as that of its argument but 
there are a few obvious exceptions such as the type conversion 
functions.   

In the descriptions below, the number and data type of the 
arguments of each intrinsic function are indicated by a letter:  
I = integer, R = real, D = double precision, X = complex.   

An asterisk on the left indicates that the result has the same data 
type as the arguments.  Note that if multiple arguments are 
permitted they must all have the same data type.  Thus 
I = NINT(RD) indicates that the NINT function can take a single 
real or double precision argument but its result is always integer, 
whereas * = ANINT(RD) indicates that the result has the same type 
(real or double precision) as the argument.   

\subsubsection*{Trignometric Functions} 

The functions in this group can all be used on real or double 
precision arguments, and SIN and COS can also be used on 
complex numbers.  In every case the result has the same data type 
as the argument. 

\begin{tabular}{p{1.5in}p{4.2in}} 
{\tt * = SIN(RDX)} &   sine of the angle in radians. \\ 
{\tt * = COS(RDX)   } &   cosine of the angle in radians.  \\ 
{\tt * = TAN(RD)     } &   tangent of the angle in radians.  \\ 
{\tt * = ASIN(RD)    } &   arc-sine; the result is in the range $- \pi
/2$ to 
                     $ +\pi /2$. \\ 
{\tt * = ACOS(RD)    } &   arc-cosine; the result is in the range 0 to 
                    $+ \pi$ .  \\ 
{\tt * = ATAN(RD)}     &   arc-tangent; the result is in the range $- \pi
/2$ to 
                     $ +\pi /2$. \\ 
{\tt * = ATAN2(RD,RD)}&   arc-tangent of arg1/arg2; the result is in the 
                    range $- \pi$ to $+ \pi$.  Both arguments must  
                    not be zero.  \\ 
{\tt * = SINH(RD)}      &  hyperbolic sine. \\ 
{\tt * = COSH(RD)} &  hyperbolic cosine.  \\ 
{\tt * = TANH(RD)}      &  hyperbolic tangent. \\ 
\end{tabular} 

Note that the arguments of SIN, COS, and TAN must be angles 
measured in radians (not degrees).  They can be used on angles of 
any size, positive or negative, but if the magnitude is very large the 
accuracy of the result will be reduced.  Similarly all the inverse 
trigonometric functions deliver a result in radians; the argument of 
ASIN and ACOS must be in the range -1 to +1.  
      The ATAN2 function can be useful in resolving a result into 
the correct quadrant of the circle, thus:\\ 
\verb?      ATAN(0.5)        = 0.4636476?\\ 
\verb?      ATAN2(2.0,4.0)   = 0.4636476?\\ 
\verb?      ATAN2(-2.0,-4.0) = -2.677945? ( = 0.4636476 - $\pi$).

\subsubsection*{Other Transcendental Functions} 

\begin{tabular}{p{1.5in}p{4.3in}} 

{\tt * = SQRT(RDX)}    &  square root.  \\ 
{\tt * = LOG(RDX)}     &  natural logarithm, i.e.\ log to base e 
(where e = 2.718281828...).  \\ 
{\tt * = EXP(RDX)}     &  returns the exponential, i.e.\ e to the 
                    power of the argument.  This is the 
                    inverse of the natural logarithm.    \\ 
{\tt * = LOG10(RD)}     &  logarithm to base 10.  \\ 
\end{tabular} 

Note that LOG10, which may be useful to compute decibel ratios 
etc., is the only one of this group which cannot be used on a 
complex argument.  

\subsubsection*{Type Conversion Functions} 

These functions can be used to convert from any of the four 
arithmetic data types to any of the others.  They are used 
automatically whenever mixed data types are encountered in 
arithmetic expressions and assignments.   

\begin{tabular}{ll} 
{\tt I  = INT(IRDX)}        & converts to integer by truncation. \\ 
{\tt R  = REAL(IRDX)}       & converts to real. \\ 
{\tt D  = DBLE(IRDX)}       & converts to double precision. \\ 
{\tt X = CMPLX(IRDX)}      & converts to complex. \\ 
{\tt X = CMPLX(IRD,IRD)}    & converts to complex. \\ 
\end{tabular} 

The integer conversion of INT rounds towards zero; if you need to 
round to the nearest integer use the NINT function (described 
below).  The CMPLX function produces a value with a zero 
imaginary component unless it is used with two arguments (or one 
which is already complex).  It is important to realise that many 
conversions lose information: in particular a double precision 
value is likely to lose significant digits if converted to any other 
data type. 

\subsubsection*{Minimum and Maximum} 

The MIN and MAX functions are unique in being able to take any 
number of arguments from two upwards; the result has the same 
data type as the arguments.  

\begin{tabular}{p{1.9in}p{3.8in}} 
{\tt * = MIN(IRD,IRD,...)}     & returns the smallest of its arguments.\\ 
{\tt * = MAX(IRD,IRD,...)}     & returns the largest of its arguments. \\ 
\end{tabular} 

These two functions can, of course, be combined to limit a value 
to a certain range.  For example, to limit a value TEMPER to the 
range 32 to 212 you can use an expression such as:\\ 
\verb?      MAX(32.0, MIN(TEMPER, 212.0))?\\ 
Note that the minimum of the range is an argument of the MAX 
function and vice-versa. 

To find the largest (or smallest) element of a large array it is 
necessary use a loop.  
\begin{verbatim}
*Find largest value in array T of N elements:  
      TOP = T(1) 
      DO 25,I = 2,N 
         TOP = MAX(T(I), TOP) 
25    CONTINUE 
*TOP now contains the largest element of T. 
\end{verbatim} 

\subsubsection*{Other Functions} 

\begin{tabular}{p{1.9in}p{3.8in}} 
{\tt *  = AINT(RD)}       & Truncates the fractional part (i.e.\ as INT) 
                    but preserves the data type.  \\ 
{\tt *  = ANINT(RD)}      & Rounds to the nearest whole number.  \\ 
{\tt I  = NINT(RD)}       & Converts to integer by rounding to the nearest 
                    whole number.  \\ 
{\tt *  = ABS(IRD)}       & Returns the absolute value of a number 
                    (i.e.\ it changes the sign if negative).  \\ 
{\tt R  = ABS(X)}        & Computes the modulus of a complex number 
                    (i.e.\ the square-root of the sum of the squares 
                    of the two components).  \\ 
{\tt *  = MOD(IRD,IRD)}   & returns A1 modulo A2, i.e.\ the remainder 
                    after dividing A1 by A2.  \\ 
{\tt *  = SIGN(IRD,IRD)}  & performs sign transfer: if A2 is negative 
                    the result is -A1, if A2 is zero or positive 
                    the result is A1.  \\ 
{\tt *  = DIM(IRD,IRD)}   & returns the positive difference of A1 and 
                    A2, i.e.\ if A1 {\tt >} A2 it returns (A1-A2), 
                    otherwise zero.  \\ 
{\tt D  = DPROD(R,R)}     & Computes the double precision product of two 
                    real values.   \\ 
{\tt R  = AIMAG(X)}      & Extracts the imaginary component of a 
                    complex number.  Note that the real 
                    component can be obtained by using the 
                    REAL function.  \\ 
{\tt X = CONJG(X)}      & Computes the complex conjugate of a 
                    complex number.  \\ 
\end{tabular} 

The NINT and ANINT functions round upwards if the fractional 
part of the argument is 0.5 or more, whereas INT and AINT always 
round towards zero.  Thus:\\
\verb?      INT(+3.5) =  3       NINT(+3.5) =  4 ?\\
\verb?      INT(-3.5) = -3       NINT(-3.5) = -4 ?\\
The fractional part of a floating point number, X, can easily be 
found either by:\\
\verb?      X - AINT(X)?\\ 
or\\ 
\verb?      MOD(X, 1.0)?\\
In either case, if X is negative the result will also be negative.  The 
ABS function can always be used to alter the sign if required.  

The MOD function has other uses.  For example it can find the day 
of the week from an absolute day count such as Modified Julian 
Date (MJD):\\
\verb?      MOD(MJD,7)?\\
has a value between 0 and 6 for days from Wednesday to Tuesday.  
Similarly if you use the ATAN2 function but want the result to lie 
in the range 0 to 2*pi (rather than -pi to +pi) then, assuming the 
value of TWOPI is suitably defined, the required expression is:\\
\verb?      MOD(ATAN2(X,Y) + TWOPI, TWOPI)?

\subsection{Arithmetic Assignment Statements} 

An arithmetic assignment statement has the form:\\ 
\verb?      ? {\em arithmetic-var} {\tt =} {\em arithmetic-expression}\\ 
where {\em arithmetic-var} can be an arithmetic variable or array 
element.  For example, the following assignment statement is valid 
provided that N, K, and ANGLE are all defined values: \\ 
\verb?      IMAGE(N/2+1,3*K-1) = SIN(ANGLE)**2 + 1.0?\\ 
If the object on the left has a different data type from that of the 
expression on the right then a data type conversion is applied 
automatically.  The type conversion function (INT, REAL, DBLE, 
or CMPLX) is selected to match the object on the left.  Note that 
many type conversions lose information.  If the object on the left 
is an array element, its subscripts can be arbitrary integer 
expressions, but all the operands in these expressions must be 
defined before the statement is executed and each must be in the 
range declared for the corresponding subscript of the array. 

Remember with an integer item on the left and an expression of 
one of the floating-point types, the INT function is invoked: if the 
NINT function is really needed then it must be used explicitly to 
convert the value of the expression.  

\section{Character Handling and Logic} 

This section describes the facilities for handling non-numerical 
data in Fortran.  Character data are actually present in almost all 
programs, if only in the form of file names and error messages, but 
the facilities for character manipulation are now quite powerful.  
The logical data type is even more indispensible since a logical 
expression is used in every {\tt IF} statement. 

\subsection{Character Facilities} 

The character data type differs from all the others in one important 
respect: every character item has a fixed length.  This specifies the 
number of characters it holds. 

The length of a literal character constant is just the number of 
characters between the enclosing apostrophes (except that two 
consecutive apostrophe within the string count as one).  Thus:\\ 
\verb?      'it''s' ?\\ 
is a character constant of length four.  Because the length of every 
character variable, array, and function has to be specified in 
advance it is nearly always necessary to use {\tt CHARACTER}  
statements to declare them, for example: \\ 
\verb?      CHARACTER NAME*20, ADDRSS(3)*40,  ZIP*7?\\ 
The same applies to named character constants but for these a 
special notation sets the length to that of the attached constant, 
which saves the trouble of counting characters: 
\begin{verbatim}
      CHARACTER TITLE*(*) 
      PARAMETER (TITLE = 'Latest mailing list') 
\end{verbatim} 

The fixed length of character objects makes it easy to output data 
in a fixed format as when printing a table with neatly aligned 
columns, but sometimes it would be more convenient to have a 
variable length string type as some other languages do.  The rules 
for character assignment go some way towards this: if an 
expression is too short then blanks are appended to it; if it is too 
long then characters are removed from the right-hand end.  For 
many purposes, therefore, it is only necessary to ensure that 
character variables are at least as long as the longest string you 
need to store in them.  

When transferring character information to procedures the length 
of the dummy argument can be set automatically to that of the 
corresponding actual argument.  With this passed length notation 
it is easy to write general-purpose character handling procedures.  
This is described further in section 9.5. 

The most common operations carried out on character strings are 
splitting them up and joining them together.  Any section of a 
character variable or array element can be extracted by using the 
substring notation.  Strings (and substrings) can be joined end to 
end by using the concatenation operator in a character expression.  
These are described in the next two sections. 

Another fairly common requirement is to search for a particular 
sequence of characters within a longer string: this can be done with 
the intrinsic function {\tt INDEX}. 

Other intrinsic functions {\tt ICHAR} and {\tt CHAR} are provided to 
convert a single character to an integer or vice-versa according to its 
position within the native character set.  More complicated conversions 
from a numerical data type to character form and vice-versa are best 
carried out using the internal file {\tt READ} and {\tt WRITE} statements 
which allow the power of the format specification to be applied to the task. 
This mechanism is described in section 10.3. 

Character strings can be compared to each other using relational 
operators or intrinsic functions.  The latter use the ASCII collating 
sequence irrespective of the native character code.  Further details 
are given in section 7.6. 

\subsection{Character Substrings} 

The substring notation can be used to select any contiguous section 
of any character variable or array element.  The characters in any 
string are numbered starting from one on the left: the lower bound 
cannot be altered as it can in arrays.  A substring is selected simply 
by giving the first and last character positions of the extract.  For 
example, with: 
\begin{verbatim}
      CHARACTER METAL*10 
      METAL = 'CADMIUM' 
\end{verbatim} 
then {\tt METAL(1:3)} has the value {\tt 'CAD'} while {\tt METAL(8:8)} 
has the value {\em blank} because the value is padded out with blanks to 
its declared length. 

Substrings must be at least one character long.  They can be used 
in general in the same ways as character variables.  Continuing 
with the last example, the assignment statement: \\ 
\verb?      METAL(3:4) = 'ES'?\\ 
will change the value of METAL to {\tt 'CAESIUM   '} (with three 
blanks at the end, since the total length stays at 10). 

\subsubsection*{Substring Rules} 

The parentheses denoting a substring must contain a colon: there 
may be an integer expression on either side of the colon.  The first 
expression denotes the initial character position, the second one 
the last character position.  Both values must be within the range 
1 to LEN, where LEN is the length of the parent string, and the 
length of the resulting substring must not be less than one.  

Although the colon must always be present, the two integer 
expressions are optional.  The default value for the first one is one, 
the default for the second is the position of the last character of the 
parent string.  Thus, staying with the last example: 
{\tt METAL(:2)} has the value {\tt 'CA'}  
while {\tt METAL(7:)} has the value  {\tt 'M' } with three blanks. 

With array elements the substring expression follows the sub-script 
expression, for example: 
\begin{verbatim}
      CHARACTER PLAY(30)*80 
      PLAY(10) = 'AS YOU LIKE IT' 
\end{verbatim} 
Then the substring {\tt PLAY(10)(4:11)} has the value {\tt 'YOU LIKE'}. 
Substrings can be used in expressions anywhere except in the 
definition of a statement function; they can also be used on the 
left-hand side of an assignment statement, and can also be 
defined by input/output statements.  

\subsection{Character Expressions} 

The character operator {\tt //} is used to concatenate, or join, two 
character strings.  It is, in fact, the only character operator that 
Fortran provides.  Thus: \\
\centerline{{\tt 'CUP' // 'BOARD'}  becomes {\tt 'CUPBOARD'}} 
The length of the result is just the sum of the lengths of the 
operands.  Parentheses may be used in character expressions but 
make no difference to the result.  Note that any embedded or 
trailing blanks (spaces) will be reproduced exactly in the resulting 
string. 

The general form of a character-expression is thus:\\ 
\verb+      + {\em character-operand}\\ 
or\verb+    + {\em character-expression} {\tt //} {\em
character-operand}\\ 
where {\em character-operand} can be any of the following: 
\begin{itemize} 
\item character constant (literal or named), 
\item character variable,  
\item character array element,  
\item character substring, 
\item character function reference. 
\end{itemize} 

There is one special restriction on character concatenation in 
procedures: a passed-length dummy argument can only be an 
operand of the concatenation operator in an assignment statement.  
This seemingly arbitrary rule allows the compiler to determine how 
much work-space is required.  

\subsection{Character Assignment Statements} 

The character assignment statement has the general form:\\ 
\verb+      + {\em char-var} {\tt =} {\em character-expression} \\ 
where {\em char-var} can be a character variable, array element, or 
substring. 

There is one important restriction on character assignment 
statements: none of the characters being referenced in the 
expression on the right may be defined in char-var on the left, that 
is to say there can be no overlap.  Thus the assignment statement:\\ 
\verb?      STRING(1:N) = STRING(10:)?\\ 
is valid only as long as N is no higher than 9.  It is, of course, easy 
to get around this restriction by using a temporary character 
variable with a suitable length. 

Note when a value is assigned to a substring (as in the last 
example) the other characters in the parent string are not affected 
at all.  If the string was previously undefined then the other 
character positions will still be undefined; otherwise they will 
retain their previous contents. 

The expression and the character object to which its value is 
assigned may have different lengths: if the expression is longer 
then the excess characters on the right are lost; if it is shorter then 
blanks are appended.  Care is needed to declare adequate lengths 
or else the results can be unexpected: 
\begin{verbatim}
      CHARACTER AUTHOR*30, SHORT*5, EXPAND*10 
      AUTHOR = 'SHAKESPEARE, WILLIAM' 
      SHORT = AUTHOR 
      EXPAND = SHORT 
\end{verbatim} 
The resulting value of {\tt EXPAND} will be \verb?'SHAKE     '? where the 
last five characters are blanks. 

\subsection{Character Intrinsic Functions} 

The four main character intrinsic functions are described in this 
section.  There are another four functions provided to compare 
character strings with each other using the ASCII collating 
sequence: these are described in section 7.6.  

\subsubsection*{{\tt CHAR} and {\tt ICHAR}} 

These two functions perform integer to character conversion and 
vice-versa using the internal code of the machine.  Although most 
computers now use the ASCII character code, it is by no means 
universal, so these functions can only be used in a very limited way 
in portable software. 

{\tt CHAR(I)} returns the character at position {\tt I} in the code 
table.  For example, on a machine using ASCII code, {\tt CHAR(74)} = {\tt 
'J'}, since ``{\tt J}'' is the character number 74 in the ASCII code table. 

{\tt ICHAR(STRING)} returns the integer position in the code table of 
the first character of the argument {\tt STRING}.  For example, on a 
machine using ASCII code,\\
\centerline{{\tt ICHAR('JOHN')} returns  74} 
\centerline{{\tt ICHAR('john')} returns 106} 

\subsubsection*{{\tt INDEX}} 

{\tt INDEX} is a search function; it takes two character arguments and 
returns an integer result.  {\tt INDEX(S1, S2)} searches for the 
character-string {\tt S2} in another string {\tt S1}, which is usually 
longer.  If {\tt S2} is present in {\tt S1} the function returns the 
character position at which it starts.  If there is no match (or 
{\tt S1} is shorter than {\tt S2}) then it returns the value zero.  For 
example: 
\begin{verbatim}
      CHARACTER*20 SPELL 
      SPELL  = 'ABRACADABRA' 
      K      = INDEX(SPELL, 'RA') 
\end{verbatim} 
Here {\tt K will} be set to 3 because this is the position of the first 
occurrence of the string {\tt 'RA'}. To find the second occurrence it is 
necessary to restart the search at the next character in the main 
string, for example:\\ 
\verb?      L = INDEX(SPELL(K+1:), 'RA')?\\ 
This will return the value 7 because the first occurrence of {\tt 'RA'} 
in the substring {\tt 'ACADABRA'} is at position 7.  To find its position 
in the parent string the offset, {\tt K}, must be added, making 10. 

The {\tt INDEX} function is often useful when manipulating character 
information.  Suppose, for example, we have a string NAME 
containing the a person's surname and initials, e.g.\\ 
\verb?      Mozart,W.A ?\\ 
The name can be reformatted to put the initials before the surname 
and omit the comma like this: 
\begin{verbatim}
      CHARACTER NAME*25, PERSON*25 
*... 
      KCOMMA = INDEX(NAME, ',') 
      KSPACE = INDEX(NAME, ' ') 
      PERSON = NAME(KCOMMA+1:KSPACE-1) // NAME(1:KCOMMA-1) 
\end{verbatim} 
Then PERSON will contain the string {\tt 'W.A.Mozart'} (with blanks 
appended to the length of 25).  Note that a separate variable, 
{\tt PERSON,} was necessary because of the rule about overlapping 
strings in assignments. 

\subsubsection*{{\tt LEN}} 

The {\tt LEN} function takes a character argument and returns its length 
as an integer.  The argument may be a local character variable or 
array element but this will just return a constant.  {\tt LEN} is more 
useful in procedures where character dummy arguments (and 
character function names) may have their length passed over from 
the calling unit, so that the length may be different on each 
procedure call.  The length returned by {\tt LEN} is that declared for the 
item.  Sometimes it is more useful to find the length excluding 
trailing blanks.  The next function does just that, using {\tt LEN} in the 
process. 
\begin{verbatim}
      INTEGER FUNCTION LENGTH(STRING) 
*Returns length of string ignoring trailing blanks 
      CHARACTER*(*) STRING 
      DO 15, I = LEN(STRING), 1, -1 
         IF(STRING(I:I) .NE. ' ') GO TO 20 
15    CONTINUE 
20    LENGTH = I 
      END 
\end{verbatim} 

\subsection{Relational Expressions} 

A relational expression compares the values of two arithmetic 
expressions or two character expressions: the result is a logical 
value, either true or false.  Relational expressions are commonly 
used in {\tt IF} statements, as in this example:  
\begin{verbatim}
      IF(SENSOR .GT. UPPER) THEN 
          CALL COOL 
      ELSE IF(SENSOR .LT. LOWER) THEN 
          CALL HEAT 
      END IF 
\end{verbatim} 

The relational operators have forms such as {\tt .GT.} and {\tt .LT.}
because 
the Fortran character set does not include the usual characters {\tt .} 
and {\tt <}.  Relational expressions are most commonly used in {\tt IF} 
statements, but any logical variable or array element may be used 
to store a logical value for use later on. 
\begin{verbatim}
      CHARACTER*10 OPTION 
      LOGICAL EXIT 
      EXIT = OPTION .EQ. 'FINISH'  
*... 
      IF(EXIT) STOP 'Finish requested' 
\end{verbatim} 

Logical expressions are covered in more detail in the next section. 

\subsubsection*{General Forms of Relational Expression} 

\verb+      + {\em arithmetic-exprn rel-op arithmetic-exprn}\\ 
or\verb+    + {\em character-exprn rel-op character-exprn}\\ 
In either case the resulting expression has the logical type.  The 
relational operator {\em rel-op} can be any of the following: 

\begin{center} 
\begin{tabular}{ll} 
{\tt .EQ.           } & equal to   \\ 
{\tt .GE.           } & greater than or equal to  \\ 
{\tt .GT.           } & greater than  \\ 
{\tt .LE.           } & less than or equal to  \\ 
{\tt .LT.           } & less than  \\ 
{\tt .NE.           } & not equal to  \\ 
\end{tabular} 
\end{center} 

Note that these operators need a decimal point at either end to 
distinguish them from symbolic names.   

\subsubsection*{Arithmetic Comparisons} 

When the two arithmetic values of differing data type are 
compared, a conversion is automatically applied to one of them (as 
in arithmetic expressions) to bring it to the type of the other.  The 
direction of conversion is always:  \\
\centerline{{\em integer}  converts to {\em real}
converts to {\tt complex} or {\em double precision}.} 
When comparing integer expressions, there is a considerable 
difference between the {\tt .LE.} and {\tt .LT.} operators, and similarly 
between {\tt .GE.} and {\tt .GT.}, so that you should consider carefully
what 
action is required in the limiting case before selecting the 
appropriate operator. 

In comparisons involving the other arithmetic types you should 
remember that the value of a number may not be stored exactly.  
This means that it is unwise to rely on tests involving the .EQ. and 
.NE. operators except in special cases, for example if one of the 
values has previously been set to zero or some other small integer.  

There are two restrictions on complex values: firstly they cannot 
be compared at all to ones of double precision type.  Secondly they 
cannot use relational operators other than .EQ. and .NE. because 
there is no simple linear ordering of complex numbers.  

\subsubsection*{Character comparisons} 

A character value can only be compared to another character value; if 
they do not have the same length then the shorter one is padded out with 
blanks to the length of the other before the comparison takes place. 
Tests for equality (or inequality) do not depend on the character code, 
the two strings are just compared character by character until a 
difference is found.  Comparisons using the other operators ({\tt .GE.,} 
{\tt .GT.,} {\tt .LE.,} and {\tt .LT.}) do, however, depend on the local 
character code.  The two expressions are compared one character position 
at a time until a difference is found: the result then depends on the 
relative positions of the two characters in the local collating sequence, 
i.e.\ the order in which the characters appear in the character code 
table. 

The Fortran Standard specifies that the collating sequence used by 
all systems must have the following basic properties:  
\begin{itemize} 
\item  all the upper-case letters are in order, A {\tt <} B {\tt <} C etc. 
\item  all digits are in order, 0 {\tt <} 1 {\tt <} 2 etc. 
\item  all digits precede all letters or vice-versa, 
\item  the blank (space) character precedes letters and digits.   
\end{itemize} 

 
It does not, however, specify whether letters precede digits or 
follow them.  As a result, if strings of mixed text are sorted using 
relational operators the results may be machine dependent.  For 
example, the expression\\ 
\verb?      'APPLE' .LT. 'APRICOT'?\\ 
is always true because at the two strings first differ at the third 
character position, and the letter 'P' precedes 'R' in all Fortran 
collating sequences.  However:\\ 
\verb?      'A1' .GT. 'AONE'?\\ 
will have a value true if your system uses EBCDIC but false if it 
uses ASCII, because the digits follow letters in the former and 
precede them in the latter. 

In order to allow character comparisons to be made in a truly 
portable way, Fortran has provided four additional intrinsic 
functions.  These perform character comparisons using the ASCII 
collating sequence no matter what the native character code of the 
machine.  These functions are: 

\begin{center} 
\begin{tabular}{ll} 
      {\tt LGE(S1, S2)}      & greater than or equal to  \\ 
      {\tt LGT(S1, S2)}      & greater than  \\ 
      {\tt LLE(S1, S2)}      & less than or equal to  \\ 
      {\tt LLT(S1, S2)}      & less than.  \\ 
\end{tabular} 
\end{center} 

They take two character arguments (of any length) and return a 
logical value.  Thus the expression:\\ 
\verb?      LGT('A1', 'AONE')?\\ 
will always have the value false.   

Character comparisons are case-sensitive on machines which have 
lower-case letters in their character set.  It is advisable to convert 
both arguments to the same case beforehand. 

\subsubsection*{Guidelines} 

Systems which supports both upper and lower-case characters are 
usually case-sensitive: before testing for the presence of particular 
keywords or commands it is usually best to convert the input 
string to a standard case, usually upper-case.  Unfortunately there 
are no standard intrinsic functions to do this, though many systems 
provide them as an extension. 

In character sorting operations where the strings contain mixtures 
of letters, digits, or other symbols, you should use the intrinsic 
functions to make the program portable.  In other character 
comparisons, however, the relational operator notation is probably 
preferable because it has a more familiar form and may be slightly 
more efficient. 

\subsection{Logical Expressions} 

Logical expressions can be used in logical assignment statements, 
but are most commonly encountered in {\tt IF} statements where there 
is a compound condition, for example: 
\begin{verbatim}
       IF(AGE .GE. 60 .OR. (STATUS .EQ. 'WIDOW' .AND. 
     $   NCHILD .GT. 0) THEN 
\end{verbatim} 
This combines the values of three relational expressions, two of 
them comparing arithmetic values, the other character values.  The 
logical operators such as {\tt .AND.} and {\tt .OR.} also need decimal
points 
at either end to distinguish them from symbolic names.  The {\tt .OR.} 
operator performs an inclusive or, the exclusive or operator is 
called {\tt .NEQV.}. 

\subsubsection*{Rules} 

A logical expression can have any of the following forms: 
\begin{itemize} 
\item       {\em logical-term} 
\item       {\tt .NOT.}  {\em logical-term} 
\item       {\em logical-expression}   ~~{\em logical-operator}   ~~{\em 
logical-term} 
\end{itemize} 
Where: {\em logical-term} can be any of the following: 
\begin{itemize} 
\item logical constant (literal or named), 
\item  logical variable, 
\item  logical array element, 
\item  logical function reference, 
\item logical expression enclosed in parentheses, 
\item relational expression. 
\end{itemize} 

and the logical operator can be any of the following: 

\begin{center} 
\begin{tabular}{ll} 
{\tt .AND.     } & logical and  \\ 
{\tt .OR.      } & logical inclusive or  \\ 
{\tt .EQV.     } & logical equivalence  \\ 
{\tt .NEQV.    } & logical non-equivalence (i.e.\ exclusive or).  \\ 
\end{tabular} 
\end{center} 

Note that the rules of logical expressions only allow two 
successive operators to occur if the second of them is the unary 
operator .NOT. which negates the value of its operand.  The effects 
of the four binary logical operators are shown in the table below for 
the four possible combinations of operands, x and y. 

\begin{center} 
\begin{tabular}{cccccc} 
\hline 
x & y & x {\tt .AND.} y & x {\tt .OR.} y & x {\tt .EQV.} y & x {\tt
.NEQV.} y \\ 
\hline 
false & false & false & false & true & false \\ 
true & false & false & true & false & true \\ 
false & true & false & true & false & true \\ 
true & true & true & true & true & false \\ 
\hline 
\end{tabular} 
\end{center} 

Note that a logical expression can have operands which are 
complete relational expressions, and these can in turn contain 
arithmetic expressions.  The complete order of precedence of the 
operators in a general expression is as follows: 
\begin{enumerate} 
\item arithmetical operators (in the order defined in section 
          6.1 above). 
\item relational operators 
\item {\tt .NOT.} 
\item {\tt .AND.} 
\item {\tt .OR.}  
\item {\tt .EQV. } and  {\tt .NEQV.} 
\end{enumerate} 

If the operators .EQV. and .NEQV. are used at the same level in an 
expression they are evaluated from left to right. 

These rules reduce the need for parentheses in logical 
expressions, thus:\\ 
\verb?      (X .GT. A) .OR. (Y .GT. B)?\\ 
would have exactly the same meaning if all the parentheses had 
been omitted. 

A Fortran system is not required to evaluate every term in a logical 
expression completely if its value can be determined more simply.  
In the above example, if X had been greater than A then it would 
not be necessary to compare Y and B for the expression would 
have been true in either case.  This improves efficiency but means 
that functions with side-effects should not be used.  

\subsubsection*{Guidelines} 

Complicated logical and relational expressions can be hard to read 
especially if they extend on to several successive lines.  It helps to 
line up similar conditions on successive lines, and to use 
parentheses.   

\subsection{Logical Assignment Statements} 

A logical assignment statement has the form: \\
\centerline{ {\em logical-var} = {\em logical-expression}} 
Where the {\em logical-var} can be a logical variable or array element.  
Logical variables and array elements are mainly used to store the 
values of relational expressions until some later point where they 
are used in {\tt IF} statements.   

\section{Control Statements} 

Executable statements are normally executed in sequence except as 
specified by control statements.  The {\tt END=} and {\tt ERR=} keywords 
of input/output statements can also affect the execution sequence. 

\subsection{Control Structures} 

\subsubsection*{Branches} 

The best way to select alternative paths through a program is to use 
the block-{\tt IF} structure: this may comprise a single block to be 
executed when a specified condition is true or several blocks to 
cover several eventualities.  Where the {\tt IF}-block would only 
contain one statement it is possible to use an abbreviated form 
called (for historical reasons) the logical-IF statement.  

There is also a computed {\tt GO TO} statement which can produce a 
multi-way branch similar to the ``case'' statements of other 
languages.   

\subsubsection*{Loops} 

Another fundamental requirement is that of repetition.  If the 
number of cycles is known in advance then the {\tt DO} statement 
should be used.  This also controls a block of statements known as 
the {\tt DO}-loop.  A {\tt CONTINUE} statement usually marks the end of 
a {\tt DO}-loop. 

 Fortran has no direct equivalent of the ``do while'' and ``repeat 
until'' forms available in some program languages for loops of an 
indefinite number of iterations, but they can be constructed using 
simple {\tt GO TO} and {\tt IF} statements.  

\subsubsection*{Other Control Statements} 

The {\tt STOP} statement can be used to terminate execution.  Other 
statements which affect execution sequence are described in other 
sections: the {\tt END} statement was covered in section 4.7; procedure 
calls including the {\tt CALL} and {\tt RETURN} statements are described 
in section 9. 

\subsection{\texttt{IF}-Blocks} 
The simplest form of {\tt IF}-block looks like this: 
\begin{verbatim}
      IF(N .NE. 0) THEN 
         AVERAG = SUM / N 
         AVGSQ  = SUMSQ / N 
      END IF 
\end{verbatim} 

The statements in the block are only executed if the condition is 
true.  In this example the statements in the block are not executed 
if N is zero in order to avoid division by zero. 

The {\tt IF}-block can also contain an {\tt ELSE} statement to handle the 
alternative:  
\begin{verbatim}
      IF(B**2 .GE. 4.0 * A * C) THEN 
          WRITE(UNIT=*,FMT=*)'Real roots' 
      ELSE 
          WRITE(UNIT=*,FMT=*)'No real roots' 
      END IF 
\end{verbatim} 

Since the {\tt IF} statement contains a logical expression its value can 
only be true or false, thus one or other of these blocks will always 
be executed. 

If there are several alternative conditions to be tested, they can be 
specified with {\tt ELSE IF} statements:  
\begin{verbatim}
      IF(OPTION .EQ. 'PRINT') THEN 
           CALL OUTPUT(ARRAY) 
      ELSE IF(OPTION .EQ. 'READ') THEN 
           CALL INPUT(ARRAY) 
      ELSE IF(OPTION .EQ. 'QUIT') THEN 
           CLOSE(UNIT=OUT) 
           STOP 'end of program' 
      ELSE 
           WRITE(UNIT=*,FMT=*)'Incorrect reply, try again...' 
      END IF 
\end{verbatim} 

There can be any number of ELSE IF blocks but in each case one, 
and only one, will be executed each time.  Without an ELSE block 
on the end and nothing would have happened when an invalid 
option was selected.  

\subsubsection*{Block-IF General Rules} 

The general form of the block-if structure is as follows: 
\begin{verbatim}
      IF( logical-expression ) THEN 
            a block of statements  
      ELSE IF( logical-expression ) THEN 
            another block of statements  
      ELSE 
            a final block of statements 
      END IF 
\end{verbatim} 
The {\tt IF THEN, ELSE IF}, and {\tt ELSE} statements each govern one 
block of statements.  There can be any number of {\tt ELSE IF} 
statements.  The {\tt ELSE} statement (together with its block) is also 
optional, and there can be at most one of these.   

The first block of statements is executed only if the first expression is 
true.  Each block after an {\tt ELSE IF} is executed only if none of the 
preceding blocks have been executed and the attached {\tt ELSE IF} 
expression is true.  If there is an {\tt ELSE} block it is executed only 
if none of the preceding blocks has been executed. 

After a block has been executed control is transferred to the 
statement following the {\tt END IF} statement at the end of the 
structure (unless the block ends with some statement which 
transfers control elsewhere). 

Any block can contain a complete block-IF structure properly 
nested within it, or a complete {\tt DO}-loop, or any other executable 
statements (except {\tt END}). 

It is illegal to transfer control into any block from outside it, but 
there is no restriction on transferring control out of a block. 

The rules for logical expressions are covered in section 7.7.   

\subsubsection*{Guidelines} 

The indentation scheme shown in the examples above is not 
mandatory but the practice of indenting each block by a few 
characters relative to the rest of the program is strongly 
recommended.  It makes the structure of the block immediately 
apparent and reduces the risk of failing to match each IF with an 
END IF.  An indenting scheme is especially useful when {\tt IF}-blocks 
are nested within others.  For example: 
\begin{verbatim}
      IF(POWER .GT. LIMIT) THEN 
          IF(.NOT. WARNED) THEN 
              CALL SET('WARNING') 
              WARNED = .TRUE. 
          ELSE 
              CALL SET('ALARM') 
          END IF 
      END IF 
\end{verbatim}
The limited width of the statement field can be a problem when {\tt
IF}-blocks are nested to a very great depth: but this tends to mean that
the program unit is getting too complicated and that it will usually be
beneficial to divide it into subroutines. If you accidentally omit an
{\tt END IF} statement the compiler will flag the error but will not know
where you forgot to put it.  In such cases the compiler may get confused
and generate a large number of other error messages.

When an {\tt IF}-block which is executed frequently contains a large 
number of {\tt ELSE IF} statements it will be slightly more efficient to 
put the most-likely conditions near the top of the list as when they 
occur the tests lower down in the list will not need to be executed. 

\subsection{\texttt{DO}-Loops} 

The {\tt DO} statement controls a block of statements which are 
executed repeatedly, once for each value of a variable called the 
loop-control variable.  The number of iterations depends on the 
parameters of the {\tt DO} statement at the heads of the loop.  The first 
item after the keyword ``{\tt DO}'' is the label which is attached to the 
last statement of the loop.  For example: 
\begin{verbatim}
*Sum the squares of the first N elements of the array X 
      SUM = 0.0 
      DO 15, I = 1,N 
          SUM = SUM + X(I)**2 
15    CONTINUE 
\end{verbatim} 
If we had wanted only to sum alternate elements of the array we 
could have used a statement like:\\ 
\verb?      DO 15,I = 1,N,2?\\ 
and then the value of I in successive loops would have been 1, 3, 
5, etc.  The final value would be N if N were odd, or only to N-1 
if N were even.  If the third parameter is omitted the step-size is 
one; if it is negative then the steps go downwards.  For example 
\begin{verbatim}
      DO 100,I = 5,1,-1 
          WRITE(UNIT=*,FMT=*) I**2 
100   CONTINUE 
\end{verbatim} 
will produce 5 records containing the values 25, 16, 9, 4, and 1 
respectively. 

Loops can be nested to any reasonable depth.  Thus the following 
statements will set the two dimensional array FIELD to zero. 
\begin{verbatim}
      REAL FIELD(NX, NY) 
      DO 50, IY = 1,NY 
         DO 40, IX = 1,NX 
             FIELD(IX,IY) = 0.0 
40       CONTINUE 
50    CONTINUE 
\end{verbatim} 

\subsubsection*{General Form of {\tt DO} Statement} 

The {\tt DO} statement has two forms:\\ 
\verb+      DO+ {\em label} , {\em variable} {\tt =} {\em start , limit, 
step}\\ 
\verb+      DO+ {\em label} , {\em variable} {\tt =} {\em start , limit}\\ 
In the second form the step size is implicitly one.   

The {\em label} marks the final statement of the loop.  It must be
attached 
to an executable statement further on in the program unit.  The 
rules permit this statement to be any executable statement except 
another control statement, but it is strongly recommended that you 
use the {\tt CONTINUE} statement here.  {\tt CONTINUE} has no other 
function except to act as a dummy place-marker. 

The comma after the label is optional but, as noted in section 1.4, 
is a useful precaution. 

The {\em variable} which follows is known as the loop control variable 
or loop index; it must be a variable (not an array element) but may 
have integer, real, or double precision type.  

The {\em start,} {\em limit,} and {\em step} values may be expressions of 
any form of integer, real, or double precision type.  If the step value 
is present it must not be zero, if omitted it is taken as one.  The 
number of iterations is computed before the start of the first one, using 
the formula: \\
\centerline{ iterations = MAX(INT(0, (limit - start + step) / step))} 
Note that if the limit value is less than start the iteration count is 
zero unless step is negative.  A zero iteration count is permitted but 
means that the contents of the loop will not be executed at all and 
control is transferred to the first statement after the end of the loop.  
The loop control variable does not necessarily reach the limiting 
value, especially if the step-size is larger than one. 

Statements within the loop are permitted to alter the value of the 
expressions used for start, limit, or step but this has no effect on 
the iteration count which is fixed before the first iteration starts. 

The loop control variable may be used in expressions but a new 
value must not be assigned to it within the loop. 

{\tt DO}-loops may contain other {\tt DO}-loops completely nested within
them 
provided that a different loop control variable is used in each one. 
Although it is permissible for two different loops to terminate on the 
same statement, this can be very confusing.  It is much better to use a 
separate {\tt CONTINUE} statement at the end of each loop.  Similarly 
complete {\tt IF}-blocks may be nested within {\tt DO}-loops, and
vice-versa. 

Other control statements may be used to transfer control out of the 
range of a {\tt DO}-loop but it is illegal to try to jump into a loop from 
outside it.  If you exit from a loop prematurely in this way the loop 
control variable keeps its current value and may be used outside to 
determine how many loops were actually executed. 

After the normal termination of a {\tt DO}-loop the loop control 
variable has the value it had on the last iteration plus one extra 
increment of the step value.  Thus with: 
\begin{verbatim}
       DO 1000, NUMBER = 1,100,3 
1000   CONTINUE 
\end{verbatim} 
On the last iteration NUMBER would be 99, and on exit from the 
loop NUMBER would be 102.  This provision can be useful in the 
event of exit from a loop because of some error: 
\begin{verbatim}
       PARAMETER (MAXVAL = 100) 
       REAL X(MAXVAL) 
       DO 15, I = 1,MAXVAL 
            READ(UNIT=*, FMT=*, END=90) X(I) 
15     CONTINUE 
90     NVALS = I - 1 
\end{verbatim} 
The action of the statement labelled 90 is to set NVALS to the 
number of values actually read from the file whether there was a 
premature exit because the end-of-file was detected or it reached 
the end of the array space at MAXVAL. 

\subsubsection*{Guidelines} 

If you use a loop-control variable of any type other than integer there 
is a risk that rounding errors will accumulate as it is incremented 
repeatedly.  In addition, if the expressions for the start, limit, and 
step values are not of integer type the number of iterations may not 
be what you expect because the formula uses the INT function (not 
NINT).  None of these problems can occur if integer quantities are 
used throughout the {\tt DO} statement.  

\subsection{Logical-IF Statement} 

The logical-IF statement is best regarded as a special case of the 
{\tt IF}-block when it only contains one statement. Thus: 
\begin{verbatim}
       IF(E .NE. 0.0) THEN 
           RECIPE = 1.0 / E 
       END IF 
\end{verbatim} 
can be replaced by a single logical-IF statement:\\ 
\verb?      IF(E .NE. 0.0) RECIPE = 1.0 / E?

The general form of the logical-IF statement is:\\ 
\verb+      IF(+ {\em logical-expression} {\tt )} {\em statement}\\ 
The statement is executed only if the {\em logical expression} has a true 
value.  Any executable statement can follow except {\tt DO, IF, 
ELSE IF, ELSE, END IF}, or {\tt END}. 

\subsection{Unconditional \texttt{GO TO} Statement} 

The unconditional {\tt GO TO} statement simply produces a transfer of 
control to a labelled executable statement elsewhere in the program 
unit.  Its general form is:\\ 
\verb+      GO TO+ {\em label}\\  
Note that control must not be transferred into an {\tt IF}-block or a
{\tt DO}-loop from outside it.  

\subsubsection*{Guidelines} 

The unconditional {\tt GO TO} statement makes it possible to construct 
programs with a very undisciplined structure; such programs are 
usually hard to understand and to maintain.  Good programmers 
use {\tt GO TO} statements and labels very sparingly.  Unfortunately it 
is not always possible to avoid them entirely in Fortran because of 
a lack of alternative control structures. 

The next example finds the highest common factor of two integers 
M and N using a Euclid's algorithm.  It can be expressed roughly: 
      while (M  N)  
            subtract the smaller of M and N from the other 
      repeat until they are equal. 
\begin{verbatim}
       PROGRAM EUCLID 
       WRITE(UNIT=*, FMT=*) 'Enter two integers' 
       READ(UNIT=*, FMT=*) M, N 
10       IF(M .NE. N) THEN 
            IF(M .GT. N) THEN 
                M = M - N 
            ELSE 
                N = N - M 
            END IF 
            GO TO 10 
       END IF 
       WRITE(UNIT=*, FMT=*)'Highest common factor = ', M 
       END 
\end{verbatim} 

\subsection{Computed \texttt{GO TO} Statement} 

The computed {\tt GO TO} statement is an alternative to the block-IF 
when a large number of options are required and they can be 
selected by the value of an integer expression.  The general form 
of the statement is:\\ 
\verb+      GO TO(+ {\em label1, label2, ... labelN} {\tt ),} {\em 
integer-expression}\\ 
The comma after the right parenthesis is optional. 

The {\em expression} is evaluated; if its value is one then control is 
transferred to the statement attached to the first label in the list; if 
it is two control goes to the second label, and so on.  If the value 
of the expression is less than one or higher than N (where there are 
N labels in the list) then the statement has no effect and execution 
continues with the next statement in sequence.  The same label 
may be present more than once in the list. 

The computed {\tt GO TO} suffers from many of the same drawbacks 
as the unconditional {\tt GO TO}, since if its branches are used without 
restraint they can become impenetrable thickets.  The best way is 
to follow the computed {\tt GO TO} statement with the sections of code 
in order, all except the last terminated with its own unconditional 
{\tt GO TO} to transfer control to the end of the whole structure. 

Any computed {\tt GO TO} structure could be replaced by an {\tt IF}-block 
with a suitable number of {\tt ELSE IF} clauses.  If there are a very 
large number of cases then this would be a little less efficient; this 
has to be balanced against the increased clarity of the IF structure 
compared to the label-ridden {\tt GO TO}.  

An example of the use of the computed {\tt GO TO} is given here in a 
subroutine which computes the number of days in a month, given 
the month number MONTH between 1 and 12, and the four-digit 
year number in YEAR.  Note that each section of code except the 
last is terminated with a {\tt GO TO} statement to escape from the 
structure. 
\begin{verbatim}
      SUBROUTINE CALEND(YEAR, MONTH, DAYS) 
      INTEGER YEAR, MONTH, DAYS 
      GO TO(310,280,310,300,310,300,310,310,300,310,300,310)MONTH 
*           Jan Feb Mar Apr May Jun Jly Aug Sep Oct Nov Dec 
      STOP 'Impossible month number' 
*February: has 29 days in leap year, 28 otherwise. 
280   IF(MOD(YEAR,400) .EQ. 0  .OR. (MOD(YEAR,100) .NE. 0 
     $                          .AND. MOD(YEAR,4) .EQ. 0)) THEN 
          DAYS = 29 
      ELSE 
          DAYS = 28 
      END IF 
      GO TO 1000 
*   Short months 
300   DAYS = 30 
      GO TO 1000 
*   Long months 
310   DAYS = 31 
* return the value of DAYS 
1000  END  
\end{verbatim} 

\subsection{\texttt{STOP} Statement} 

The {\tt STOP} statement simply terminates the execution of the 
program and returns control to the operating system.  Its general 
form is: \\ 
\verb?      STOP '? {\em character constant} {\tt '} \\ 
The character constant (which must be a literal and not named constant) is 
optional: if present its value is ``made available'' to the user; usually 
it the message appears on your terminal.  For compatibility with 
Fortran66 it is possible to use a string of one to five decimal digits 
instead of the character constant. 

Ideally a program should only return control to the operating system from 
one point, the end of the main program, where the {\tt END} statement 
does all that is necessary.  In practice, even in the best-planned 
programs, situations can arise which make it pointless to continue.  If 
these are detected in the main program there is always the option of 
jumping to the {\tt END} statement, but within procedures there may be no 
choice but to use a {\tt STOP} statement. 

\section{Procedures} 

Any set of computations can be encapsulated in a procedure.  The 
main purpose of a procedure is to allow the same set of operations 
to be invoked at different points in a program.  Procedures also 
make it possible to use the same code in several different 
programs.  It is good practice to split a large program into sections 
whenever it becomes too large to be handled conveniently in one 
piece.  The optimum size of a program unit is quite small, probably 
no more than 100 lines. 

Four different forms of procedure can be used in Fortran 
programs:- 
\begin{itemize} 
\item Intrinsic functions 
\item Statement functions 
\item External functions (also known as function subprograms) 
\item Subroutines. 
\end{itemize}
Intrinsic functions are provided automatically by the Fortran system,
whereas the other three forms of procedure are user-written.  Statement
functions, which are defined with the statement function statement, can
only be used in the program unit in which they were defined and are
subject to other special restrictions.  External functions and
subroutines are two alternative forms of external procedure: each is
specified as a separate program unit and can be used (with only a few
restrictions) anywhere else in the program.

\subsection{Intrinsic Functions} 

Intrinsic functions have a number of unique properties.  The data 
type of each intrinsic function is known to the Fortran system and 
is not subject to the normal rules.  {\tt IMPLICIT} and type statements 
alone have no effect on them.  Some intrinsic functions have 
generic names: when these are used the compiler selects the 
appropriate specific function according to the data type of the 
arguments. 

A few intrinsic functions such as MAX, MIN, and CMPLX, are 
allowed to have a variable number of arguments, but all of the 
arguments must have the same data type.  User-written procedures 
cannot have optional arguments or generic type. 

Although intrinsic functions can be used in any program unit, 
their names are not global, nor are they reserved words.  It is, 
however, best to avoid choosing a name for a variable or array 
which is identical to that of an intrinsic function.  It may cause 
confusion and in the long run it may make it more difficult to 
enhance the program.  A name clash is more serious if it involves 
an external function or subroutine, for in this case the external 
procedure name must be specified in an {\tt EXTERNAL} statement to 
resolve the ambiguity.  By this means it is possible to substitute an 
external function of your own for one of the intrinsic functions. 

The Fortran Standard specifies a fairly extensive set of intrinsic 
functions which must always be available but it does not prevent 
the provision of additional ones.  Many systems provide additional 
intrinsic functions which, for example, obtain the current date and 
time, generate pseudo-random numbers, or evaluate Gaussian 
probability.  The main drawback in using non-standard functions 
is that you may have to find a substitute if your program is moved 
to another system which does not have the same extensions.  

The standard intrinsic functions for the arithmetic types are 
described in detail in section 6.2; those used with character-strings 
are covered in section 7.5.  A complete alphabetical list is provided 
in the appendix. 

\subsection{Statement Functions} 

Statement functions can be defined within any executable program 
unit by means of statement function statements.  They can only be 
used, however, within the same program unit.  Although statement 
functions have limited uses, they are unjustly neglected by many 
programmers. 

The statement function statement resembles an ordinary 
assignment statement.  For example:\\ 
\verb?      FAHR(CELS) = 32.0 + 1.8 * CELS?\\ 
The function FAHR converts a temperature in degrees Celsius to 
its equivalent in Fahrenheit.  Thus FAHR(20.0) would return a 
value 68.0 approximately.   

A statement function can have any number of dummy arguments 
(such as CELS above) all of which must appear in the expression 
on the right-hand side; this expression may also include constants, 
variables, or array elements used elsewhere in the program.  When 
the function is called the current values of these items will be used.  
For example:  
\begin{verbatim}
       REAL M1, M2, G, R 
       NEWTON(M1, M2, R) = G * M1 * M2 / R**2 
\end{verbatim} 

A reference to the function in an assignment statement such as:\\ 
\verb?      FORCE = NEWTON(X, Y, DIST) ?\\ 
will return a value depending on the values of the actual arguments 
X, Y, and DIST, and that of the variable G at the time the function 
is referenced.  

Definitions of statement functions can also include references to 
intrinsic functions, external functions, or previously defined 
statement functions:  
\begin{verbatim}
      PARAMETER (PI = 3.14159265, DTOR = PI/180.0) 
      SIND(THETA) = SIN(THETA * DTOR) 
      COSD(THETA) = COS(THETA * DTOR) 
      TAND(THETA) = SIND(THETA) / COSD(THETA) 
\end{verbatim} 
These definitions allow trigonometry on angles specified in 
degrees rather than radians. 

The scope of each dummy argument name (such as {\tt THETA} above) 
is that of the statement alone; these names can be used elsewhere 
in the program unit as variables of the same data type with no 
effect whatever on the evaluation of the function.   

Statement functions can have any data type; the name and 
arguments follow the normal type rules.  They can be useful in 
character handling, for example: 
\begin{verbatim}
      LOGICAL MATH, DIGIT, DORM 
      CHARACTER C*1 
      DIGIT(C) = LGE(C, '0') .AND. LLE(C, '9') 
      MATH(C)  = INDEX('+-*/', C) .NE. 0 
      DORM(C)  = DIGIT(C) .OR. MATH(C) 
\end{verbatim} 
These three functions each return a logical value when presented with a 
single character argument: {\tt DIGIT} tests to see whether the character 
is a digit, {\tt MATH} whether it is an operator symbol, and {\tt DORM} 
will test for either condition.  Note the use of the lexical comparison 
functions {\tt LGE} and {\tt LLE} in the definition of {\tt DIGIT} which 
make it completely independent of the local character code. 

\subsubsection*{Statement Function Rules } 

Statement function statements must appear after any the 
specification statements but before all executable statements in the 
program unit.  They may be intermixed with {\tt DATA} and {\tt FORMAT}  
statements.  The general form is:\\ 
\verb+      + {\em function} {\tt (} {\em dummy1,} {\em dummy2,} ... {\em 
dummyN} {\tt ) =} {\em expression}\\ 
The {\em function} may have any data type; the {\em expression} will
normally 
have the same data type but if both have an arithmetic type then 
the normal conversion rules for arithmetic assignment statements 
apply.  

The name of the function must be distinct from all other symbolic 
names in the program unit.  It may appear in type statements but 
not in other specification statements.  (There is one exception: a 
common block is permitted to have the same name as a statement 
function but since common block names always appear between 
slashes there is little risk of confusion).  If the function has 
character type its length must be an integer constant expression. 

The dummy arguments are simply symbolic names.  A name may 
not appear more than once in the same list.  These names may be 
used elsewhere in the program unit as variables of the same data 
type.        

The expression must contain the dummy arguments as operands.  
The operands may also include: 
\begin{itemize} 
\item  literal constants, named constants, variables, and array 
elements; these will have their values at the time the 
function is executed and must then be defined.  
\item  references to intrinsic and external functions, 
\item  references to statement functions defined earlier in the 
same program unit, 
\item  complete expressions enclosed in parentheses. 
\end{itemize} 

 
Note that character substrings are not permitted.  The variables and 
array elements used in the expression must be defined at the time 
that the function reference is executed.  

\subsubsection*{Guidelines} 

Although statement functions have a limited role to play in 
programs because they can only be defined in a single statement, 
references to statement functions they may be executed more 
efficiently than references to external functions; many modern 
compilers expand statement function references to in-line code 
when it is advantageous to do so. 

If the same statement function is needed in more than one program unit it 
would is possible to use an {\tt INCLUDE} facility to provide the same 
definition each time, but it will usually be better to use an external 
function instead. 

\subsection{External Procedures} 

There are two forms of external procedure, both of which take the 
form of a complete program unit. 
\begin{itemize} 

\item  External functions, which are specified by a program 
unit starting with a {\tt FUNCTION} statement. They are 
executed whenever the corresponding function is used 
as an operand in an expression. 
\item  Subroutines, which are specified by a program unit 
starting with a {\tt SUBROUTINE} statement.  They are 
executed in response to a {\tt CALL} statement. 
\end{itemize} 

In either form the last statement of the program unit must be an 
{\tt END} statement.  Any other statements (except {\tt PROGRAM} or 
{\tt BLOCK DATA} statements) may be used within the program unit.  

 
There are two statements provided especially for use in external 
procedures.  The {\tt SAVE} statement ensures that the values of local 
variables and arrays are preserved after the procedure returns 
control to the calling unit: these values will then be available if the 
procedure is executed subsequently.  The {\tt RETURN} statement may 
be used to terminate the execution of the procedure and cause an 
immediate return to the control of the calling unit.  Execution of 
the {\tt END} statement at the end of the procedure has exactly the 
same effect.  Both of these are described in full later in the section. 

Most Fortran systems also allow external procedures to be 
specified in languages other than Fortran: they can be called in the 
same way as Fortran procedures but their internal operations are, 
of course, beyond the scope of this book.   

It is best to think of the subroutine as the more general form of 
procedure; the external function should be regarded as a special 
case for use when you only need to return a single value to the 
calling unit.   

Here is a simple example of a procedure which converts a time of 
day in hours, minutes, and seconds into a count of seconds since 
midnight.  Since only one value needs to be returned, the 
procedure can have the form of an external function. (In fact this 
is such a simple example that it would have been possible to define 
it as a statement function.) 
\begin{verbatim}
*TSECS converts hours, minutes, seconds to total seconds. 
       REAL FUNCTION TSECS(NHOURS, MINS, SECS) 
       INTEGER NHOURS, MINS 
       REAL SECS 
       TSECS = ((NHOURS * 60) + MINS) * 60 + SECS 
       END 
\end{verbatim} 
Thus if we use a function reference like TSECS(12,30,0.0) in an 
expression elsewhere in the program it will convert the time to 
seconds since midnight (about 45000.0 seconds in this case).  The 
items in parentheses after the function name :\\ 
\verb?      (12,30,0.0) ?\\ 
are known as the actual arguments of the function; these values are 
transferred to the corresponding dummy arguments\\ 
\verb?      (NHOURS, MINS, SECS)?\\ 
of the procedure before it is executed.  In this example the 
argument list is used only to transfer information into the function 
from outside, the function name itself returns the required value to 
the calling program.  In subroutines, however, there is no function 
name to return information but the arguments can be used for 
transfers in either direction, or both.  The rules permit them to be 
used in this more general way in functions, but it is a practice best 
avoided. 

The next example performs the inverse conversion to the TSECS 
function.  Since it has to return three values to the calling program 
unit the functional form is no longer appropriate, and a subroutine 
will be used instead. 
\begin{verbatim}
*Subroutine HMS converts TIME in seconds into hours, mins,secs. 
      SUBROUTINE HMS(TIME, NHOURS, MINS, SECS) 
      REAL TIME, SECS 
      INTEGER NHOURS, MINS 
      NHOURS = INT(TIME / 3600.0) 
      SECS   = TIME - 3600.0 * NHOURS 
      MINS   = INT(SECS / 60.0) 
      SECS   = TIME - 60.0 * MINS 
      END 
\end{verbatim} 
In this case the subroutine could be executed by using a statement 
such as: 
\begin{verbatim}
       CALL HMS(45000.0, NHRS, MINS, SECS) 
       WRITE(UNIT=*, FMT=*) NHRS, MINS, SECS 
\end{verbatim} 
Here the first argument transfers information into the subroutine, 
the other three are used to return the values which it calculates.  
You do not have to specify whether a particular argument is to 
transfer information in or out (or in both directions), but there are 
rules about the form of actual argument that you can use in each 
case.  These are explained in full below. 

\subsubsection*{Procedure Independence} 

Each program unit has its own independent set of symbolic names 
and labels.  Type statements and {\tt IMPLICIT} statements may be 
used to specify their data types.   

External procedures can themselves call any other procedures and 
these may call others in turn, but procedure are not allowed to call 
themselves either directly or indirectly; that is recursive calling is 
not permitted in Fortran. 

\subsubsection*{Information Transfer} 

Information can be transferred to and from an external procedure 
by any of three methods. 
\begin{itemize} 
\item  An argument list: as shown in the two examples above.  
This is the preferred method of interfacing as it is the 
most flexible and modular.  It is described in detail in 
the remainder of this section. 
\item  Common blocks: these are lists of variables or arrays 
which are stored in areas of memory shared 
between two or more program units.  They are useful in 
special circumstances when procedures have to be 
coupled closely together, but are otherwise less 
satisfactory.  Common blocks are covered in detail in 
section 12. 
\item  External files: interfacing via external files is neither 
convenient nor efficient but it is mentioned here to point 
out that external files are global.  Once a file has been 
opened in any program unit it can be accessed anywhere 
in the program provided that the appropriate I/O unit 
number is available.  A unit number can be passed into 
a procedure as an integer argument. 
\end{itemize} 

 
\subsubsection*{Procedure Execution} 

It is not necessary to know how the Fortran system actually 
transfers information from one procedure to another to make use 
of the system, but the rules governing the process are somewhat 
complicated and it may be easier to understand them if you 
appreciate the basis on which they have been formulated.  The 
rules in the Fortran Standard are based on the assumption that the 
address of an actual argument is transferred in each case: this may 
or may not be true in practice but the properties will be the same 
as if it is. 

This means that when you reference a dummy variable or assign a 
new value to one you are likely to be using the memory location 
occupied by the actual argument.  By this means even large arrays 
can be transferred efficiently to procedures.  A slight modification 
of this system is needed for items of character type so that the 
length of the item can be transferred as well as its address.   

When a function reference or {\tt CALL} statement is executed any 
expressions in the argument list are evaluated; the addresses of the 
arguments are then passed to the procedure.  When it returns 
control this automatically makes updated values available to the 
corresponding items in the actual argument list. 

\subsubsection*{Functions with Side-effects } 

The rules of Fortran allow functions to have side-effects, that is to 
alter their actual arguments or to change other variables within 
common blocks.  Functions with side-effects cannot be used in 
expressions where any of the other operands of the expression 
would be affected, nor can they be used in subscript or substring 
references when any other expression used in the same references 
would be affected.  This rule ensures that the value of an 
expression cannot depend arbitrarily on the way in which the 
computer chooses to evaluate it.   

There are also restrictions on functions which make use of 
input/output statements even on internal files: these cannot be used 
in expressions in other I/O statements.  This is to avoid the I/O 
system being used recursively.  

By far the best course is to use the subroutine form for any 
procedure with side-effects.  

\subsection{Arguments of External Procedures} 

Arguments can pass information into a procedure or out from it, or 
in both directions.  This just depends on the way that the dummy 
argument is used within the procedure.  Although any argument 
order is permitted, it is common practice to put input arguments 
first, then those that pass information both ways, and then 
arguments which just return information from the procedure. 

The rules for argument association are the same for both forms of 
external procedure.  The list of dummy arguments (sometimes 
called formal arguments) of an external procedure is specified in 
its {\tt FUNCTION} or {\tt SUBROUTINE} statement.  There can be any 
number of arguments, including none at all.  If there are no 
arguments then the parentheses can be omitted in the {\tt CALL} and 
{\tt SUBROUTINE} statement but not in a {\tt FUNCTION} statement or 
function reference.  

The dummy argument list is simply a list of symbolic names which 
can represent any mixture of 
\begin{itemize} 
\item  variables  
\item  arrays 
\item  procedures.   
\end{itemize} 

A name cannot, of course, appear twice in the same dummy 
argument list.   

Dummy variables, arrays, and procedures are distinguished only by 
the way that they are used within the procedure.  The dimension 
bounds of a dummy arrays must be specified in a subsequent type 
or {\tt DIMENSION} statement; dummy procedures must appear in a 
{\tt CALL} or {\tt EXTERNAL} statement or be used in a function 
reference; anything else is, by elimination, a dummy argument 
variable. 

Dummy argument variables and arrays can be used in executable 
statements in just the same way as local items of the same form, 
but they cannot appear in {\tt SAVE,} {\tt COMMON,} {\tt DATA,} or 
{\tt EQUIVALENCE} statements. 

\subsubsection*{Argument Association} 

The actual arguments of the function reference or {\tt CALL} statement 
become associated with the corresponding dummy arguments of 
the {\tt FUNCTION} or {\tt SUBROUTINE} statement.  The main rules are 
as follows: 
\begin{itemize} 

\item  There must be the same number of actual and dummy 
arguments; they are associated solely by their position in 
the two lists.  Optional arguments are not permitted in 
Fortran77. 
\item  If the dummy argument is a variable, array, or procedure 
used as a function then the corresponding actual 
argument must have the same data type.  
\item  If the dummy argument is an array then its array bounds 
must not be larger than those of the corresponding actual 
argument.  Alternatively the dimension bounds of a 
dummy array can be passed in by means of other 
procedure arguments to form an adjustable array.  This option 
and the assumed-size array are both described in section 
9.6. 
\item  If the dummy argument is a character item then its 
length must not be greater than that of the corresponding 
actual argument.  Alternatively there is a passed-length 
option for character arguments: see section 9.5. 
\end{itemize} 

Because program units are compiled independently, it is difficult 
for the compiler to check for mismatches in actual and dummy 
argument lists.  Although mismatches could, in principle, be 
detected by the linker, this rarely seems to happen in practice.  
Errors, particularly mismatches of data type or array bounds, are 
especially easy to make but hard to detect.  Sometimes the only 
indication is that the program produces the wrong answer.  This 
shows how important it is to check procedure interfaces.  

\subsubsection*{Duplicate Arguments } 

The same actual argument cannot be used more than once in a 
procedure call if the corresponding dummy arguments are assigned 
new values.  For example, with: 
\begin{verbatim}
       SUBROUTINE FUNNY(X, Y) 
       X = 2.0 
       Y = 3.0 
       END 
\end{verbatim} 
A call such as: \\ 
\verb?      CALL FUNNY(A, A)?\\ 
would be illegal because the system would try to assign 2.0 
and 3.0 to the variable A in some unpredictable order, so one cannot be
certain of the result.

A similar restriction applies to variables which are returned via a 
common block and also through the procedure argument list.  

\subsection{Variables as Dummy Arguments} 

If the dummy argument of a procedure is a variable and it has a 
value assigned to it within the procedure, then the corresponding 
actual argument can be: 
\begin{itemize} 
\item  a variable, 
\item  an array element, or  
\item  a character substring. 
\end{itemize} 

 
If, however, the dummy variable preserves its initial value 
throughout the execution then the actual argument can be any of 
these three forms above or alternatively: 
\begin{itemize} 
\item  an expression of any form (including a constant). 
\end{itemize} 

The reason for this restrictions is easy to see by considering the 
ways of calling the subroutine SILLY in the next example: 
\begin{verbatim}
       SUBROUTINE SILLY(N, M) 
       N = N + M 
       END 
\end{verbatim} 

If it is called with a statement such as: 
\begin{verbatim}
       NUMBER = 10 
       CALL SILLY(NUMBER, 5) 
\end{verbatim} 
then the value of NUMBER will be updated to 15 as a result of the 
call.  But it is illegal to call the function with a constant as the first 
argument, thus: 
\begin{verbatim}
       CALL SILLY(10, 7) 
\end{verbatim}
because on exit the subroutine will attempt to return the value of 17 to
the actual argument which was specified as the constant ten. The effects
of committing such an error are system-dependent. Some systems detect the
attempt to over-write a constant and issue an error message; others
ignore the attempt and allow the program to continue; but some systems
will actually go ahead and over-write the constant with a new value, so
that if you use the constant 10 in some subsequent statement in the
program you may get a value of 17.  Since this can have very puzzling
effects and be hard to diagnose, it is important to avoid doing this
inadvertently.

If you make use of procedures written by other people you may be 
worried about unintentional effects of this sort.  In principle it 
should be possible to prevent a procedure altering a constant 
argument by turning each one into an expression, for example like 
this:\\ 
\verb?      CALL SILLY(+10, +5)?\\ 
or\\ 
\verb?      CALL SILLY((10), (5))?\\ 
Although either of these forms should protect the constants, it is 
still against the rules of Fortran for the procedure to attempt to 
alter the values of the corresponding dummy arguments.  You will 
have to judge whether it is better to break the rules of the language 
than to risk corrupting a constant. 

\subsubsection*{Expressions, Subscripts, and Substrings} 

If the actual argument contains expressions then these are 
evaluated before the procedure starts to execute; even if the 
procedure later modifies operands of the expression this has no 
effect on the value passed to the dummy argument.  The same rule 
applies to array subscript and character substring expressions. For 
example, if the procedure call consists of:\\ 
\verb?      CALL SUB( ARRAY(N), N, SIN(4.0*N), TEXT(1:N) )?\\ 
and the procedure assigns a new value to the second argument, N, 
during its execution, it has no effect on the other arguments which 
all use the original value of N.  The updated value of N will, of 
course, be passed back to the calling unit. 

\subsubsection*{Passed-length Character Arguments} 

A character dummy argument will have its length set automatically 
to that of the corresponding actual argument if the special length 
specification of {\tt *(*)} is used.   

To illustrate this, here is a procedure to count the number of 
vowels in a character string.  It uses the intrinsic function LEN to 
determine the length of its dummy argument, and the INDEX 
function to see whether each character in turn is in the set 
``AEIOU'' or not. 
\begin{verbatim}
       INTEGER FUNCTION VOWELS(STRING) 
       CHARACTER*(*) STRING 
       VOWELS = 0 
       DO 25, K = 1,LEN(STRING) 
          IF( INDEX('AEIOU', STRING(K:K)) .NE. 0) THEN 
               VOWELS = VOWELS + 1 
          END IF 
25     CONTINUE 
       END 
\end{verbatim} 
Note that the function has a data type which is not the default for 
its initial letter so that it will usually be necessary to specify its 
name in a {\tt INTEGER} statement in each program unit which 
references the function. 

This passed-length mechanism is recommended not only for 
general-purpose software where the actual argument lengths are 
unknown, but in all cases unless there is a good reason to specify 
a dummy argument of fixed length.  

There is one restriction on dummy arguments with passed length: 
they cannot be operands of the concatenation operator (//) except 
in assignment statements.  Note that the same form of length 
specification ``{\tt *(*)}'' can be used for named character constants but 
with a completely different meaning: named constants are not 
subject to this restriction. 

\subsection{Arrays as Arguments} 

If the dummy argument of a procedure is an array then the actual 
argument can be either: 
\begin{itemize} 
\item  an array name (without subscripts) 
\item  an array element. 
\end{itemize} 

The first form transfers the entire array; the second form, which 
just transfers a section starting at the specified element, is 
described in more detail further on.  

The simplest, and most common, requirement is to make the entire 
contents of an array available in a procedure.  If the actual 
argument arrays are always going to be the same size then the 
dummy arrays in the procedure can use fixed bounds.  For 
example: 
\begin{verbatim}
       SUBROUTINE DOT(X, Y, Z) 
*Computes the dot product of arrays X and Y of 100 elements 
* producing array Z of the same size. 
       REAL X(100), Y(100), Z(100) 
       DO 15, I = 1,100 
          Z(I) = X(I) * Y(I) 
15     CONTINUE 
       END 
\end{verbatim} 
This procedure could be used within a program unit like this: 
\begin{verbatim}
       PROGRAM PROD 
       REAL A(100), B(100), C(100) 
       READ(UNIT=*,FMT=*)A,B 
       CALL DOT(A, B, C) 
       WRITE(UNIT=*,FMT=*)C 
       END 
\end{verbatim} 
This is perfectly legitimate, if inflexible, since it will not work on 
arrays of any other size.   

\subsubsection*{Adjustable Arrays} 

A more satisfactory solution is to generalise the procedure so that 
it can be used on arrays of any size.  This is done by using an 
adjustable arrays declaration.  Here the operands in each 
dimension bound expression may include integer variables which 
are also arguments of the procedure (or members of a common 
block).  The following example shows how this may be done:  
\begin{verbatim}
       SUBROUTINE DOTPRO(NPTS, X, Y, Z) 
       REAL X(NPTS), Y(NPTS), Z(NPTS) 
       DO 15, I = 1,NPTS 
* etc. 
\end{verbatim} 
In this case the calling sequence would be something like:\\ 
\verb?      CALL DOTPRO(100, A, B, C)?\\
An adjustable array declaration is permitted only for arrays which are
dummy arguments, since the actual array space has in this case already
been allocated in the calling unit or at some higher level. The method
can be extended in the obvious way to cover multi-dimensional arrays and
those with upper and lower bounds, for example:
\begin{verbatim}
       SUBROUTINE MULTI(MAP, K1, L1, K2, L2, TRACE) 
       DOUBLE PRECISION MAP(K1:L1, K2:L2) 
       REAL TRACE(L1-K1+1) 
\end{verbatim} 
The adjustable array mechanism can, of course, be used for arrays 
of any data type; an adjustable array can also be passed as an actual 
argument of a procedure with, if necessary, the array bounds 
passed on in parallel. 

Each array bound of a dummy argument array may be an integer 
expression involving not only constants but also integer variables 
passed in to the procedure either as arguments or by means of a 
common block.  The extent of each dimension of the array must 
not be less than one and must not be greater than the extent of the 
corresponding dimension of the actual argument array. 

If any integer variable (or named constant) used in an array-bound 
expression has a name which does not imply integer type then the 
{\tt INTEGER} statement which specifies its type must precede its use 
in a dimension-bound expression.  

\subsubsection*{Assumed-size Arrays} 

There may be circumstances in which it is impracticable to use 
either fixed or adjustable array declarations in a procedure because 
the actual size of the array is unknown when the procedure starts 
executing.  In this case an assumed-size array is a viable 
alternative.  These are also only permitted for dummy argument 
arrays of procedures, but here the array is, effectively, declared to 
be of unknown or indefinite size.  For example: 
\begin{verbatim}
       REAL FUNCTION ADDTWO(TABLE, ANGLE)  
       REAL TABLE(*) 
       N = MAX(1, NINT(SIN(ANGLE) * 500.0)) 
       ADDTWO = TABLE(N) + TABLE(N+1) 
       END 
\end{verbatim}
Here the procedure only knows that array TABLE is one-dimensional with a
lower-bound of one: that is all it needs to know to access the
appropriate elements N and N+1.  In executing the procedure it is our
responsibility to ensure that the value of ANGLE will never result in an
array subscript which is out of range.  This is always a danger with
assumed-size arrays.  Because the compiler does not have any information
about the upper-bound of an assumed-size array it cannot use any
array-bound checking code even if it is normally able to do this.  An
assumed-size array can only have the upper-bound of its last dimension
specified by an asterisk, all the other bounds (if any) must conform to
the normal rules (or be adjustable using integer arguments).

An assumed size dummy argument array is specified with an 
asterisk as the upper bound of its last (or only) dimension.  All the 
other dimension bounds, if any, must conform to normal rules for 
local arrays or adjustable arrays.   

There is one important restriction on assumed size arrays: they cannot be 
used without subscripts in I/O statements, for example in the input list 
of a {\tt READ} statement or the output list of a {\tt WRITE} statement. 
This is because the compiler has no information about the total size of 
the array when compiling the procedure. 

\subsubsection*{Array Sections} 

The rules of Fortran require that the extent of an array (in each 
dimension if it is multi-dimensional) must be at least as large in the 
actual argument as in the dummy argument, but they do not require 
actual agreement of both lower and upper bounds. For example: 
\begin{verbatim}
       PROGRAM CONFUS 
       REAL X(-1:50), Y(10:1000) 
       READ(UNIT=*,FMT=*) X, Y 
       CALL OUTPUT(X) 
       CALL OUTPUT(Y) 
       END 

       SUBROUTINE OUTPUT(ARRAY) 
       REAL ARRAY(50) 
       WRITE(UNIT=*,FMT=*) ARRAY 
       END 
\end{verbatim} 
The effect of this program will be to output the elements X(-1) to 
X(48) since X(48) corresponds to ARRAY(50), and then output 
Y(10) to Y(59) also.  The subroutine will work similarly on a slice 
through a two-dimensional array: 
\begin{verbatim}
       PROGRAM TWODIM 
       REAL D(100,20) 
* ... 
       NSLICE = 15 
       CALL OUTPUT(D(1,NSLICE)) 
\end{verbatim} 
In this example the slice of the array from elements D(1,15) to 
D(50,15) will be written to the output file.  In order to work out 
what is going to happen you need to know that Fortran arrays are 
stored with the first subscript most rapidly varying, and that the 
argument association operates as if the address of the specified 
element were transferred to the base address of the dummy 
argument array.   

The use of an array element as an actual argument when the 
dummy argument is a complete array is a very misleading notation 
and the transfer of array sections should be avoided if at all 
possible. 

\subsubsection*{Character Arrays } 

When a dummy argument is a character array the passed-length 
mechanism can be used in the same way as for a character variable.  
Every element of the dummy array has the length that was passed 
in from the actual argument. 

For example, a subroutine designed to sort an array of character 
strings into ascending order might start with specification 
statements like these: 
\begin{verbatim}
       SUBROUTINE SORT(NELS, NAMES) 
       INTEGER NELS 
       CHARACTER NAMES(NELS)*(*) 
\end{verbatim} 
Alternatively the actual argument can be a character variable or 
substring.  In such cases it usually makes more sense not to use the 
passed-length mechanism.  For example an actual argument 
declared:\\ 
\verb?      CHARACTER*80 LINE?\\ 
could be passed to a subroutine which declared it as an array of 
four 20-character elements: 
\begin{verbatim}
       SUBROUTINE SPLIT(LINE) 
       CHARACTER LINE(4)*20 
\end{verbatim} 

Although this is valid Fortran, it is not a very satisfactory 
programming technique to use a procedure call to alter the shape 
of an item so radically. 
       
\subsection{Procedures as Arguments} 

Fortran allows one procedure to be used as the actual argument of 
another procedure.  This provides a powerful facility, though one 
that most programmers use only rarely.  Procedures are normally 
used to carry out a given set of operations on different sets of data; 
but sometimes you want to carry out the same set of operations on 
different functional forms.  Examples include: finding the gradient 
of a function, integrating the area under a curve, or simply plotting 
a graph.  If the curve is specified as a set of data points then you 
can simply pass over an array, but if it is specified by means of 
some algorithm then the procedure which evaluates it can itself be 
an actual argument. 

In the next example, the subroutine GRAPH plots a graph of a 
function MYFUNC between specified limits, with its argument 
range divided somewhat arbitrarily into 101 points.  For simplicity 
it assumes the existence of a subroutine PLOT which moves the 
pen to position (X,Y).  Some other subroutines would, in practice, 
almost certainly be required. 
\begin{verbatim}
      SUBROUTINE GRAPH(MYFUNC, XMIN, XMAX) 
*Plots functional form of MYFUNC(X) with X in range XMIN:XMAX. 
      REAL MYFUNC, XMIN, XMAX 
      XDELTA = (XMAX - XMIN) / 100.0 
      DO 25, I = 0,100 
          X = XMIN + I * XDELTA 
          Y = MYFUNC(X) 
          CALL PLOT(X, Y) 
25    CONTINUE 
      END 
\end{verbatim} 
The procedure {\tt GRAPH} can then be used to plot a function simply 
by providing its name them as the first argument of the call.  The 
only other requirement is that the name of each function used as an 
actual argument in this way must be specified in an {\tt INTRINSIC} or 
{\tt EXTERNAL} statement, as appropriate.  Thus: 
\begin{verbatim}
       PROGRAM CURVES 
       INTRINSIC SIN, TAN 
       EXTERNAL MESSY  
       CALL GRAPH(SIN, 0.0, 3.14159) 
       CALL GRAPH(TAN, 0.0, 0.5) 
       CALL GRAPH(MESSY, 0.1, 0.9) 
       END 

       REAL FUNCTION MESSY(X) 
       MESSY = COS(0.1*X) + 0.02 * SIN(SQRT(X))  
       END 
\end{verbatim} 
This will first plot a graph of the sine function, then of the tangent 
function with a different range, and finally produce another plot of 
the external function called {\tt MESSY}.  These functions must, of 
course, have the same procedure interface themselves and must be 
called correctly in the {\tt GRAPH} procedure.   

It is possible to pass either a function or a subroutine as an actual 
argument in this way: the only difference is that a {\tt CALL} statement 
is used instead of a function reference to execute the dummy 
procedure.  It is possible to pass a procedure through more than 
one level of procedure call in the same way.  Continuing the last 
example, another level could be introduced like this: 
\begin{verbatim}
       PROGRAM CURVE2 
       EXTERNAL MESSY 
       INTRINSIC SIN, TAN 
       CALL GRAPH2(PRETTY) 
       CALL GRAPH2(TAN) 
       END 

       SUBROUTINE GRAPH2(PROC) 
       EXTERNAL PROC 
       CALL GRAPH(PROC, 0.1, 0.7) 
       END 
\end{verbatim} 
Thus the procedure {\tt GRAPH2} sets limits to each plot and passes 
the procedure name on to {\tt GRAPH}.  The symbolic name {\tt PROC} 
must be declared in an {\tt EXTERNAL} statement as it is a dummy 
procedure: an {\tt EXTERNAL} statement is required whether the 
actual procedure at the top level is intrinsic or external.  The syntax 
of the {\tt INTRINSIC} and {\tt EXTERNAL} statements is given in section 
9.12 below. 

The name of an intrinsic function used as an actual argument must 
be a specific name and not a generic one.  This is the only 
circumstance in which you still have to use specific names for 
intrinsic functions.  A full list of specific names is given in 
the appendix. A few of the most basic intrinsic functions which are 
often expanded to in-line code (those for type conversion, lexical 
comparison, as well as MIN and MAX) cannot be passed as actual 
arguments.   

\subsection{Subroutine and Call Statements} 

It is convenient to describe these two statements together as they 
have to be closely matched in use.  The general form of the 
{\tt SUBROUTINE} statement is:\\ 
\verb+      SUBROUTINE+ {\em name} {\tt (} {\em dummy1,} {\em dummy2,}
...  
{\em dummyN} {\tt )}\\ 
or\\ 
\verb+      SUBROUTINE+ {\em name}\\ 
The second form just indicates that if there are no arguments then 
the parentheses are optional. 

The symbolic name of the subroutine becomes a global name; it 
must not be used at all within the program unit and must not be 
used for any other global item within the entire executable 
program. 

The dummy arguments are also simply symbolic names.  The way 
in which these are interpreted is covered in the next section. 

The {\tt CALL} statement has similar general forms:\\ 
\verb+      CALL+ {\em name} {\tt (} {\em arg1,} {\em arg2,} ... {\em 
argN} {\tt )} \\ 
or\\ 
\verb+      CALL+ {\em name}\\ 
Again, if there are no arguments the parentheses are optional. 

The name must be that of a subroutine (or dummy subroutine).  
Each arg is an actual argument which can be a variable, array, 
substring, array element or any form of expression.  The permitted 
forms, which depend on the form of the corresponding dummy 
argument and how it is used within the subroutine, are fully 
described in the preceding sections.  

\subsection{\texttt{RETURN} Statement} 

The {\tt RETURN} statement just consists of the keyword 

{\tt RETURN} 

Its effect is to stop the procedure executing and to return control, 
and where appropriate argument and function values, to the calling 
program unit.  The execution of the {\tt END} statement at the end of 
the program unit has the exactly the same effect, so that {\tt RETURN} 
is superfluous in procedures which have only one entry and one 
exit point (as all well-designed procedures should).  It is, however, 
sometimes convenient to use {\tt RETURN} for an emergency exit.  
Here is a somewhat simple-minded example just to illustrate the 
point: 
\begin{verbatim}
     REAL FUNCTION HYPOT(X, Y) 
*Computes the hypotenuse of a right-angled triangle. 
      REAL X, Y 
      IF(X .LE. 0.0 .OR. Y .LE. 0.0) THEN 
          WRITE(UNIT=*,FMT=*)'Warning: impossible values' 
          HYPOT = 0.0 
          RETURN 
      END IF 
      HYPOT = SQRT(X**2 + Y**2) 
      END 
\end{verbatim} 
This function could be used in another program unit like this:  
\begin{verbatim}
       X = HYPOT(12.0, 5.0) 
       Y = HYPOT(0.0, 5.0) 
\end{verbatim} 
which would assign to X the value of 13.0000 approximately, 
while the second function call would cause a warning message to 
be issued and would return a value of zero to Y. 

In the external function shown above it would have been perfectly 
possible to avoid having two exits points by an alternative ending 
to the procedure, such as: 
\begin{verbatim}
      IF(X .LE. 0.0 .OR. Y .LE. 0.0) THEN 
          WRITE(UNIT=*,FMT=*)'Warning: impossible values' 
          HYPOT = 0.0 
      ELSE  
          HYPOT = SQRT(X**2 + Y**2) 
      END IF 
      END 
\end{verbatim} 
In more realistic cases, however, the main part of the calculation 
would be much longer than just one statement and it might then be 
easier to understand the working if a {\tt RETURN} statement were to 
be used than with almost all of the procedure contained within an 
ELSE-block.  A third possibility for emergency exits is to use an 
unconditional {\tt GO TO} statement to jump to a label placed on the 
{\tt END} statement. 

\subsection{\texttt{FUNCTION} Statement} 

The {\tt FUNCTION} statement must be the first statement of every 
external function.  Its general form is:\\ 
\verb+      + {\em type } {\tt FUNCTION(} {\em dummy1,} {\em dummy2,} ... 
{\em dummyN} {\tt )} \\ 
The {\em type} specification is optional: if it is omitted then the type
of 
the result is determined by the usual rules.  The function name may 
have its type specified by a type or {\tt IMPLICIT} statement which 
appears later in the program unit.  If the function is of type 
character then the length may be specified by a literal constant (but 
not a named constant) or may be given in the form 
{\tt CHARACTER*(*)} in which case the length will be passed in as 
the length declared for the function name in the calling program 
unit.   

There may be any number of dummy arguments including none, 
but the parentheses must still be present.  Dummy arguments may, 
as described in section 9.4, be variables, arrays, or procedures.     

The function name may be used as a variable within the function 
subprogram unit; a value must be assigned to this variable before 
the procedure returns control to the calling unit.  If the function 
name used the passed-length option then the corresponding 
variable cannot be used as an operand of the concatenation 
operator except in an assignment statement.  The passed-length 
option is less useful for character functions than for arguments 
because the length is inevitably the same for all references from the 
same program unit.  For example: 
\begin{verbatim}
       PROGRAM FLEX 
       CHARACTER CODE*8, CLASS*6, TITLE*16  
       CLASS = CODE('SECRET') 
       TITLE = CODE('ORDER OF BATTLE') 
       END 

       CHARACTER*(*) FUNCTION CODE(WORD) 
       CHARACTER WORD*(*), BUFFER*80 
       DO 15, K = 1,LEN(WORD) 
            BUFFER(K:K) = CHAR(ICHAR(WORD(K:K) + 1) 
15     CONTINUE 
       CODE = BUFFER 
       END 
\end{verbatim} 
Unfortunately, although this function can take in an argument of 
any length up to 80 characters long and encode it, it can only 
return a result of exactly 8 characters long when called from the 
program FLEX, so that it will not produce the desired result when 
provided with the longer character string.  This limitation could be 
overcome with the use of a subroutine with a second passed-length 
argument to handle the returned value. 

Functions without arguments do not have a wide range of uses but 
applications for them do occur up from time to time, for example 
when generating random numbers or reading values from an input 
file.  For example: 
\begin{verbatim}
       PROGRAM COPY 
       REAL NEXT 
       DO 10,I = 1,100 
             WRITE(UNIT=*,FMT=*) NEXT() 
10     CONTINUE 
       END 

       REAL FUNCTION NEXT()  
       READ(UNIT=*,FMT=*) NEXT  
       END 
\end{verbatim} 
The parentheses are needed on the function call to distinguish it 
from a variable.  The function statement itself also has to have the 
empty pair of parentheses, presumably to match the call. 

\subsection{\texttt{SAVE} Statement} 

{\tt SAVE} is a specification statement which can be used to ensure that 
variables and arrays used within a procedure preserve their values 
between successive calls to the procedure.  Under normal 
circumstances local items will become ``undefined'' as soon as the 
procedure returns control to the calling unit.  It is often useful to 
store the values of certain items used on one call to avoid doing 
extra work on the next.  For example: 
\begin{verbatim}
       SUBROUTINE EXTRA(MILES) 
       INTEGER MILES, LAST 
       SAVE LAST 
       DATA LAST /0/ 
       WRITE(UNIT=*, FMT=*) MILES - LAST, ' more miles.' 
       LAST = MILES 
       END 
\end{verbatim} 
This subroutine simply saves the value of the argument {\tt MILES} 
each time and subtracts it from the next one, so that it can print out 
the incremental value.  The value of {\tt LAST} had to be given an 
initial value using a {\tt DATA} statement in order to prevent its use 
with an undefined value on the initial call. 

Local variables and arrays and complete named common 
blocks can be saved using {\tt SAVE} statements, but not variables and 
arrays which are dummy arguments or which appear within 
common blocks.   

Items which are initially defined with a {\tt DATA} statement but which 
are never updated with a new value do not need to be saved. 

The {\tt SAVE} statement has two alternative forms:\\ 
\verb?      SAVE? {\em item, item, ... item}\\ 
\verb?      SAVE?\\ 
Where each {\em item} can be a local variable or (unsubscripted) array, 
or the name of a common block enclosed in slashes.  The second 
form, with no list of items, saves all the allowable items in the 
program unit.  This form should not be used in any program unit 
which uses a common block unless all common blocks used in that 
program unit are also used in the main program or saved in every 
program unit in which it appears.  The {\tt SAVE} statement can be 
used in the main program unit (so that it could be packaged with 
other specifications in an {\tt INCLUDE} file) but has no effect.   

Many current Fortran systems have a simple static storage 
allocation scheme in which all variables are saved whether {\tt SAVE}  
is used or not.  But on small computers which make use of disc 
overlays, or large ones with virtual memory systems, this may not 
be so.  You should always take care to use the {\tt SAVE} statement 
anywhere that its use is indicated to make your programs safe and 
portable.  Even where it is at present strictly redundant it still 
indicates to the reader that the procedure works by retaining 
information from one call to the next, and this is valuable in itself. 

\subsection{\texttt{EXTERNAL} and \texttt{INTRINSIC} Statements} 

The {\tt EXTERNAL} statement is used to name external procedures 
which are required in order to run a given program unit.  It may 
specify the name of any external function or subroutine.  It is 
required in three rather different circumstances: 
\begin{itemize} 

\item  whenever an external procedure or dummy procedure is 
used as the actual argument of another procedure call; 
\item  to call any procedure which has a name duplicating an 
intrinsic function; 
\item  to ensure that a named block data subprogram is linked 
into the complete executable program.  This specialised 
use is covered further in section 12.4. 
\end{itemize} 

 
The {\tt INTRINSIC} statement is used to declare a name to be that of 
an intrinsic function.  It is normally necessary only when that 
function is to be used as the actual argument of another procedure 
call, but may also be advisable when calling a non-standard 
intrinsic function to remove any ambiguity which might arise if an 
external function of the same name also existed. 

The general form of the two statements is the same:\\ 
\verb+      EXTERNAL+  {\em ename,} {\em ename,} ... {\em ename}\\ 
\verb+      INTRINSIC+ {\em iname,} {\em iname,} ... {\em iname}\\ 
Where {\em ename} can be the name of an external function or 
subroutine or a dummy procedure; {\em iname} must be specific name of 
an intrinsic function.  For example, to use the real and double 
precision versions of the trigonometric functions as actual 
arguments we need:\\ 
\verb?      INTRINSIC SIN, COS, TAN, DCOS, DSIN, DTAN?\\ 
When the function name SIN is used as an actual argument it refers 
to the specific real sine function; in other contexts it still has its 
usual generic property. 
The use of procedures as actual arguments is covered in detail in 
section 9.7; a list of specific names of intrinsic functions is given 
in the appendix. 


\section{Input/Output Facilities} 

The I/O system of Fortran is relatively powerful, flexible, and
well-defined.  Programs can be portable and device-independent even if
they make extensive use of input/output facilities: this is difficult if
not impossible in many other high-level languages.  The effects of the
hardware and operating system cannot, of course, be ignored entirely but
they usually only affect fairly minor matters such as the forms of
file-name and the maximum record length that can be used.

The {\tt READ} and {\tt WRITE} statements are most common and 
generally look like this: 
\begin{verbatim}
       READ(UNIT=*, FMT=*) NUMBER 
       WRITE(UNIT=13, ERR=999) NUMBER, ARRAY(1), ARRAY(N) 
\end{verbatim} 
The pair of parentheses after the word {\tt READ} or {\tt WRITE} encloses 
the control-list: a list of items which specifies where and how the 
data transfer takes place.  The items in this list are usually specified 
with keywords.  The list of data items to be read or written follow 
the control-list.   

Other input/output statements have a similar form except that they 
only have a control-list.  There are the file-handling statements 
{\tt OPEN,} {\tt CLOSE,} and {\tt INQUIRE,} as well as the {\tt REWIND}
and 
{\tt BACKSPACE} statements which alter the currently active position 
within a file. 

Before covering these statements in detail, it is necessary to 
explain some of the concepts and terminology involved. 

\subsection{Files, I/O Units, and Records} 

In Fortran the term file is used for anything that can be handled 
with a {\tt READ} or {\tt WRITE} statement: the term covers not just data 
files stored on disc or tape and also peripheral devices such as 
printers or terminals.  Strictly these should all be called external 
files, to distinguish them from internal files.   

An internal file is nothing more than a character variable or array 
which is used as a temporary file while the program is running.  
Internal files can be used with {\tt READ} and {\tt WRITE} statements in 
order to process character information under the control of a 
format specification.  They cannot be used by other I/O statements.  

Before an external file can be used it must be connected to an I/O 
unit.  I/O units are integers which may be chosen freely from zero 
up to a system-dependent limit (usually at least 99).  Except in 
{\tt OPEN} and {\tt INQUIRE} statements, files can only be referred to via 
their unit numbers.  

The {\tt OPEN} statement connects a named file to a numbered unit.  It 
usually specifies whether the file already exists or whether a new 
one is to be created, for example:  
\begin{verbatim}
      OPEN(UNIT=1, FILE='B:INPUT.DAT', STATUS='OLD') 
      OPEN(UNIT=9, FILE='PRINTOUT', STATUS='NEW') 
\end{verbatim} 
For simplicity most of the examples in this section show an actual 
integer as the unit identifier, but it helps to make software more 
modular and adaptable if a named constant or a variable is used 
instead.  

I/O units are a global resource.  A file can be opened in any 
program unit; once it is open I/O operations can be performed on 
it in any program unit provided that the same unit number is used.  
The unit number can be held in an integer variable and passed to 
the procedure as an argument.  

The connection between a file and a unit, once established, 
persists until: 
\begin{itemize} 
\item   the program terminates normally (at a {\tt STOP} 
statement or the {\tt END} of the main program); 
\item   another {\tt OPEN} statement connects a different 
file to the same unit; 
\item   or a {\tt CLOSE} statement is executed on that 
unit. 
\end{itemize} 

 
Although all files are closed when the program exits, it is good 
practice to close them explicitly as soon as I/O operations on them 
are completed.  If the program terminates abnormally, for example 
because an error occurs or it is aborted by the user, any files which 
are open, especially output files, may be left with incomplete or 
corrupted records. 

The {\tt INQUIRE} statement can be used to obtain information about 
the current properties of external files and I/O units.  {\tt INQUIRE} is 
particularly useful when writing library procedures which may 
have to run in a variety of different program environments.  You 
can find out, for example, which unit numbers are free for use or 
whether a particular file exists and if so what its characteristics are.  

\subsubsection*{Records} 

A file consists of a sequence of records.  In a text file a record 
corresponds to a line of text; in other cases a record has no 
physical basis, it is just a convenient collection of values chosen to 
suit the application.  There is no need for a record to correspond to 
a disc sector or a tape block.  {\tt READ} and {\tt WRITE} statements 
always start work at the beginning of a record and always transfer 
a whole number of records.  

The rules of Fortran set no upper limit to the length of a record but, 
in practice, each operating system may do so.  This may be 
different for different forms of record.  

\subsubsection*{Formatted and Unformatted Records} 

External files come in two varieties according to whether their 
records are formatted or unformatted.  Formatted records store data 
in character-coded form, i.e.\ as lines of text.  This makes them 
suitable for a wide range of applications since, depending on their 
contents, they may be legible to humans as well as computers.  The 
main complication for the programmer is that each {\tt WRITE} or 
{\tt READ} statement must specify how each value is to be converted 
from internal to external form or vice-versa.  This is usually done 
with a format specification. 

Unformatted records store data in the internal code of the computer 
so that no format conversions are involved.  This has several 
advantages for files of numbers, especially floating-point numbers. 
Unformatted data transfers are simpler to program, faster in 
execution, and free from rounding errors.  Furthermore the 
resulting data files, sometimes called binary files, are usually much 
smaller.  A real number would, for example, have to be turned into 
a string of 10 or even 15 characters to preserve its precision on a 
formatted record, but on an unformatted record a real number 
typically occupies only 4 bytes i.e.\ the same as 4 characters.  The 
drawback is that unformatted files are highly system-specific.  
They are usually illegible to humans and to other brands of 
computer and sometimes incompatible with files produced by other 
programming languages on the same machine.  Unformatted files 
should only be used for information to be written and read by 
Fortran programs running on the same type of computer.  

\subsubsection*{Sequential and Direct Access} 

All peripheral devices allow files to be processed sequentially: you 
start at the beginning of the file and work through each record in 
turn.  One important advantage of sequential files is that different 
records can have different lengths; the minimum record length is 
zero but the maximum is system-dependent.   

Sequential files behave as if there were a pointer attached to the 
file which always indicates the next record to be transferred.  On 
devices such as terminals and printers you can only read or write 
in strict sequential order, but when a file is stored on disc or tape 
it is possible to use the {\tt REWIND} statement to reset this pointer to 
the start of the file, allowing it to be read in again or re-written.  
On suitable files the {\tt BACKSPACE} statement can be used to move 
the pointer back by one record so that the last record can be read 
again or over-written.   
  
One unfortunate omission from the Fortran Standard is that the 
position of the record pointer is not defined when an existing 
sequential file is opened.  Most Fortran systems behave sensibly 
and make sure that they start at the beginning of the file, but there 
are a few rogue systems around which make it advisable, in 
portable software, to use {\tt REWIND} after the {\tt OPEN} statement.  
Another problem is how to append new records to an existing 
sequential file.  Some systems provide (as an extension) an 
``append'' option in the {\tt OPEN} statement, but the best method using 
Standard Fortran is to open the file and read records one at a time 
until the end-of-file condition is encountered; then use 
{\tt BACKSPACE} to move the pointer back and clear the end-of-file 
condition.  New records can then be added in the usual way.   

The alternative access method is direct-access which allows records to be
read and written in any order.  Most systems only permit this for files
stored on random-access devices such as discs; it is sometimes also
permitted on tapes.  All records in a direct-access file must be the same
length so that the system can compute the location of a record from its
record number.  The record length has to be chosen when the file is
created and (on most systems) is then fixed for the life of the file.  In
Fortran, direct-access records are numbered from one upwards; each {\tt
READ} or {\tt WRITE} statement specifies the record number at which the
transfer starts.

Records may be written to a direct-access file in any order. Any 
record can be read provided that it exists, i.e.\ it has been written at 
some time since the file was created.  Once a record has been 
written there is no way of deleting it, but its contents can be 
updated, i.e.\ replaced, at any time. 

A few primitive operating systems require the maximum length of 
a direct-access file to be specified when the file is created; this is 
not necessary in systems which comply fully with the Fortran 
Standard.       

\subsection{External Files} 

Formatted and unformatted records cannot be mixed on the same file and on
most systems files designed for sequential-access are quite distinct from
those created for direct-access: thus there are four different types of
external file.  There is no special support in Standard Fortran for any
other types of file such as indexed-sequential files or name-list files.

\subsubsection*{Formatted Sequential Files} 

These are often just called text files.  Terminals and printers should
always be treated as formatted sequential files.  Data files of this type
can be created in a variety of ways, for example by direct entry from the
keyboard, or by using a text editor.  Some Fortran systems do not allow
records to be longer than a normal line of text, for example 132
characters.  Unless a text file is pre-connected it must be opened with
an {\tt OPEN} statement, but the {\tt FORM=} and {\tt ACCESS=} keywords
are not needed as the default values are suitable:\\
\verb?      OPEN(UNIT=4, FILE='REPORT', STATUS='NEW')?\\
All data transfers must be carried out under format control.  There 
are two options with files of this type: you can either provide your 
own format specification or use list-directed formatting.   

The attraction of list-directed I/O is that the Fortran system does 
the work, providing simple data transfers with little programming 
effort.  They are specified by having an asterisk as the format 
identifier: 
\begin{verbatim}
      WRITE(UNIT=*, FMT=*)'Enter velocity: ' 
      READ(UNIT=*, FMT=*, END=999) SPEED 
\end{verbatim} 
List-directed input is quite convenient when reading numbers from 
a terminal since it allows virtually ``free-format'' data entry.  It may 
also be useful when reading data files where the layout is not 
regular enough to be handled by a format specification.  List-directed output is satisfactory when used just to output a character 
string (as in the example above), but it produces less pleasing 
results when used to output numerical values since you have no 
control over the positioning of items on the line, the field-width, 
or the number of decimal digits displayed.  Thus:  
\begin{verbatim}
      WRITE(UNIT=LP, FMT=*)' Box of',N,' costs �',PRICE 
\end{verbatim} 
will produce a record something like this:\\ 
\verb? Box of          12 costs �          9.5000000?\\ 
List-directed output is best avoided except to write simple 
messages and for diagnostic output during program development. 
The rules for list-directed formatting are covered in detail in 
section 10.10. 

The alternative is to provide a format specification: this provides 
complete control over the data transfer.  The previous example can 
be modified to use a format specification like this:  
\begin{verbatim}
      WRITE(UNIT=LP, FMT=55)'Box of',N,' costs �',PRICE 
55    FORMAT(1X, A, I3, A, F6.2) 
\end{verbatim} 
and will produce a record like this:\\ 
\verb? Box of 12 costs �  9.5?0\\ 
The format specification is provided in this case by a {\tt FORMAT} 
statement: its label is the format identifier in the {\tt WRITE}
statement.  
Other ways of providing format specifications are described in 
section 10.6.  

One unusual feature of input under control of a format specification is 
that each line of text will appear to be padded out on the right with an 
indefinite number of blanks irrespective of the actual length of the data 
record.  This means that, among other things, it is not possible to 
distinguish between an empty record and one filled with blanks.  If 
numbers are read from an empty record they will simply be zero. 

\subsubsection*{Unformatted Sequential Files} 

Unformatted sequential files are often used as to transfer data from 
one program to another.  They are also suitable for scratch files, 
i.e.\ those used temporarily during program execution.  The only 
limit on the length of unformatted records is that set by the 
operating system; most systems allow records to contain a few 
thousand data items at least.  The {\tt OPEN} statement must specify 
the file format, but the default access method is ``sequential''.  Each 
{\tt READ} or {\tt WRITE} statement transfers one unformatted record.   

For example, these statements open an existing unformatted file 
and read two records from it: 
\begin{verbatim}
      OPEN(UNIT=15, FILE='BIN', STATUS='OLD', FORM='UNFORMATTED') 
      READ(15) HEIGHT, LENGTH, WIDTH 
      READ(15) ARRAYP, ARRAYQ 
\end{verbatim} 
{\tt BACKSPACE} and {\tt REWIND} statements may generally be used on 
all unformatted sequential files. 

\subsubsection*{Unformatted Direct-access Files} 

Since direct-access files are readable only by machine, it seems 
sensible to use unformatted records to maximise efficiency.  The 
{\tt OPEN} statement must specify {\tt ACCESS='DIRECT'} and also 
specify the record length.  Unfortunately the units used to measure 
the length of a record are not standardised: some systems measure 
them in bytes, others in numerical storage units, i.e.\ the number of 
real or integer variables a record can hold (see section 5.1).  This 
is a minor obstacle to portability and means that you may need to 
know how many bytes your machine uses for each numerical 
storage unit, although this is just about the only place in Fortran 
where this is necessary.  Most systems will allow you to open an 
existing file only if the record length is the same as that used when 
the file was created.   
  
Each {\tt READ} and {\tt WRITE} statement transfers exactly one record 
and must specify the number of that record: an integer value from 
one upwards.  The record length must not be greater than that 
declared in the {\tt OPEN} statement; if an output record is not 
completely filled the remainder is undefined. 

To illustrate how direct-access files can be used, here is a 
complete program which allows a very simple data-base, such as 
a set of stock records, to be examined.  Assuming that the record 
length is measured in numerical storage units of 4 bytes, the 
required record length in this case can be computed as follows: 

\begin{center} 
\begin{tabular}{lll} 
NAME  & 1 CHARACTER*10 variable  & 10 chars = 10 bytes.\\ 
STOCK & 1 INTEGER variable       & 1 unit =   4 bytes\\ 
PRICE & 1 REAL variable          & 1 unit =   4 bytes\\ 
\end{tabular} 
\end{center} 

The total record length is 18 bytes or 5 numerical storage units 
(rounding up to the next integer). 
\begin{verbatim}
      PROGRAM DBASE1 
      INTEGER STOCK, NERR 
      REAL PRICE 
      CHARACTER NAME*10 
*Assume record length in storage units holding 4 chars each.        
      OPEN(UNIT=1, FILE='STOCKS', STATUS='OLD', 
     $ ACCESS='DIRECT', RECL=5) 
100   CONTINUE 
*Ask user for part number which will be used as record number. 
      WRITE(UNIT=*,FMT=*)'Enter part number (or zero to quit):' 
      READ(UNIT=*,FMT=*) NPART 
      IF(NPART .LE. 0) STOP 
      READ(UNIT=1, REC=NPART, IOSTAT=NERR) NAME, STOCK, PRICE 
      IF(NERR .NE. 0) THEN 
          WRITE(UNIT=*,FMT=*)'Unknown part number, re-enter' 
          GO TO 100 
      END IF 
      WRITE(*,115)STOCK, NAME, PRICE 
115   FORMAT(1X,'Stock is',I4, 1X, A,' at �', F8.2, ' each')  
      GO TO 100 
      END 
\end{verbatim} 
The typical output record of the program will be of the form:\\ 
\verb?      Stock is 123 widgets    at �  556.89 each?\\ 
This program could be extended fairly easily to allow the contents 
of the record to be updated as the stock changes. 

\subsubsection*{Formatted Direct-access Files} 

Formatted direct-access files are slightly more portable than the
unformatted form because their record length is always measured in
characters.  Otherwise there is little to be said for them.  The {\tt
OPEN} statement must specify both {\tt ACCESS='DIRECT'} and {\tt
FORM='FORMATTED'} and each {\tt READ} and {\tt WRITE} statement must
contain both format and record-number identifiers.  List-directed
transfers are not permitted.  If the format specification requires more
than one record to be used, these additional records follow on
sequentially from that specified by REC=.  It is an error to try to read
beyond the end of a record, but an incompletely filled record will be
padded out with blanks.

\subsection{Internal Files} 

An internal file is an area of central memory which can be used as 
if it were a formatted sequential file.  It exists, of course, only 
while the program is executing.  Internal files are used for a variety 
of purposes, particularly to carry out data conversions to and from 
character data type.  Some earlier versions of Fortran included 
{\tt ENCODE} and {\tt DECODE} statements: the internal file {\tt READ} 
(which replaces {\tt DECODE}) and internal file {\tt WRITE} (which 
replaces {\tt ENCODE}) are simpler, more flexible, and entirely 
portable. 

An internal file can only be used with {\tt READ} and {\tt WRITE} 
statements and an explicit format specification is required: 
list-directed transfers are not permitted.  The unit must have character 
data type but it can be a variable, array element, substring, or a 
complete array.  If it is a complete array then each array element 
constitutes a record; in all other cases the file only consists of one 
record.  Data transfers always start at the beginning of the internal 
file, that is an implicit rewind is performed each time.  The record 
length is the length of the character item.  It is illegal to try to 
transfer more characters than the internal file contains, but if a 
record of too few characters is written it will be padded out with 
blanks.  The {\tt END=} and {\tt IOSTAT=} mechanisms can be used to 
detect the end-of-file. 

An internal file {\tt WRITE} is typically used to convert a numerical 
value to a character string by using a suitable format specification, 
for example: 
\begin{verbatim}
      CHARACTER*8 CVAL 
      RVALUE = 98.6 
      WRITE(CVAL, '(SP, F7.2)') RVALUE 
\end{verbatim} 
The {\tt WRITE} statement will fill the character variable CVAL with 
the characters {\tt ' +98.60 '} (note that there is one blank at each end
of 
the number, the first because the number is right-justified in the 
field of 7 characters, the second because the record is padded out 
to the declared length of 8 characters). 

Once a number has been turned into a character-string it can be processed
further in the various ways described in section 7.  This makes it
possible, for example, to write numbers left-justified in a field, or
mark negative numbers with with ``DR'' (as in bank statements) in or even
use a pair of parentheses (as in balance-sheets).  With suitable
arithmetic you can even output values in other number-bases such as octal
or hexadecimal.  Even more elaborate conversions may be achieved by first
writing a suitable format specification into a character string and then
using that format to carry out the desired conversion.

Internal file {\tt READ} statements can be used to decode a character 
string containing a numerical value.  One obvious application is to 
handle the user input to a command-driven program.  Suppose the 
command line consists of a word followed, optionally, by a 
number (in integer or real format), with at least one blank 
separating the two.  Thus the input commands might be something 
like:  
\begin{verbatim}
UP 4 
RIGHT 123.45 
\end{verbatim} 
A simple way to deal with this is to read the whole line into a 
character variable and then use the INDEX function to locate the 
first blank.  The preceding characters constitute the command 
word, those following can be converted to a real number using an 
internal file {\tt READ}. For example:  
\begin{verbatim}
      CHARACTER CLINE*80 
* . . . 
100   WRITE(UNIT=*,FMT=*)'Enter command: ' 
      READ(*, '(A)', IOSTAT=KODE) CLINE 
      IF(KODE .NE. 0) STOP 
      K = INDEX(CLINE, ' ') 
*The command word is now in CLINE(1:K-1); Assume the  
* number is in the next 20 characters: read it into RVALUE 
      READ(CLINE(K+1:), '(BN,F20.0)', IOSTAT=KODE) 
RVALUE 
      IF(KODE .NE. 0) THEN 
          WRITE(UNIT=*,FMT=*)'Error in number: try again' 
          GO TO 100 
      END IF 
\end{verbatim} 
Note that the edit descriptor {\tt BN} is needed to ensure that any 
trailing blanks will be ignored; the {\tt F20.0} will then handle any real 
or integer constant anywhere in the next 20 characters.  A field of 
blanks will be converted into zero. 

\subsection{Pre-Connected Files} 
\subsubsection*{Terminal Input/Output} 

Many programs are interactive and need to access the user's terminal.
Although the terminal is a file which can be connected with an {\tt OPEN}
statement, its name is system-dependent.  Fortran solves the problem by
providing two special files usually called the standard input file and
the standard output file.  These files are pre-connected, i.e.\ no {\tt
OPEN} statement is needed (or permitted).  They are both formatted
sequential files and, on interactive systems, handle input and output to
the terminal.  You can {\tt READ} and {\tt WRITE} from these files simply
by having an asterisk ``{\tt *}'' as the unit identifier.  These files
make terminal I/O simple and portable; examples of their use can be found
throughout this book.

When a program is run in batch mode most systems arrange for 
standard output to be diverted to a log file or to the system printer.  
There may be some similar arrangement for the standard input file.  

The asterisk notation has one slight drawback: the unit numbers is 
often specified by an integer variable so that the source of input or 
destination of output can be switched from one file to another 
merely be altering the value of this integer.  This cannot be done 
with the standard input or output files.  

\subsubsection*{Other Pre-connected Files} 

In order to retain compatibility with Fortran66, many systems 
provide other pre-connected files.  It used to be customary to have 
unit 5 connected to the card-reader, and unit 6 to the line printer.  
Other units were usually connected to disc files with appropriate 
names: thus unit 39 might be connected to a file called 
{\tt FTN039.DAT} or even {\tt TAPE39}. These pre-connections are 
completely obsolete and should be ignored: an {\tt OPEN} statement 
can supersede a pre-connection on any numbered unit.  
Unfortunately these obsolete pre-connections can have unexpected 
side effects.  If you forget to open an output file you may find that 
your program will run without error but that the results will be 
hidden on a file with one of these special names. 

\subsection{Error and End-Of-File Conditions} 

Errors in most executable statements can be prevented by taking 
sufficient care in writing the program, but in I/O statements errors 
can be caused by events beyond the control of the programmer: for 
example through trying to open a file which no longer exists, 
writing to a disc which is full, or reading a data file which has been 
created with the wrong format.  Since I/O statements are so 
vulnerable, Fortran provides an error-handling mechanism for 
them.  There are actually two different ways of handling errors 
which may be used independently or in combination.  

Firstly, you can include in the I/O control list an item of the form:\\ 
\verb?      IOSTAT=integer-variable?\\ 
When the statement has executed the integer variable (or array 
element) will be assigned a value representing the I/O status.  If the 
statement has completed successfully this variable is set to zero, 
otherwise it is set to some other value, a positive number if an 
error has occurred, or a negative value if the end of an input file 
was detected.  Since the value of this status code is system-dependent, 
in portable software the most you can do is to compare 
it to zero and, possibly, report the actual error code to the user.  
Thus: 
\begin{verbatim}
100   WRITE(UNIT=*, FMT=*)'Enter name of input file: ' 
      READ(UNIT=*, FMT=*) FNAME 
      OPEN(UNIT=INPUT, FILE=FNAME, STATUS='OLD', IOSTAT=KODE) 
      IF(KODE .NE. 0) THEN 
          WRITE(UNIT=*,FMT=*)FNAME, ' cannot be opened' 
          GO TO 100 
      END IF 
\end{verbatim} 
This simple error-handling scheme makes the program just a little 
more user-friendly: if the file cannot be opened, perhaps because 
it does not exist, the program asks for another file-name. 

The second method is to include an item of the form\\ 
\verb?      ERR=label?\\ 
which causes control to be transferred to the statement attached to 
that label in the event of an error.  This must, of course, be an 
executable statement and in the same program unit.  For example:  
\begin{verbatim}
      READ(UNIT=IN, FMT=*, ERR=999) VOLTS, AMPS 
      WATTS = VOLTS * AMPS 
* rest of program in here . . . . . and finally 
      STOP 
999   WRITE(UNIT=*,FMT=*)'Error reading VOLTS or AMPS' 
      END 
\end{verbatim} 
This method has its uses but is open to the same objections as the 
{\tt GO TO} statement: it often leads to badly-structured programs with 
lots of arbitrary jumps.  

By using both {\tt IOSTAT=} and {\tt ERR=} in the same statement it is
possible to find out the actual error number and jump to the
error-handling code.  The presence of either keyword in an I/O statement
will allow the program to continue after an I/O error; on most systems it
also prevents an error message being issued.

The {\tt ERR=} and {\tt IOSTAT=} items can be used in all I/O statements.
Professional programmers should make extensive use of these
error-handling mechanisms to enhance the robustness and user-friendliness
of their software.

There is one fairly common mistake which does not count as an 
errors for this purpose: if you write a number to a formatted record 
using a field width too narrow to contain it, the field will simply be 
filled with asterisks. 

If an error occurs in a data transfer statement then the position of 
the file becomes indeterminate.  It may be quite difficult to locate 
the offending record if an error is detected when transferring a 
large array or using a large number of records.  

 
\subsubsection*{End-of-file Detection} 

A {\tt READ} statement which tries to read a record beyond the end of a
sequential or internal file will trigger the end-of-file condition. If an
item of the form: {\tt IOSTAT=integer-variable} is included in its
control-list then the status value will be returned as some negative
number.  If it includes an item of the form: {\tt END=label} then control
is transferred to the labelled statement when the end-of-file condition
is detected.

The {\tt END=} keyword may only be used in {\tt READ} statements, but it 
can be used in the presence of both {\tt ERR=} and {\tt IOSTAT=}  
keywords.  End-of-file detection is very useful when reading a file 
of unknown length, but some caution is necessary.  If you read 
several records at a time from a formatted file there is no easy way 
of knowing exactly where the end-of-file condition occurred.  The 
data list items beyond that point will have their values unaltered.  
Note also that there is no concept of end-of-file on direct-access 
files: it is simply an error to read a record which does not exist, 
whether it is beyond the ``end'' of the file or not. 

Most systems provide some method for signalling end-of-file on terminal 
input: those based on the ASCII code often use the character ETX which is 
usually produced by pressing control/Z on the keyboard (or EOT which is 
control/D).  After an end-of-file condition has been raised in this way 
it may persist, preventing further terminal input to that program. 

Formally, the Fortran Standard only requires Fortran systems to 
detect the end-of-file condition on external files if there is a special 
``end-file'' record on the end.  The {\tt END FILE} statement is provided 
specifically to write such a record.  In practice, however, virtually 
all Fortran systems respond perfectly well when you try to read the 
first non-existent record, so that the {\tt END FILE} statement is 
effectively obsolete and is not recommended for general use.  

\subsection{Format Specifications} 

Every READ {\tt} or {\tt WRITE} statement which uses a formatted external 
file or an internal file must include a format identifier.  This may 
have any of the following forms: 
\begin{description} 
\item[{\tt FMT=*}] This specifies a list-directed transfer (and is only 
permitted for external sequential files).  Detailed 
rules are given in section 10.10 below.  
\item[{\tt FMT=}{\em label}] The label must be attached to a FORMAT 
statement in the same program unit which 
provides the format specification. 
\item[{\tt FMT=}{\em char-exp}] The value of the character expression is a 
complete format specification.   
\item[{\tt FMT=}{\em char-array}] The elements of the character array
contain 
the format specification, which may occupy as many elements of the array 
as are necessary. 
\end{description} 
Note that the characters {\tt FMT=} may be omitted if it is the second 
item in the I/O control list and if the unit identifier with {\tt UNIT=}  
omitted comes first. 

A format specification consists a pair of parentheses enclosing a 
list of items called edit descriptors.  Any blanks before the left 
parenthesis will be ignored and (except in a FORMAT statement) 
all characters after the matching right parenthesis are ignored. 

In most cases the format can be chosen when the program is 
written and the simplest option is to use a character constant: \\ 
\verb?      WRITE(UNIT=LP, FMT='(1X,A,F10.5)') 'Frequency =', HERTZ?\\ 
Alternatively you can use a {\tt FORMAT} statement:  
\begin{verbatim}
      WRITE(UNIT=LP, FMT=915) 'Frequency =', HERTZ 
915   FORMAT(1X, A, F10.5) 
\end{verbatim}
This allows the same format to be used by more than one data-transfer
statement.  The {\tt FORMAT} statement may also be the neater form if the
specification is long and complicated, or if character-constant
descriptors are involved, since the enclosing apostrophes have to be
doubled up if the whole format is part of another character constant.

It is also possible to compute a format specification at run-time by 
using a suitable character expression.  By this means you could, for 
example, arrange to read the format specification of a data file 
from the first record of the file.  The program fragment below 
shows how to output a real number in fixed-point format (F10.2) 
when it is small, changing to exponential format (E18.6) when it 
is larger.  A threshold of a million has been chosen here. 
\begin{verbatim}
      CHARACTER F1*(*), F2*12, F3*(*) 
*Items F1, F2, F3 hold the three parts of a format specification. 
*F1 and F3 are constants, F2 is a variable. 
      PARAMETER (F1 = '(1X,''Peak size ='',') 
      PARAMETER (F3 = ')') 
*... calculation of PEAK assumed to be in here 
      IF(PEAK .LT. 1.0E6) THEN 
          F2 = 'F10.2' 
      ELSE 
          F2 = 'E18.6' 
      END IF 
      WRITE(UNIT=*, FMT=F1//F2//F3) PEAK 
\end{verbatim} 
Note that the apostrophes surrounding the character constant {\tt 'Peak 
size ='} have been doubled in the {\tt PARAMETER} statement because 
they are inside another character constant.  Here are two examples 
of output records, with blanks shown explicitly:  
\begin{verbatim}
Peak size =  12345.67 
Peak size =      0.987654E+08 
\end{verbatim} 

\subsubsection*{{\tt FORMAT} statement} 

The {\tt FORMAT} statement is classed as non-executable and can, in 
principle, go almost anywhere in the program unit.  A {\tt FORMAT} 
statement can, of course, be continued so its maximum length is 20 
lines.  The same {\tt FORMAT} statement can be used by more than 
one data transfer statement and, unless it contains character 
constant descriptors, used for both input and output.  Since it is 
very easy to make a mistake in matching the items in a data 
transfer list with the edit descriptors in the format specification, it 
makes sense to put the {\tt FORMAT} statement as close as possible to 
the {\tt READ} and {\tt WRITE} statements which use it.   

\subsection{Format Edit Descriptors} 

There are two types of edit descriptor: data descriptors and control 
descriptors. 

A data descriptor must be provided for each data item transferred 
by a {\tt READ} or {\tt WRITE} statement; the descriptors permitted depend 
on the data type of the item.  The data descriptors all start with a 
letter indicating the data type followed by an unsigned integer 
denoting the field width, for example:  
{\tt I5 } denotes an integer field 5 characters wide,  
{\tt F9.2} denotes a floating-point field 9 character wide with 2 
digits after the decimal point.   
Full details of all the data descriptors are given in the next section. 

The control descriptors are used for a variety of purposes, such as 
tabbing to specific columns, producing or skipping records, and 
controlling the transfer of subsequent numerical data.  They are 
described fully in section 10.9. 

Note that only literal constants are permitted within format 
specifications, not named constants or variables. 

\subsection{Format Data Descriptors \texttt{A, E, F, G, I, L}} 

A data descriptor must be provided for each data item present (or 
implied) in a data transfer list.  Real, double precision, and 
complex items may use any of the E, F, or G descriptors but in all 
other cases the data type must match.  Two floating-point 
descriptors are needed for each complex value. 

\begin{center} 
\begin{tabular}{ll} 
\hline 
Data type & Data descriptors \\ 
\hline 
Integer    & {\tt Iw, Iw.m} \\ 
Real, Double Precision, or Complex & {\tt Ew.d, Ew.dEe, Fw.d, Gw.d,
Gw.dEe} \\ 
Logical    & {\tt Lw} \\ 
Character  & {\tt A, Aw} \\ 
\hline 
\end{tabular} 
\end{center} 

The letters {\tt w, m, d}, and {\tt e} used with these data descriptors 
represent unsigned integer constants; {\tt w} and {\tt e} must be greater
than 
zero.   

\begin{center} 
\begin{tabular}{ll} 
{\tt w     } & is the total field width.\\ 
{\tt m     } & is the minimum number of digits produced on output.\\ 
{\tt d     } & is the number of digits after the decimal point.\\ 
{\tt e     } & is the number of digits used for the exponent.\\ 
\end{tabular} 
\end{center} 
Any data descriptor can be preceded by a repeat-count (also an 
unsigned integer), thus:\\ 
\verb?      3F6.0? is equivalent to \verb? F6.0,F6.0,F6.0?\\ 
This facility is particularly useful when handling arrays. 

\subsubsection*{General Rules for Numeric Input/Output} 

Numbers are always converted using the decimal number base: 
there is no provision in Standard Fortran for transfers in other 
number bases such as octal or hexadecimal.  More complicated 
conversions such as these can be performed with the aid of internal 
files.  

On output numbers are generally right-justified in the specified field; 
leading blanks are supplied where necessary.  Negative values are always 
preceded by a minus sign (for which space must be allowed in the field); 
zero is always unsigned; the {\tt SP} and {\tt SS} descriptors control 
whether positive numbers are to be preceded by a plus sign or not.  A 
number which is too large to fit into its field will appear instead as a 
set of w asterisks. 

On input numbers should be right-justified in each field.  All forms 
of constant permitted in a Fortran program can safely be used in an 
input field of the corresponding type, as long there are no 
embedded or trailing blanks.  Leading blanks are always ignored; 
a field which is entirely blank will be read as zero.  The treatment 
of embedded and trailing blanks can be controlled with the {\tt BN} and 
{\tt BZ} descriptors.  The rules here are another relic of very early 
Fortran systems.  

When reading a file which has been connected by means of an 
{\tt OPEN} statement (provided it does not contain {\tt BLANK='ZERO'}) 
all embedded and trailing blanks in numeric input fields are treated 
as nulls, i.e.\ they are ignored.  In all other cases, such as input from 
the standard pre-connected file or from an internal file, embedded 
and trailing blanks are treated as zeros.  These defaults can be 
altered with the {\tt BN} and {\tt BZ} control descriptors.    
It is hard to imagine any circumstances in which it is desirable to 
interpret embedded blanks as zeros; the default settings are 
particularly ill-chosen since numbers entered by a user at a 
terminal are often left-justified and may appear to be padded out 
with zeros.  Errors from this source can be avoided by using BN at 
the beginning of all input format specifications. 

\subsubsection*{Integer Data ({\tt Iw, Iw.m})} 

An integer value written with Iw appears right-justified in a field 
of w characters with leading blanks.  Iw.m is similar but ensures 
that at least m digits will appear even if leading zeros are 
necessary.  This is useful, for instance, to output the times in hours 
and minutes:  
\begin{verbatim}
       NHOURS = 8 
       MINUTE = 6 
       WRITE(UNIT=*, FMT='(I4.2, I2.2)') NHOURS, MINUTE 
\end{verbatim} 
The output record (with blanks shown explicitly) is: 
\begin{verbatim}
  0806 
\end{verbatim} 
On input {\tt Iw} and {\tt Iw.m} are identical.  Note that an integer 
field must not contain a decimal point, exponent, or any punctuation 
characters such as commas. 

\subsubsection*{Floating Point Data ({\tt Ew.d, Ew.dEe, Fw.d, Gw.d, 
Gw.dEe})} 

Data of any of the floating-point types (Real, Double Precision, 
and Complex) may be transferred using any of the descriptors {\tt E}, 
{\tt F,} or {\tt G}. For each complex number two descriptors must be 
provided, one for each component; these components may be 
separated, if required, by control descriptors.  On output numbers 
are rounded to the specified number of digits.  All floating-point 
data transfers are affected by the setting of the scale-factor; this is 
initially zero but can be altered by the {\tt P} control descriptor, as 
explained in the section 10.9. 

Output using {\tt Fw.d} produces a fixed-point value in a field of {\tt }w 
characters with exactly {\tt d} digits after the decimal point.  The 
decimal point is present even if {\tt w} is zero, so that if a sign is 
produced there is only space for, at most, {\tt w-2} digits before the 
decimal point.  If it is really important to suppress the decimal 
point in numbers with no fractional part one way is to use a format 
specification of the form {\tt (F15.0,TL1})... so that the next field
starts 
one place to the left and over-writes the decimal point.  Another 
way is to copy the number to an integer variable and write it with 
an I descriptor, but note the limited range of integers on most 
systems.  {\tt F} format is especially convenient in tabular layouts since 
the decimal points will line up in successive records, but it is not 
suitable for very large or small numbers. 

Output with {\tt Ew.d} produces a number in exponential or ``scientific''
notation.  The mantissa will be between 0.1 and 1 (if the scale-factor is
zero).  The form {\tt Ew.dEe} specifies that there should be exactly {\tt
e} digits in the exponent.  This form must be used if the exponent will
have more than three digits (although this problem does not arise on
machines on which the number range is too small).  {\tt E} format can be
used to handle numbers of any magnitude. The disadvantage is that
exceptionally large or small values do not stand out very well in the
resulting columns of figures.

{\tt Gw.d} is the general-purpose descriptor: if the value is greater than 
0.1 but not too large to fit it the field it will be written using a 
fixed-point format with {\tt d} digits in total and with 4 blanks at the 
end of the field; otherwise it is equivalent to {\tt Ew.d} format.  The 
form {\tt Gw.dEe} allows you to specify the length of the exponent; if 
a fixed-point format is chosen there are {\tt e+2} blanks at the end.   

The next example shows the different properties of these three 
formats on output: 
\begin{verbatim}
      X = 123.456789 
      Y = 0.09876543 
      WRITE(UNIT=*, FMT='(E12.5, F12.5, G12.5)') X,X,X, Y,Y,Y 
\end{verbatim} 
produces two records (with $\sqcup$ representing the blank): 
\begin{verbatim}
 0.12346E+03   123.45679  123.46     
 0.98766E-01     0.09877 0.98766E-01 
\end{verbatim} 
On input all the {\tt E,} {\tt F,} and {\tt G} descriptors have identical 
effects: if the input field contains an explicit decimal point it always 
takes precedence, otherwise the last d digits are taken as the decimal 
fraction.  If an exponent is used it may be preceded by {\tt E} or {\tt 
D} (but the exponent letter is optional if the exponent is signed).  If 
the input field provides more digits than the internal storage can 
utilise, the extra precision is ignored.  It is usually best to use ({\tt 
Fw.0}) which will cope with all common floating-point or even integer 
forms. 

\subsubsection*{Logical Data ({\tt Lw})} 

When a logical value is written with Lw the field will contain the letter 
{\tt T} or {\tt F} preceded by {\tt (w-1)} blanks.  On input the field 
must contain the letter {\tt T} or {\tt F}; the letter may be preceded by 
a decimal point and any number of blanks.  Characters after the {\tt T} 
or {\tt F} are ignored. Thus the forms {\tt .TRUE.} and {\tt .FALSE.} are 
acceptable. 

\subsubsection*{Character Data ({\tt A} and {\tt Aw})} 

If the A descriptor is used without an explicit field-width then the 
length of the character item in the data-transfer list determines it.  
This is generally what is required but note that the position of the 
remaining items in the record will change if the length of the 
character item is altered. 

If it is important to use fixed column layouts the form {\tt Aw} may be 
preferred: it always uses a field w characters wide.  On output if 
the actual length len is less than w the value is right-justified in the 
field and preceded by {\tt (w-len)} blanks; otherwise only the first {\tt
w} 
characters are output, the rest are ignored.  On input if the length 
len is less than w then the right-most len characters are used, 
otherwise w characters will be read into the character variable with 
{\tt (len-w)} blanks appended. 

\subsection{Format Control Descriptors} 

Control descriptors do not correspond to any item in the data-transfer 
list: they are obeyed when the format scan reaches that point in the 
list.  A format specification consisting of nothing but control 
descriptors is valid only if the {\tt READ} or {\tt WRITE} statement has
an empty 
data-transfer list. 

\begin{center} 
\begin{tabular}{ll} 
\hline 
Control Function & Control Descriptor \\ 
\hline 
Skip to next record & {\tt /} \\ 
Move to specified column position & {\tt Tn, TLn, TRn, nX} \\ 
Output a character constant & {\tt 'any char string'} \\ 
Stop format scan if data list empty & {\tt :} \\ 
Control {\tt +} before positive numbers & {\tt SP, SS, S} \\ 
Treat blanks as nulls/zeros & {\tt BN, BZ} \\ 
Set scale factor for numeric transfers & {\tt kP} \\ 
\hline 
\end{tabular} 
\end{center} 

Here {\tt n} and {\tt k} are integer constants, {\tt k} may have a sign
but n must 
be non-zero and unsigned.  The control descriptors such as {\tt SP}, 
{\tt BN,} {\tt kP} affect all numbers transferred subsequently.  The
settings 
are unaffected by forced reversion but the system defaults are 
restored at the start of the next {\tt READ} or {\tt WRITE} operation.  

Any list of edit descriptors may be enclosed in parentheses and 
preceded by an integer constant as a repetition count, e.g.\\ 
\verb?      2(I2.2, '-'),I2.2 ?\\ 
is equivalent to\\ 
\verb?      I2.2, '-', I2.2, '-', I2.2 ?\\ 
These sub-lists can be nested to any reasonable depth, but the 
presence of internal pairs of parentheses can have special effects 
when forced reversion takes place, as explained later. 

Commas may be omitted between items in the following special cases: 
either side of a slash ({\tt /}) or colon ({\tt :}) descriptor, and after 
a scale-factor ({\tt kP}) if it immediately precedes a {\tt D,} {\tt E,} 
{\tt F,} or {\tt G} descriptor. 

\subsubsection*{Record Control ({\tt /})} 

The slash descriptor ({\tt /}) starts a new record on output or skips to a 
new record on input, ignoring anything left on the current record.  
On a text file a record normally corresponds to a line of text.  Note 
that a formatted transfer always process at least one record: if the 
format contains N slashes then a total of (N+1) records are 
processed.  With N consecutive slashes in an output format there 
will be (N-1) blank lines; on input then (N-1) lines will be ignored.  
Note that if a formatted sequential file is sent to a printer the first 
character of every record may be used for carriage-control (see 
section 10.11).  It is good practice to put {\tt 1X} at the beginning of 
every format specification and after every slash to ensure single 
line spacing.  Here, for example, there is a blank line after the 
column headings. 
\begin{verbatim}
      WRITE(UNIT=LP, FMT=95) (NYEAR(I), POP(I), I=1,NYEARS) 
95    FORMAT(1X,'Year  Population', //, 100(1X, I4, F12.0, /)) 
\end{verbatim} 

\subsubsection*{Column Position Control ({\tt Tn, TLn, TRn, nX})} 

These descriptors cause subsequent values to be transferred 
starting at a different column position in the record.  They can, for 
instance, be used to set up a table with headings positioned over 
each column.  In all these descriptors the value of n must be 1 or 
more.  Columns are numbered from 1 on the left (but remember 
that column 1 may be used for carriage-control if the output is sent 
to a printer).  

\begin{center} 
\begin{tabular}{p{0.5in}p{5in}} 
\hline 
{\tt Tn } & causes subsequent output to start at column n.\\ 
{\tt TRn} & causes a shift to the right by n columns. \\ 
{\tt TLn} & causes a shift to the left by n columns (but it will not 
move the position to the left of column 1).  \\ 
{\tt nX } & is exactly equivalent to TRn.\\ 
\hline 
\end{tabular} 
\end{center} 
       
On input {\tt TLn} can be used to re-read the same field again, possibly 
using a different data descriptor.  On output these descriptors do 
not necessarily have any direct effect on the record: they do not 
cause any existing characters to be replaced by blanks, but when 
the record is complete any column positions embedded in the 
record which are still unset will be replaced by blanks.  Thus: 
\begin{verbatim}
      WRITE(UNIT=LP, FMT=9000) 
9000  FORMAT('A', TR1000, TL950, 'Z') 
\end{verbatim} 
will cause a record of 52 characters to be output, middle 50 of 
them blanks. 

\subsubsection*{Character Constant Output ({\tt 'string'})} 

The character constant descriptor can only be used with {\tt WRITE} 
statements: the character string is simply copied to the output 
record.  As in all character constants an apostrophe can be 
represented in the string by two successive apostrophes, and blanks 
are significant.  

\subsubsection*{Sign Control ({\tt SP, SS, S})} 

After {\tt SP} has been used, positive numbers will be written with a 
leading {\tt +} sign; after {\tt SS} has been used the {\tt +} sign is 
suppressed. The {\tt S} descriptor restores the initial default which is 
system-dependent.  These descriptors have no effect on numerical input. 
The initial default is restored at the start of every new formatted 
transfer. 

\subsubsection*{Blank Control ({\tt BN, BZ})} 

After {\tt BN} is used all embedded and trailing blanks in numerical 
input fields are treated as nulls, i.e.\ ignored.  After {\tt BZ} they are 
treated as zeros.  These descriptors have no effect on numerical 
output.  The initial default, which depends on the {\tt BLANK=} item 
in the {\tt OPEN} statement, is restored at the start of every new 
formatted transfer. 

\subsubsection*{Scale Factor Control ({\tt kP})} 

The scale factor can be used to introduce a scaling by any power 
of 10 between internal and external values when {\tt E, F}, or {\tt G} 
descriptors are used.  In principle this could be useful when 
dealing with data which are too large, or too small, for the 
exponent range of the floating-point data types of the machine, but 
other problems usually make this impracticable.  The scale 
factor can result in particularly insidious errors when used with {\tt F} 
descriptors and should be avoided by all sensible programmers.  
The rules are as follows. 

The initial scale factor in each formatted transfer is zero.  If the 
descriptor {\tt kP} is used, where {\tt k} is a small (optionally signed) 
integer, then it is set to {\tt k}. It affects all subsequent floating
point 
values transferred by the statement.  On input there is no effect if 
the input field contains an explicit exponent, otherwise\\ 
\centerline{{\em internal-value} = {\em external-value} / $10^{k}$}\\ 
On output the effect depends on the descriptor used.  With {\tt E} 
descriptors the decimal point is moved {\tt k} places to the right and 
the exponent reduced by {\tt k} so the effective value is unaltered. 
With {\tt F} descriptors there is always a scaling: \\ 
\centerline{{\em external-value} = {\em internal-value} * $10^{k}$}\\ 
With {\tt G} descriptors the scale-factor is ignored if the value is in
the 
range for {\tt F}-type output, otherwise it has the same effects as with 
{\tt E} descriptors. 

\subsubsection*{Scan Control ({\tt :}) and Forced Reversion} 

The list of edit descriptors is scanned from left to right (apart from 
the effect of parentheses) starting at the beginning of the list 
whenever a new data transfer statement is executed.  The action of 
the I/O system depends jointly on the next edit descriptor and the 
next item in the data-transfer list.  If a data descriptor comes next then 
the next data item is transferred if one exists, otherwise the format 
scan comes to an end.  If a colon descriptor ({\tt :}) comes next and the 
data-transfer list is empty the format scan ends, otherwise the 
descriptor has no effect.  If any other control descriptor comes next 
then it is obeyed whether or not the data-transfer list is empty. 

If the format list is exhausted when there are still more items in the 
data-transfer list then forced reversion occurs: the file is positioned 
at the beginning of the next record and the format list is scanned 
again, starting at the left-parenthesis matching the last preceding 
right-parenthesis.  If this is preceded by a repeat-count then this 
count is re-used.  If there is no preceding right-parenthesis then the 
whole format is re-used.  Forced reversion has no effect upon the 
settings for scale-factor, sign, or blank control.  Forced reversion 
can be useful when reading or writing an array contained on a 
sequence of records since it is not necessary to know how many 
records there are in total, but when producing printed output it is 
easy to forget that a carriage-control character is required for each 
record, even those produced by forced reversion. 

\subsection{List-Directed Formatting} 

List-directed output uses a format chosen by the system according 
to the data type of the item.  The exact layout is system-dependent, 
but the general rules are as follows. 

\subsubsection*{List-directed Output} 

Each {\tt WRITE} statement starts a new record; additional records are 
produced when necessary.  Each record starts with a single blank 
to provide carriage-control on printing.  Arithmetic data types are 
converted to decimal values with the number of digits appropriate 
for the internal precision; integer values will not have a decimal 
point, the system may choose fixed-point or exponential 
(scientific) form for floating-point values depending on their 
magnitude.  Complex values are enclosed in parentheses with a 
comma separating the two parts. 

Logical values are output as a single letter, either {\tt T} or {\tt F}. 
Character values are output without enclosing apostrophes; if a 
character string is too long for one record it may be continued on 
the next. 
Except for character values, each item is followed by at least one 
blank or a comma (or both) to separate it from the next value. 

\subsubsection*{List-directed Input} 

The rules for List-directed input effectively allow free-format entry for
numerical data.  Each {\tt READ} statement starts with a new record and
reads as many records as are necessary to satisfy its data-transfer list.
The input records must contain a suitable sequence of values and
separators.

The values may be given in any form which would be acceptable 
in a Fortran program for a constant of the corresponding type, 
except that embedded blanks are only permitted in character 
values.  When reading a real or double-precision value an integer 
constant will be accepted; when reading a logical value only the 
letter {\tt T} or {\tt F} is required (a preceding dot and any following 
characters will be ignored).  Note that a character constant must be 
enclosed in apostrophes and a complex constant must be enclosed 
in parentheses with a comma between the two components.  If a 
character constant is too long to fit on one record it may be 
continued on to the next; the two parts of a complex constant may 
also be given on two records. 

The separator between successive values must be one or more 
blanks, or a comma, or both.  A new record may start at any point 
at which a blank would be permitted.   

If several successive items are to have the same value a repetition 
factor can be used: this has the form {\em n}{\tt *}{\em constant} where 
{\em n} is an unsigned integer.  Blanks are not allowed on either side of 
the asterisk. 

Two successive commas represent a null value: the corresponding 
variable in the {\tt READ} statement has its value unchanged.  It is also 
possible to use the form n* to represent a set of n null values. 

A slash ({\tt /}) may be used instead of an item separator; it has the 
effect of completing the current {\tt READ} without further input; all 
remaining items in its data transfer list are unchanged in value. 

List-directed output files are generally compatible with list-directed
input, unless they contain character items, which will not have the
enclosing apostrophes which are required on input.

\subsection{Carriage-Control and Printing} 

Although a format specification allows complete control over the 
layout of each line of text, it does not include any way of 
controlling pagination.  The only way to do this is by using a 
unique and extraordinary mechanism dating back to the earliest 
days of Fortran.  Even if you are not concerned with pagination 
you still need to know about the carriage-control convention since 
it is liable to affect every text file you produce.  

Whenever formatted output is sent to a ``printer'', the first character 
of every record is removed and used to control the vertical spacing.  
This carriage-control character must be one of the four listed in the 
the table below.  

\begin{center} 
\begin{tabular}{cl} 
\hline 
Character & Vertical spacing before printing \\ 
\hline 
{\em blank} & Advance to next line \\ 
{\tt 0} & Advance two lines \\ 
{\tt 1} & Advance to top of next page \\ 
{\tt +} & No advance, i.e.\ print on same line\\ 
\hline 
\end{tabular} 
\end{center} 

An empty record is treated as if it started with a single blank.  For 
example, these statements start a new page with a page number at 
the top and a title on the third line:  
\begin{verbatim}
      WRITE(LP, 55) NUMBER, 'Report and Accounts' 
55    FORMAT('1PAGE', I4, /, '0', A) 
\end{verbatim} 
This carriage-control convention is an absurd relic which 
causes a multitude of problems in practice.  Firstly, systems differ 
on what they call a ``printer'': it may or may not apply to visual 
display terminals or to text initially saved on a disc file and later 
printed out.  Some operating systems have a special file type for 
Fortran formatted output which is treated differently by printers 
(and terminals).  Others have been known to chop off the first 
character of all files sent to the system printer so that special 
utilities are needed to print ordinary text. 

To be on the safe side you should always provide an explicit 
carriage-control character at the start of each format specification 
and after each slash.  Special care is needed in formats which use 
forced reversion.  Normal single spacing is obtained with a blank, 
conveniently produced by the {\tt 1X} edit descriptor.  If you forget 
and accidentally print a number at the start of each record with a 
leading digit {\tt 1} then each record will start a new page.  

The effect of {\tt +} as a carriage-control character would be more 
useful if its effects were more predictable.  Some devices over-print the 
previous record, allowing the formation of composite characters (for
example overprinting equals with a slash could give you a 
not-equals sign), while
others append to it, and some (including many visual display terminals) 
erase what was there before.  Obviously you cannot rely on this and in 
portable software there is no 
alternative but to ignore the {\tt +} case altogether. 

Standard Fortran can only use the four carriage-control characters 
listed in the table but many systems use other symbols for special 
formatting purposes, such as setting vertical spacing, changing 
fonts, and so on.  One extension which is widely available is the 
use of the currency symbol {\tt \$} to suppress carriage-return at the end 
of the line.  This can be useful when producing terminal prompts 
as it allows the reply to be entered on the same line.  There is, 
unfortunately, no way of doing this in Standard Fortran. 

The rules for list-directed output ensure that the lines are
single-spaced by requiring at least one blank at the start of every
record.

\subsection{Input/Output Statements and Keywords} 

The I/O statements fall into three groups: 
\begin{itemize} 
\item  The data transfer statements {\tt READ} and {\tt WRITE}.  
\item  The file connection statements {\tt OPEN, CLOSE,} and 
{\tt INQUIRE}. 
\item  The file positioning statements {\tt REWIND} and 
{\tt BACKSPACE}.  
\end{itemize} 

All these statements have a similar general form (except that only 
the {\tt READ} and {\tt WRITE} statements can have a data-transfer
list):\\ 
\verb+      READ(+ {\em control-list} {\tt )} {\em input-list}\\ 
\verb+      WRITE(+ {\em control-list} {\tt )} {\em output-list}\\ 
The items in each list are separated by commas.  Those in the 
control list are usually specified by keywords, in which case the 
order does not matter, although it is conventional to have the unit 
identifier first.  For compatibility with Fortran66, if the unit 
identifier does come first then the keyword {\tt UNIT=} may be 
omitted.  Furthermore, when this keyword is omitted in {\tt READ} and 
{\tt WRITE} statements and the format identifier is second its keyword 
may also be omitted.  Thus these two statements are exactly 
equivalent: 
\begin{verbatim}
      READ(UNIT=1, FMT=*, ERR=999) AMPS, VOLTS, HERTZ 
      READ(1, *, ERR=999) AMPS, VOLTS, HERTZ 
\end{verbatim} 
Use of this abbreviated form is a matter of taste: for the sake of 
clarity the keywords will all be shown in other examples.   

Many of the keywords in the control-list can take a character 
expression as an argument: in such cases any trailing blanks in the 
value will be ignored.  This makes it easy to use character variables 
to specify file names and characteristics. 

There is one general restriction on expressions used in all I/O 
statements: they must not call external functions which themselves 
execute further I/O statements.  This restriction avoids the 
possibility of recursive calls to the I/O system. 

\subsection{\texttt{OPEN} Statement} 

The {\tt OPEN} statement is used to connect a file to an I/O unit and 
describe its characteristics.  It can open an existing file or create a 
new one.  If the unit is already connected to another file then this 
is closed before the new connection is made, so that it is 
impossible to connect two files simultaneously to the same unit.  
It is an error to try to connect more than one unit simultaneously 
to the same file.  In the special case in which the unit and file are 
already connected to each other, the {\tt OPEN} statement may be used 
to alter the properties of the connection, although in practice only 
the {\tt BLANK=} (and sometimes {\tt RECL=}) values can be changed in 
this way. 

The Fortran Standard does not specify the file position when an 
existing sequential file is opened.  Although most operating 
systems behave sensibly, in portable software a {\tt REWIND} 
statement should be used to ensure that the file is rewound before 
you read it.  

The general form of the {\tt OPEN} statement is just:\\ 
\verb+      OPEN(+ {\em control-list} {\tt )}\\ 
The {\em control-list} can contain any of the following items in any 
order: 
\begin{description} 
\item{{\tt UNIT=}{\em integer-expression}} 
species the I/O unit number which must be 
zero or above; the upper limit is system-dependent, typically 99 or 
255.  The unit identifier must always be given, there is no default 
value. 

\item{{\tt STATUS=}{\em character-expression}} 
describes or specifies the file status.  The 
value must be one of: 

\begin{tabular}{p{1in}p{4.5in}} 
{\tt 'OLD'      } &  The file must exist. \\ 
{\tt 'NEW'      } & The file must not already exist, a new file is 
created. \\ 
{\tt 'SCRATCH'  } & An unnamed temporary file is created; it is deleted 
automatically when the program exits. \\ 
{\tt 'UNKNOWN'  } & The effect is system-dependent, but usually 
means that an old file will be used if one 
exists, otherwise a new file will be created.\\ 
\end{tabular} 

The default value is {\tt 'UNKNOWN',} but it is unwise to omit the {\tt
STATUS} keyword because the effect of {\tt 'UNKNOWN'} is so ill-defined.

\item{{\tt FILE=}{\em character-expression}}
specifies the file-name (but any trailing blanks will be ignored).  The
forms of file-name acceptable are system-dependent: a complete
file-specification on some operating systems may include the device,
user-name, directory path, file-type, version number etc. and may require
various punctuation marks to separate these.  In portable software, where
the name has to be acceptable to a variety of operating systems, short
and simple names should be used.  Alternatively the {\tt FILE=}
identifier may be a character variable (or array element) so that the
user can choose a file-name at run-time.  There is no default for the
file-name so one should be specified in all cases unless {\tt
STATUS='SCRATCH'} in which case the file must not be named.

\item{{\tt ACCESS=}{\em character-expression}} 
specifies the file access method.  The 
value may be either: 

\begin{tabular}{p{1.2in}p{4.3in}} 
{\tt 'SEQUENTIAL'} & a sequential file: this is the default. \\ 
{\tt 'DIRECT'    } & a direct-access file: in this case the {\tt RECL=} 
keyword is also needed.\\ 
\end{tabular} 

\item{{\tt FORM=}{\em character-expression}} 
specifies the record format.  The value 
may be either: 

\begin{tabular}{ll} 
{\tt 'FORMATTED'      } &  the default for a sequential file.\\ 
{\tt 'UNFORMATTED'    } &  the default for a direct-access file.\\ 
\end{tabular} 

\item{{\tt RECL=}{\em integer-expression}}
specifies the record length.  This must be given for a direct-access file
but not otherwise.  The record-length is measured in characters for a
formatted file, but is in system-dependent units (often numeric storage
units) for an unformatted file.

\item{{\tt BLANK=}{\em character-expression}} 
specifies how embedded and trailing 
blanks in numerical input fields of formatted files are to be treated 
(in the absence of explicit format descriptors {\tt BN} or {\tt BZ}). The 
value may be either:  

\begin{tabular}{ll} 
{\tt 'NULL' } & blanks treated as nulls, i.e.\ ignored: the default. \\ 
{\tt 'ZERO' } & blanks treated as zeros. \\ 
\end{tabular} 

The default value is likely to be the sensible choice in all cases. 

\item{{\tt IOSTAT=}{\em integer-variable}}
(or array element) returns the I/O status code after execution of the
{\tt OPEN} statement.  This will be zero if no error has occurred,
otherwise it will return a system-dependent positive value.

\item{{\tt ERR=}{\em label}} 
transfers control to the labelled executable statement in the same 
program unit in the event of an 
error. 

\end{description} 

\subsection{\texttt{CLOSE} Statement} 

The {\tt CLOSE} statement is used to close a file and break its 
connection to a unit.  The unit and the file (if it still exists) are then 
free for re-use in any way.  If the specified unit is not connected to 
a file the statement has no effect.  The general form of the 
statement is:\\ 
\verb+      CLOSE(+ {\em control-list} {\tt )}\\ 
where the {\em control-list} may contain the following items: 
\begin{description} 
\item{{\tt UNIT=}{\em integer-expression}} 
specifies the unit number to close (the same as in the {\tt OPEN}
statement).  

\item{{\tt STATUS=}{\em character-expression}} 
specifies the status of the file after closure. 
The expression must have a value of either: 
{\tt 'KEEP'  } for the file to be preserved, or 
{\tt 'DELETE'} for the file to be deleted after closure. 
The default is {\tt STATUS='KEEP'} except for files opened with 
{\tt STATUS='SCRATCH':} such files are always deleted after closure 
and {\tt STATUS='KEEP'} is not permitted. 
\item{{\tt IOSTAT}{\em =integer-variable} and {\tt ERR}{\em =label}} 
are both permitted, as in the {\tt OPEN} statement (but not much 
can go wrong with a {\tt CLOSE} statement). 
\end{description} 

\subsection{\texttt{INQUIRE} Statement} 

The {\tt INQUIRE} statement can be used in two slightly different 
forms: 
\\ 
\verb?      INQUIRE(UNIT=? {\em integer-expression, inquire-list} {\tt
)}\\ 
\verb?      INQUIRE(FILE=? {\em character-expression, inquire-list} {\tt 
)}\\ 
The first form, an inquire by unit, returns information about the 
unit and, if it is connected to a file, about the file as well.  If it is 
not connected to a file then most of the arguments will be 
undefined or return a value of {\tt 'UNKNOWN'} as appropriate. 

The second form, inquire by file, can always be used to find out 
whether a named file exists, i.e.\ can be opened by a Fortran 
program.  Any trailing blanks in the character expression are 
ignored, and the forms of file-name acceptable are, as in the {\tt OPEN} 
statement, system-dependent. If the file exists and is connected to 
a unit then much more information can be obtained.  

The inquire-list may contain any of the items below.  Note that all 
of them (except for {\tt ERR=label}) return information by assigning a 
value to a named variable (or array element).  The normal rules of 
assignment statements apply, so that character items may 
have any reasonable length and will return a value which is padded out 
with blanks to its declared length if necessary. 

\begin{description} 
\item{{\tt IOSTAT=}{\em integer-variable}} and 
{\tt ERR=}{\em label} can both be used in the same way as in {\tt OPEN} 
or {\tt CLOSE}; note that they detect errors during the execution of the 
{\tt INQUIRE} statement itself, and do not reflect the state of the file 
or unit which is the subject of the inquiry. 

\item{{\tt EXIST=}{\em logical-variable}} 
sets the variable (or array-element) 
to {\tt .TRUE.} if the specified unit or file exists, or 
{\tt .FALSE.} if it does not.  A unit exists if it has a number in the 
permitted range.  A file exists if it can be used in an {\tt OPEN} 
statement.  A file may appear not to exist merely because the 
operating system prevents its use, for example because a password 
is needed or because some other user has already opened it. 

\item{{\tt OPENED=}{\em logical-variable}} 
sets the variable to {\tt .TRUE.} if the specified unit (or file) is 
currently connected to a file (or unit) in the program. 

\item{{\tt NUMBER=}{\em integer-variable}} 
returns the unit number of a file which is connected to the variable; 
otherwise it becomes undefined. 

\item{{\tt NAME=}{\em character-variable}}
returns the file-name to the variable if the file has a name; if not it
becomes undefined.  In the case of an inquire by file the name may not be
the same as that specified using {\tt FILE=} (because a device-name or
directory path may have been added) but the name returned will always be
suitable for use in an {\tt OPEN} statement.

\item{{\tt NAMED=}{\em logical-variable}}
sets the variable to {\tt .TRUE.} if the specified file has a name.

\item{{\tt ACCESS=}{\em character-variable}} 
returns the record access-method, either 
{\tt 'SEQUENTIAL'} or {\tt 'DIRECT'} if the file is connected; if it is
not 
connected the variable becomes undefined. 

\item{{\tt SEQUENTIAL=}{\em character-variable}} 
returns {\tt 'YES'} if the file can be opened for sequential 
access, {\tt 'NO'} if it cannot, and {\tt 'UNKNOWN'} otherwise. 

\item{{\tt DIRECT=}{\em character-variable}} 
returns {\tt 'YES'} if the file can be opened for direct 
access, {\tt 'NO'} if it cannot, and {\tt 'UNKNOWN'} otherwise. 

\item{{\tt FORM=}{\em character-variable}} 
returns {\tt 'FORMATTED'} if the file is connected for 
formatted access, returns {\tt 'UNFORMATTED'} if it is connected for 
unformatted access, or becomes undefined if there is no 
connection.  

\item{{\tt FORMATTED=}{\em character-variable}} 
returns {\tt 'YES'} if formatted access is permitted, {\tt 
'NO'} if it is not, or {\tt 'UNKNOWN'} otherwise. 

\item{{\tt UNFORMATTED=}{character-variable}} 
returns {\tt 'YES'} if unformatted access is permitted, {\tt 
'NO'} if it is not, or {\tt 'UNKNOWN'} otherwise. 

\item{{\tt RECL=}{\em integer-variable}} 
returns the record length if the file is connected for 
direct-access but becomes undefined otherwise.  Note that the units 
are characters for formatted files, but are sys\-tem-de\-pen\-dent for 
unformatted files.  

\item{{\tt NEXTREC=}{\em integer-variable}} 
returns a number which is one higher than the last 
record read or written if the file is connected for direct access.  If 
it is connected for direct access but no records have been 
transferred, the variable returns one.  If the file is not connected for 
direct access the variable becomes undefined.   

\item{{\tt BLANK=}{\em character-variable}} 
returns {\tt 'NULL'} or {\tt 'BLANK'} if the file is 
connected for formatted access according to the way embedded and trailing 
blanks are to be treated.  In other cases it becomes undefined. 
\end{description} 

\subsection{\texttt{READ} and \texttt{WRITE} Statements} 

The {\tt READ} statement reads information from one or more records 
on a file into a list of variables, array elements, etc.  The {\tt WRITE} 
statement writes information from a list of items which may 
include variables, arrays, and expressions and produces one or 
more records on a file.  Each {\tt READ} or {\tt WRITE} statement can 
only transfer one record on an unformatted file but on formatted 
files, including internal files, more than one record may be 
transferred, depending on the contents of the format specification.  

The two statements have the same general form:\\ 
\verb+      READ(+  {\em control-list} {\tt )} {\em data-list}\\ 
\verb+      WRITE(+ {\em control-list} {\tt )} {\em data-list}\\ 
The {\em control-list} must contain a unit identifier; the other items may 
be optional depending on the type of file.  The {\em data-list} is also 
optional: if it is absent the statement transfers one record (or 
possibly more under the control of a format specification). 

\subsubsection*{Unit Identifier} 
This may have any of the following forms: 
\begin{description} 
\item{\tt UNIT=} {\em integer-expression} The value of the expression
must be 
zero or greater and must refer to a valid I/O unit.  
\item {\tt UNIT=*} For the standard pre-connected input or output file. 
\item {\tt UNIT=}{\em internal-file} The internal-file may be a variable, 
array-element, substring, or array of 
type character, see section 10.3. 
\end{description} 

Note that the keyword {\tt UNIT=} is optional if the unit identifier is 
the first item in the control list. 

\subsubsection*{Format Identifier} 
A format identifier must be provided when using a formatted (or 
internal) file but not otherwise.  It may have any of the following 
forms: 
\begin{description} 
\item {\tt FMT=}{\em label} The label of a {\tt FORMAT} statement in the 
same program unit. 
\item {\tt FMT=}{\em format}  The format may be a character 
expression or character array containing 
a complete format specification (section 
10.6).  
\item {\tt FMT=*} For list-directed formatting (section 10.10). 
\end{description} 
Note that the keyword {\tt FMT=} is also optional if the format 
identifier is the second item in the control list and the first item is 
a unit identifier specified without its keyword.  

\subsubsection*{Record Number} 
A record number identifier must be provided for direct-access files 
but not otherwise.  It has the form: 

{\tt REC=}integer-expression 

The record number must be greater than zero; for {\tt READ} it must 
refer to a record which exists. 

\subsubsection*{Error and End-of-file Identifiers} 
These may be provided in any combination, but {\tt END=}label is only 
valid when reading a sequential or internal file.  See 10.5 for more 
information. 
\begin{verbatim}
END=label 
ERR=label 
IOSTAT=integer-variable 
\end{verbatim} 

The data list of a {\tt READ} statement may contain variables,
array-elements, character-substrings, or complete arrays of any data
type. An array-name without subscripts represents all the elements of the
array; this is not permitted for assumed-size dummy arguments in
procedures (because the array size is indeterminate).  The list may also
contain implied {\tt DO}-loops (explained below).

The data list of a {\tt WRITE} statement may contain any of the items 
permitted in a {\tt READ} statement and in addition expressions of any 
data type.  As in all I/O statements, expressions must not 
themselves involve the execution of other I/O statements.  


\subsubsection*{Implied {\tt DO}-loops} 

The simplest and most efficient way to read or write all the elements of
an array is to put its name, unsubscripted, in the data-transfer list.
In the case of a multi-dimensional array the elements will be transferred
in the normal storage sequence, with the first subscript varying most
rapidly.

An implied-DO loop allows the elements to be transferred 
selectively or in some non-standard order.  The rules for an 
implied-DO are similar to that of an ordinary {\tt DO}-loop but the loop 
forms a single item in the data-transfer list and is enclosed by a 
pair of parentheses rather than by {\tt DO} and {\tt CONTINUE}
statements.  
For example: 
\begin{verbatim}
      READ(UNIT=*, FMT=*) (ARRAY(I), I= IMIN, IMAX) 
      WRITE(UNIT=*, FMT=15) (M, X(M), Y(M), M=1,101,5) 
15    FORMAT(1X, I6, 2F12.3) 
\end{verbatim} 

A multi-dimensional array can be printed out in a transposed form.  
The next example outputs an array X(100,5) but with 5 elements 
across and 100 lines vertically:  
\begin{verbatim}
      WRITE(UNIT=*, FMT=5) (I,I=1,5), 
     $                     ((L,X(L,I),I=1,5),L=1,100) 
5     FORMAT(1X,'LINE', 5I10, 100(/,1X,I4, 5F10.2)) 
\end{verbatim} 

The first loop writes headings for the five columns, then the double 
loop writes a line-number for each line followed by five array 
elements.  Note that the parentheses have to be matched and that 
a comma is needed after the inner right-parenthesis since the inner 
loop is just an item in the list contained in the outer loop.  

The implied {\tt DO}-loop has the general form:\\ 
\verb+      (+ {\em data-list,} {\em loop-variable} = {\em start,} {\em 
limit,} {\em step} {\tt )}\\  
where the rules for the {\em start,} {\em limit,} and {\em step} values 
are exactly as in an ordinary {\tt DO} statement.  The {\em 
loop-variable} (normally an integer) may be used within the data-list and 
this list may, in turn, include further complete implied-DO lists. 

If an error or end-of-file condition occurs in an implied {\tt DO}-loop 
then the loop-control variable will be undefined on exit; this means 
that an explicit {\tt DO}-loop is required to read an indefinite list of 
data records and exit with knowledge of how many items were 
actually input. 

\subsection{\texttt{REWIND} and \texttt{BACKSPACE} Statements} 

These file-positioning statements may only be used on external 
sequential files; most systems will restrict them to files stored on 
suitable peripheral devices such as discs or tapes. 

{\tt REWIND} repositions a file to the beginning of information so that 
the next {\tt READ} statement will read the first record; if a {\tt WRITE} 
statement is used after {\tt REWIND} all existing records on the file are 
destroyed.  {\tt REWIND} has no effect if the file is already rewound.  
If a {\tt REWIND} statement is used on a unit which is connected but 
does not exist (e.g.\ a pre-connected output file) it creates the file. 

{\tt BACKSPACE} moves the file position back by one record so that the
record can be re-read or over-written.  There is no effect if the file is
already positioned at the beginning of information but it is an error to
back-space a file which does not exist.  It is also illegal to back-space
over records written by list-directed formatting (because the number of
records produced each time is system-dependent).  A few operating
systems find it difficult to implement the {\tt BACKSPACE} statement
directly and actually manage it only by rewinding the file and spacing
forward to the appropriate record.  It is sometimes possible to avoid
backspacing a file by allocating buffers within the program and, for a
formatted file, using the internal file {\tt READ} and {\tt WRITE}
statements.

These statements have similar general forms:\\ 
\verb+      REWIND(+ {\em control-list} {\tt )}\\ 
\verb+      BACKSPACE(+ {\em control-list} {\tt )}\\ 
where the {\em control-list} may contain: 
\begin{verbatim}
          UNIT=integer-expression 
          IOSTAT=integer-variable 
          ERR=label 
\end{verbatim} 
The unit identifier is compulsory, the others optional.  If only the 
unit identifier is used then (for compatibility with Fortran66) an 
abbreviated form of the statement is permitted:\\ 
\verb?      REWIND? {\em integer-expression}\\ 
\verb?      BACKSPACE? {\em integer-expression}\\ 
where the integer expression identifies the unit number. 

\section{\texttt{DATA} Statement} 

The {\tt DATA} statement is used to specify initial values for variables 
and array elements.  The {\tt DATA} statement is non-executable, but 
in a main program unit it has the same effect as a set of assignment 
statements at the very beginning of the program.  Thus in a main 
program a {\tt DATA} statement like this:  
\begin{verbatim}
       DATA LINES/625/, FREQ/50.0/, NAME/'PAL'/ 
\end{verbatim} 

could replace several assignment statements:  
\begin{verbatim}
       LINES = 625 
       FREQ  = 50.0 
       NAME  = 'PAL' 
\end{verbatim} 

This is more convenient, especially when initialising arrays, and 
efficient, since the work is done when the program is loaded.  

In a procedure, however, these two methods are not equivalent, 
especially in the case of items which are modified as the procedure 
executes.  A {\tt DATA} statement only sets the values once at the start 
of execution, whereas assignment statements will do so every time 
the procedure is called. 

It is important to distinguish between the {\tt DATA} and 
{\tt PARAMETER} statements.  The {\tt DATA} statement merely specifies 
an initial value for a variable (or array) which may be altered 
during the course of execution.  The {\tt PARAMETER} statement 
specifies values for constants which cannot be changed without 
recompiling the program.  If, however, you need an array of 
constants, for which there is no direct support in Fortran, you 
should use an ordinary array with a {\tt DATA} statement to initialise 
its elements (and take care not to corrupt the contents afterwards).  

\subsection{Defined and Undefined Values} 

The value of each variable and array element is undefined at the 
start of execution unless it has been initialised with a DATA 
statement.  An undefined value may only be used in executable 
statements in ways which cause it to become defined.  An item can 
become defined by its use in any of the following ways: 
\begin{itemize} 
\item  on the left-hand side of an assignment statement, 
\item  as the control variable of a {\tt DO} statement, 
\item  in the input list of a {\tt READ} statement, 
\item  as the internal file identifier of a {\tt WRITE} statement, 
\item  as the I/O status identifier in an I/O statement, 
\item  in an {\tt INQUIRE} statement except as file or unit number,  
\item  in a procedure call provided that the corresponding 
dummy argument is defined before the procedure returns 
control. 
\end{itemize} 

An undefined variable must not be used in any other way.  Errors 
caused by the inadvertent use of undefined values are easy to make 
and sometimes have very obscure effects.  It is important, 
therefore, to identify every item which needs to be initialised and 
provide a suitable set of {\tt DATA} statements.  

Modern operating systems often clear the area of memory into 
which they load a program to prevent unauthorized access to the 
data used in the preceding job.  A few operating systems preset 
their memory to a bit-pattern which corresponds to an illegal 
numerical value: this is a very helpful diagnostic facility since 
whenever an undefined variable is used in an expression it 
generates an error at run time.  Other systems merely set their 
memory to zero: this makes it more difficult to track down the use 
of indefined variables and they may only come to light when a 
program is transported to another system.  To rely on undefined 
variables and arrays having an initial value of zero is to leave the 
program completely at the mercy of changes to the operating 
system. 

\subsection{Initialising Variables} 

The simplest form of the {\tt DATA} statement consists of a list of the 
variable names each followed by a constant enclosed in a pair of 
slashes:    
\begin{verbatim}
       DOUBLE PRECISION EPOCH 
       LOGICAL OPENED 
       CHARACTER INFILE*20 
       DATA EPOCH/1950.0D0/, OPENED/.TRUE./, INFILE/'B:OBS.DAT'/ 
\end{verbatim} 

Note that {\tt DATA} statements must follow all specification 
statements.  An alternative form of the statement 
is to give first a complete list of names 
and then a separate list of constants: 
\begin{verbatim}
      DATA EPOCH, OPENED, INFILE / 1950.0D0, .TRUE., 'B:OBS.DAT'/ 
\end{verbatim} 

When there are many items to be initialised it is a matter of taste 
whether to use several {\tt DATA} statements or to use one with many 
continuation lines.  It is, of course, illegal to have the same name 
appearing twice. 

Character variables can be initialised in sections using the 
substring notation if this is more convenient:

\begin{verbatim}
      CHARACTER*52 LETTER 
      DATA LETTER(1:26)/'ABCDEFGHIJKLMNOPQRSTUVWXYZ'/, 
     $     LETTER(27:) /'abcdefghijklmnopqrstuvwxyz'/ 
\end{verbatim} 

If the length of the character constant differs from that of the 
variable then the string is truncated or padded with blanks as in an 
assignment statement.  The type conversion rules of assignment 
statements also apply to arithmetic items in {\tt DATA} statements. 

\subsection{Initialising Arrays} 

There are several ways of using {\tt DATA} statements to initialise 
arrays, all of them simpler and more efficient than the equivalent 
set of {\tt DO}-loops.  Perhaps the most common requirement is to 
initialise all the elements of an array: in this case the array name 
can appear without subscripts.  If several of the elements are to 
have the same initial value a repeat count can precede any 
constant:  
\begin{verbatim}
       REAL FLUX(1000) 
       DATA FLUX / 512*0.0, 488*-1.0 / 
\end{verbatim} 

The total number of constants must equal the number of array 
elements.  The constants correspond to the elements in the array in 
the normal storage sequence, that is with the first subscript varying 
most rapidly. 

Named constants are permitted, but not constant expressions.  The 
repeat count may be a literal or named integer constant.  
To initialise a multi-dimensional array with parameterised array 
bounds it is necessary to define another integer constant to hold the 
total number of elements: 

\begin{verbatim}
       PARAMETER (NX = 800, NY = 360, NTOTAL = NX * NY) 
       DOUBLE PRECISION SCREEN(NX,NY), ZERO 
       PARAMETER (ZERO = 0.0D0) 
       DATA SCREEN / NTOTAL * ZERO / 
\end{verbatim} 

If only a few array elements are to be initialised they can be listed 
individually:  
\begin{verbatim}
       REAL SPARSE(50,50) 
       DATA SPARSE(1,1), SPARSE(50,50) / 1.0, 99.99999 / 
\end{verbatim} 

The third, and most complicated, option is to use an implied-DO 
loop.  This operates in much the same way as an implied-DO in an 
I/O statement:  
\begin{verbatim}
       INTEGER ODD(10) 
       DATA (ODD(I),I=1,10,2)/ 5 * 43/ 
       DATA (ODD(I),I=2,10,2)/ 5 * 0 / 
\end{verbatim} 

This example has initialised all the odd numbered elements to one 
value and all the even numbered elements to another.  Note that 
the loop control variable (I in this example) has a scope which 
does not extend outside the section of the {\tt DATA} statement in 
which it is used.  Any integer variable may be used as a loop 
control index in a {\tt DATA} statement without effects elsewhere; the 
value of I itself is not defined by these statements. 

When initialising part of a multi-dimensional array it may 
occasionally be useful to nest {\tt DO}-loops like this: 
\begin{verbatim}
       DOUBLE PRECISION FIELD(5,5) 
       DATA ((FIELD(I,J),I=1,J), J=1,5) / 15 * -1.0D0 / 
\end{verbatim} 

This specifies initial values only for the upper triangle of the 
square array FIELD. 

\subsection{\texttt{DATA} Statements in Procedures} 

In procedures, {\tt DATA} statements perform a role for which 
assignment statements are no substitute.  It is quite often necessary 
to arrange for some action to be carried out at the start of the first 
call but not subsequently, such as opening a file or initialising a 
variable or array which accumulates information during 
subsequent calls. 

If information is preserved in a local variable or array from one 
invocation to another a {\tt SAVE} statement (described in section 
9.11) is also needed.  Indeed, in general any object initialised in a 
{\tt DATA} statement in a procedure also needs to be named in a {\tt SAVE} 
statement unless its value is never altered.  

In the next example the procedure opens a data file on its first call, 
using a logical variable OPENED to remember the state of the file.  
\begin{verbatim}
       SUBROUTINE LOOKUP(INDEX, RECORD)  
       INTEGER INDEX 
       REAL  RECORD 
       LOGICAL OPENED  
       SAVE OPENED 
       DATA OPENED / .FALSE. /  
*On first call OPENED is false so open the file. 
       IF(.NOT. OPENED) THEN 
            OPEN(UNIT=57, FILE='HIDDEN.DAT', STATUS='OLD', 
     $            ACCESS='DIRECT', RECL=100) 
            OPENED = .TRUE. 
       END IF 
       READ(UNIT=57, REC=INDEX) RECORD 
       END 
\end{verbatim} 

Here, for simplicity, the I/O unit number is a literal constant.  The 
procedure would be more modular if the unit number were also an 
argument of the procedure or if it contained some code, using the 
{\tt INQUIRE} statement, to determine for itself a suitable unused unit 
number.  

There is, of course, no corresponding way to determine which is 
the last call to the procedure so that the file can be closed, but this 
is not strictly necessary as the Fortran system closes all files 
automatically when the program exits.  

Note that {\tt DATA} statements cannot be used to initialise variables 
or arrays which are dummy arguments of a procedure, nor the 
variable which has the same name as the function.   

\subsection{General Rules} 

The general form of the {\tt DATA} statement is:\\ 
\verb+      DATA+ {\em nlist} {\tt /} {\em clist} {\tt /,} {\em nlist} 
{\tt /} {\em clist} {\tt /,} ...\\ 
Where {\em nlist} is a list of variable names, array names, substring 
names, and implied-DO lists, and {\em clist} is a list of items which may 
be literal or named constants or either of these preceded by a 
repeat-count and an asterisk. The repeat-count can also be an unsigned 
integer constant or named constant. 

The comma which precedes each list of names except the first is 
optional.  An implied-DO list has the general form:\\ 
\verb+      (+ {\em dlist,} {\em intvar} {\tt =} {\em start,} {\em
limit,} {\em 
step} {\tt )}\\ 
Where {\em dlist} is a list of implied-DO lists and array elements; {\em 
intvar} is an integer variable called the loop-control variable; {\em 
start}, {\em limit}, and {\em step} are integer expressions in which all 
the operands must be integer constants or loop-control variables of outer 
implied-DO lists. 

{\tt DATA} statements cannot be used to initialise items in the blank 
common block; items in named common blocks can only be 
initialised within a {\tt BLOCK DATA} program unit (see section 
12.4). 

The {\tt DATA} statements in each program unit must follow all 
specification statements but they can be interspersed with 
executable statements and statement function statements.  It is, 
however, best to follow the usual practice of putting all DATA 
statements before any of the executable statements. 

\section{Common Blocks} 

A common block is a list of variables and arrays stored in a named 
area which may be accessed directly in more than one program 
unit.  Common blocks are mainly used to transfer information from 
one program unit to another; they can be used in as an alternative 
to argument-list transfers or in addition to them.   

Common blocks are sometimes used to fit large programs into 
small computers by arranging for several program units to share a 
common pool of memory.  This is not a recommended 
programming practice and is likely to become redundant with the 
spread of virtual-memory operating systems. 

The name of a common block is an external name which must be 
different from all other global names, such as those of procedures, 
in the executable program.  The variables and arrays stored with 
the block cannot be initialised in the normal way, but only in a 
{\tt BLOCK DATA} program unit which was invented especially for 
this purpose.  

\subsection{Using Common Blocks} 

In most cases the best way to pass information from one program 
unit to another is to use the procedure argument list mechanism.  
This preserves the modularity and independence of procedures as 
much as possible.  Argument lists are, however, less satisfactory in 
a group of procedures forming a package which have to share a 
large amount of information with each other.  Procedure argument 
lists then tend to become long, cumbersome, and even inefficient.  
If this package of procedures is intended for general use it is quite 
important to keep the external interface as uncomplicated as 
possible.  This can be achieved by using the procedure argument 
lists only for import of information from and export to the rest of 
the program, and handling the communications between one 
procedure in the package and another with common blocks.  The 
user is then free to ignore the internal workings of the package. 

For example, in a simple package to handle a pen-plotter you may 
want to provide simple procedure calls such as:  
\begin{center} 
\begin{tabular}{ll} 
{\tt CALL PLOPEN} &  Initialise the plotting device\\ 
{\tt CALL SCALE(F)} & Set the scaling factor to F.\\ 
{\tt CALL MOVE(X,Y)} & Move the pen to position (X,Y)\\ 
{\tt CALL DRAW(X,Y)} & Draw a line from the last pen position to (X,Y).\\ 
\end{tabular} 
\end{center} 

These procedures clearly have to pass information from one to 
another about the current pen position, scaling factor, etc.  A 
suitable common block definition might look like this: 
\begin{verbatim}
       COMMON /PLOT/ OPENED, ORIGIN(2), PSCALE, NUMPEN 
       LOGICAL OPENED 
       INTEGER NUMPEN 
       REAL PSCALE, ORIGIN 
       SAVE /PLOT/ 
\end{verbatim} 

These specification statements would be needed in each procedure 
in the package.   

\subsubsection*{Common Block Names} 

A program unit can access the contents of any common block by 
quoting its name in a {\tt COMMON} statement.  Common block 
names are always enclosed in a pair of slashes and can only be 
used in {\tt COMMON} and {\tt SAVE} statements.  The common block 
itself has no data type and has a global name which must be 
distinct from the names of all program units.  The name should 
also be distinct from all local names in each program units which 
access the block.  Each program unit can make use of any number 
of different common blocks.  There is also a special blank or un-named common block with unique properties which are covered in 
section 12.2 below.   

The variables and arrays within a common block do not have any 
global status: they are associated with items in blocks bearing the 
same name in other program units only by their position within the 
block.  Thus, if in one program unit specifies:  
\begin{verbatim}
       COMMON /OBTUSE/ X(3) 
\end{verbatim} 

and in another: 
\begin{verbatim}
       COMMON /OBTUSE/ A, B, C 
\end{verbatim} 

then, assuming the data types are the same, X(1) corresponds to A, 
X(2) to B, and X(3) to C.  The {\tt COMMON} statements here are 
effectively setting up different names or aliases for the same set of 
memory locations.  The data types do not have to match provided 
the overall length is the same, but it is generally only possible to 
transfer information from one program unit to another if the 
corresponding items have the same data type.  If they do not, when 
one item becomes defined all names for the same location which 
have a different data type become undefined.  There is one minor 
exception to this rule: information may be transferred from a 
complex variable (or array element) to two variables of type real 
(or vice-versa) since these are directly associated with its real and 
imaginary parts. 

Usually it is necessary to arrange for corresponding items to have 
identical data types; it also minimises confusion if the same 
symbolic names are used as well.  The simplest way to achieve this 
is to use an {\tt INCLUDE} statement, if your system provides one.  
The include-file should contain not only the {\tt COMMON} statement 
but also all the associated type and {\tt SAVE} statements which are 
necessary.  It is, of course, still necessary to recompile every 
program unit which accesses the common block whenever its 
definition is altered significantly.  

\subsubsection*{Declaring Arrays} 

The bounds of an array can be declared in the {\tt COMMON} 
statement itself, or in a separate {\em type} or\\
{\tt DIMENSION} statement, but only in one of them.  Thus: 
\begin{verbatim}
       COMMON /DEMO/ ARRAY(5000) 
       DOUBLE PRECISION ARRAY 
\end{verbatim} 

is exactly equivalent to: 
\begin{verbatim}
       COMMON /DEMO/ ARRAY 
       DOUBLE PRECISION ARRAY(5000) 
\end{verbatim} 
or even: 
\begin{verbatim}
       COMMON /DEMO/ ARRAY 
       DOUBLE PRECISION ARRAY 
       DIMENSION ARRAY(5000) 
\end{verbatim} 
but the verbosity of the third form has little to recommend it. 

\subsubsection*{Data Types } 

The normal data type rules apply to variables and arrays in each 
common block.  A type statement is not required if the initial letter 
rule would have the required effect, but type statements are 
advisable, especially if the implied-type rules are anywhere 
affected by {\tt IMPLICIT} statements.  Type statements may precede 
or follow the {\tt COMMON} statement.  Similarly the lengths of 
character items should be specified in a separate type statement: 
these cannot be specified in the {\tt COMMON} statement.  

\subsubsection*{Storage Units} 

The length of each common block is measured in storage units, as 
described in section 5.1.  In summary, integer, real, and logical 
items occupy one numeric storage unit each; complex and double 
precision items occupy two each.  To maximise portability, 
character storage units are considered incommensurate with 
numerical storage units.  For this reason character and non-character items cannot be mixed in the same common block.   

In practice this often means that two common blocks are needed to 
hold a particular data structure: one for the character items and one 
for all the others.  If, in the first example, it had been necessary for 
the plotting package to store a plot title this would have to appear 
in a separate common block such as: 
\begin{verbatim}
       COMMON /PLOTC/ TITLE 
       CHARACTER TITLE*40 
       SAVE /PLOTC/ 
\end{verbatim} 

It is good practice to use related names for the blocks to indicate 
that the character and non-character items are used in conjunction. 

The length of a named common block must be the same in each 
program unit in which it appears.  Obviously the easiest way to 
ensure this is to make the common block contents identical in each 
program unit.  Note, however, that there is no requirement for data 
types to match, or for them to be listed in any particular order, 
provided the items are not used for information transfer, and 
provided the total length of the block is the same in each case.  
Thus these common blocks are both 2000 numerical storage units 
in length:  
\begin{verbatim}
       COMMON /SAME/ G(1000) 
       DOUBLE PRECISION G 

       COMMON /SAME/ A, B, C, R(1997) 
       REAL A, R 
       LOGICAL B 
       INTEGER C 
\end{verbatim} 

Items in a common block are stored in consecutive memory locations.
Unfortunately there are a few computer systems which require double precision
and complex items to be stored in even-numbered storage locations: these
may find it hard to cope with blocks which contain a mixture of data
types.  Machines with this defect can nearly always be placated by
arranging for all double precision and complex items to come at the
beginning of each block.

\subsubsection*{{\tt SAVE} Statements and Common Blocks} 

Items in common blocks may become undefined when a procedure 
returns control to the calling unit just like local variables and 
arrays.  This will not, however, occur in the case of the blank 
common block nor in any common block which is also declared in 
a program unit which is higher up the current chain of procedure 
calls.  Since the main program unit is always at the top of the chain 
any common block declared in the main program can never 
become undefined in this way.  In all other cases it is prudent to 
use {\tt SAVE} statements.   

The individual items in common blocks cannot be specified in a 
{\tt SAVE} statement, only the common block name itself.  Thus: 
\begin{verbatim}
       SAVE /SAME/, /DEMO/ 
\end{verbatim} 

If a common block is saved in any program unit then it must be 
saved in all of them.  The {\tt SAVE} statement ought therefore to be 
included with the {\tt COMMON} and associated type statements if 
{\tt INCLUDE} statements are used.  If the program is later modified so 
that the common block is also declared in the main program this 
will bring a {\tt SAVE} statement into the main program unit, but 
although it then has no effect, it does no harm. 

\subsubsection*{Restrictions} 

The dummy arguments of a procedure cannot be members of a 
common block nor, in a function, can the variable which has same 
name as the function.  There are also some restrictions on the use 
of common block items as actual arguments of procedure calls 
because of the possibility of multiple definition.  For example, if 
a procedure is defined like this: 
\begin{verbatim}
       SUBROUTINE SILLY(ARG) 
       COMMON /BLOCK/ COM 
\end{verbatim} 

And the same common block is also used in the calling unit, with 
a common block item as the actual argument, such as: 
\begin{verbatim}
       PROGRAM DUMMY 
       COMMON /BLOCK/ VALUE 
*... 
       CALL SILLY(VALUE) 
\end{verbatim} 

Then both ARG and COM within the subroutine SILLY are 
associated with the same item, VALUE, and it is therefore illegal 
to assign a new value to either of them.  

\subsection{Blank Common Blocks} 

Common blocks are sometimes also used to reduce the total 
amount of memory used by a program by arranging for several 
program units to share the same set of memory locations.  This is 
a difficult and risky procedure which should not be attempted 
unless all else fails.  

Most Fortran systems operate a storage allocation system which is 
completely static: each program unit has a separate allocation of 
memory for its local variables and arrays.  If several procedures 
each need to use large arrays internally the total amount of memory 
occupied by the program may be rather large.  If a set of 
procedures can be identified which are invoked in sequence, rather 
than one calling another, it may be feasible to reduce the total 
memory allocation by arranging for them to share a storage area.  
Each will use the same common block for their internal array 
space.   

Named common blocks are required to have the same length in each program
unit: if they are used it is necessary to work out which one needs the
most storage and pad out all the others to same length.  An alternative
is to the use the special blank (or un-named) common block which has the
useful property that it may have a different length in different program
units.

In one program unit, for example, you could specify: 
\begin{verbatim}
       COMMON // DUMMY(10000) 
\end{verbatim} 

and in another 
\begin{verbatim}
       COMPLEX SERIES(512,512), SLICE(512), EXPECT(1024) 
       COMMON // SERIES, SLICE, EXPECT 
\end{verbatim} 

The blank common block has two other special properties.  Firstly 
it cannot be initialised by a {\tt DATA} statement even within a 
{\tt BLOCK DATA} program unit (but this is not a serious limitation 
for a block used just for scratch storage).  Secondly items within 
the blank common block never become undefined as a result of a 
procedure exit.  For this reason the blank common block cannot be 
specified in a {\tt SAVE} statement.  

\subsection{\texttt{COMMON} Statement} 

A program unit may contain any number of {\tt COMMON} statements, 
each of which can define contents for any number of different 
common blocks.  {\tt COMMON} statements are specification 
statements and have a general form:  

{\tt COMMON /} name{\tt /} list-of-items , {\tt /} name {\tt /}
list-of-items  
... 

Each name is defined as a common block name, which has global 
scope.  The Fortran Standard allows it to use the same name as an 
intrinsic function, a local variable, or local array but not that of a 
named constant or an intrinsic function.  Each list of items can 
contain names of variables and arrays.  The array name may be 
followed by a dimension specification provided that each array is 
only dimensioned once in each program unit.  The comma shown 
before the second and subsequent block-name is optional.   

The name of the blank common block is normally specified as two 
consecutive slashes (ignoring any intervening blanks) but if it is 
the first block in the statement then the pair of slashes may be 
omitted.  

The contents of a common block are a concatenation of the all the 
definitions for it in the program unit.  Thus: 
\begin{verbatim}
       COMMON /ONE/ A, B, C, /TWO/ ALPHA, BETA, GAMMA 
       COMMON /TWO/ DELTA 
\end{verbatim} 

defines two blocks, /ONE/ contains three items while /TWO/ 
contains four of them. 

In procedures, variables which are dummy arguments or which are 
the same as the function name cannot appear in common blocks.  

\subsection{\texttt{BLOCK DATA} Program Units} 

The block data program unit is a special form of program unit 
which is required only if it is necessary to specify initial values for 
variables and arrays in named common blocks.  The program unit 
starts with a {\tt BLOCK DATA} statement, ends with an {\tt END} 
statement, and contains only specification and {\tt DATA} statements.  
Comment lines are also permitted.  The block data program unit is 
not executable and it is not a procedure. 

The next example initialises the items in the common block for the 
plotting package used in section 12.1, so that the initial pen 
position is at the origin, the scaling factor starts at one, and so on.  
Thus a suitable program unit would be: 
\begin{verbatim}
       BLOCK DATA SETPLT 
*SETPLT initialises the values used in the plotting package. 
       COMMON /PLOT/ OPENED, ORIGIN(2), PSCALE, NUMPEN 
       LOGICAL OPENED 
       INTEGER NUMPEN 
       REAL PSCALE, ORIGIN 
       SAVE /PLOT/ 
       DATA OPENED/.FALSE./, ORIGIN/2*0.0/, PSCALE/1.0/ 
       DATA NUMPEN/-1/ 
       END 
\end{verbatim} 

A block data unit can specify initial values for any number of 
named common blocks (blank common cannot be initialised).  
Each common block must be complete but it is not necessary to 
specify initial values for all of the items within it.  There can be 
more than one block data program unit, but a given common block 
cannot appear in more than one of them. 

For compatibility with Fortran66 it is also possible to have 
one un-named block data program unit in a program. 



\subsubsection*{Linking Block Data Program Units} 

If, when linking a program, one of the modules containing a 
procedure is accidentally omitted the linker is almost certain to 
produce an error message.  But, unless additional precautions are 
taken, this will not occur if a block data subprogram unit is 
omitted.  The program may even appear to work without it, but is 
likely to produce the wrong answer. 

There is a simple way to guard against this possibility: the name of 
the block data unit should be specified in an {\tt EXTERNAL} 
statement in at least some of the program units in which the 
common block is used.  There is no harm in declaring it in all of 
them.  This ensures that a link-time reference will be generated if 
any of these other program units are used.  There is a slight snag 
to this technique if an {\tt INCLUDE} statement is used to bring the 
common block definition into each program unit including the 
block data unit.  In order to avoid a self-reference, the include-file 
should not contain the {\tt EXTERNAL} statement. 

Despite this slight complication, this is a simple and valuable 
precaution.  It also makes it possible to hold block data units on 
object libraries and retrieve them automatically when they are 
required, just like all other types of subprogram unit. 


\section{Obsolete and Deprecated Features} 

None of the features covered here should be used in new software: 
some of them are completely obsolete, others have practical defects 
which make them unsuitable for use in well-structured software.  
These brief descriptions are provided only for the benefit of 
programmers who have to understand and update programs 
designed in earlier years.  

\subsection{Storage of Character Strings in Non-character Items} 

Before the advent of the character data type it was possible to store
text in arithmetic variables and arrays, although only very limited
manipulation was possible.  The number of characters which could be
stored in each item was entirely system-dependent.  One side-effect is
that many systems still allow the A format descriptor to match
input/output items of arithmetic types; this sometimes allows mismatches
between data-transfer lists and format descriptors to pass undetected.

\subsection{Arithmetic \texttt{IF} Statement} 

This is an executable statement with the form: 

{\tt IF(} arithmetic-expression {\tt )} label1, label2, label3 

It generally provides a three-way branch (but two of the labels may 
be identical for a two-way branch).  The expression may be an 
integer, real, or double-precision value: control is transferred to the 
statement attached to label1 if its value is negative, label2 if zero, 
or label3, if positive.   

\subsection{\texttt{ASSIGN} and assigned \texttt{GO TO} Statements} 

These two executable statements are normally used together.  The 
{\tt ASSIGN} statement assigns a statement label value to an integer 
variable.  When this has been done the variable no longer has an 
arithmetic value.  If the label is attached to an executable statement 
the variable can only be used in an assigned {\tt GO TO} statement; if 
attached to a {\tt FORMAT} statement the variable can only be used in 
a {\tt READ} or {\tt WRITE} statement.  The general forms of these 
statements are: 

{\tt ASSIGN} label {\tt TO} integer-variable 

{\tt GO TO} integer-variable ,(label, label, ... label) 

 
In the assigned {\tt GO TO} statement the comma and the entire 
parenthesised list of labels is optional.   

Assigned {\tt GO TO} can be used to provide a linkage to and from a 
section of a program unit acting as an internal subroutine, but is 
not a very convenient or satisfactory way of doing this. 

\subsection{\texttt{PAUSE} Statement} 

{\tt PAUSE} is an executable statement which halts the program in such 
a way that execution can be resumed in some way by the user (or 
on some systems by the computer operator).  The general forms of 
the statement are identical to those of {\tt STOP,} for example: 
\begin{verbatim}
       PAUSE 'NOW MOUNT THE NEXT TAPE' 
\end{verbatim} 
or 
\begin{verbatim}
       PAUSE  54321 
\end{verbatim} 

{\tt PAUSE} can be replaced by one {\tt WRITE} and one {\tt READ}
statement: 
this is more flexible and less system-dependent. 

\subsection{Alternate \texttt{RETURN}} 

The alternate {\tt RETURN} mechanism can be used in subroutines (but 
not external functions) to arrange a transfer of control to some 
labelled statement on completion of a {\tt CALL} statement.  In order 
to use it the arguments of the {\tt CALL} statement must include a list 
of labels, each preceded by an asterisk.  These labels are attached 
to points in the calling program unit at which execution may 
resume after the {\tt CALL} statement is executed.  For example: 
\begin{verbatim}
       CALL BAD(X, Y, Z, *150, *220, *390) 
\end{verbatim} 

The corresponding subroutine statement will have asterisks as 
dummy arguments for each label specification: 
\begin{verbatim}
       SUBROUTINE BAD(A, B, C, *, *, *) 
\end{verbatim} 

The return point depends on the value of an integer expression 
given in the {\tt RETURN} statement.  Thus: 
\begin{verbatim}
       RETURN 2 
\end{verbatim} 

will cause execution to be resumed at the statement attached to the 
second label argument, 220 in this case.  If the value of the integer 
expression in the {\tt RETURN} statement is not in the range 1 to n 
(where there are n label arguments) or a plain {\tt RETURN} statement 
is executed, then execution resumes at the statement after the 
{\tt CALL} in the usual way. 

The mechanism can be used for error-handling but is not very 
flexible as information cannot be passed through more than one 
procedure level. It is better to use an integer argument to return a 
status value and use that with an {\tt IF} (or even a computed {\tt GO TO} 
statement) in the calling program.   

\subsection{\texttt{ENTRY} Statement} 

{\tt ENTRY} statements can be used to specify additional entry points 
in external functions and subroutines.  {\tt ENTRY} is a non-executable 
statement which has the same form as a {\tt SUBROUTINE} statement.  
An {\tt ENTRY} statement may be used at any point in a procedure but 
all specification statements relating to its dummy arguments must 
appear in the appropriate place with the other specification 
statements.  If the main entry point is a {\tt SUBROUTINE} statement 
then all alternative entry points can be called in the same way as 
subroutines; if it is a {\tt FUNCTION} statement than all alternative 
entry point names can be used as functions.  If the main entry point 
is a character function then all the alternative entry points must 
also have that type.  Alternative entry points may have different 
lists of dummy arguments; it is up to the user to ensure that all 
those returning information to the calling program are properly 
defined before exit.  

The rules for the {\tt ENTRY} statement are necessarily complicated so 
it is easy to make mistakes.  It is generally better, or at least less 
unsatisfactory, to use a set of separate procedures which share 
information using common blocks. 

\subsection{\texttt{EQUIVALENCE} Statement} 

{\tt EQUIVALENCE} is a specification statement which causes two or 
more items (variables or arrays) to be associated with each other, 
i.e.\ to correspond to the same area of memory.  Character items can 
only be associated with other character items; otherwise the data 
types do not have to match.  As with common blocks, however, 
transfer of information is only permitted via associated items if 
their data types match.  A special exception is made for a complex 
item which is associated with two real ones. 

{\tt EQUIVALENCE} statements can be used fairly safely to provide 
a simple variable name as an alias for a particular array element or 
to associate a character variable with an array of the same length.  
For example: 
\begin{verbatim}
       CHARACTER STRING*80, ARRAY(80)*1 
       EQUIVALENCE (STRING, ARRAY) 
\end{verbatim} 

This slightly simplifies access to a single character in the string as 
the form ARRAY(K) can be used instead of STRING(K:K).  

The general form of the statement is: 

{\tt EQUIVALENCE} ( v, v, ... v ), ( v, v, ... v ), ... 

where each v is a variable, array, array element, or substring.  
Dummy arguments of procedures (and variables which are external 
function names) cannot appear.  An array name without subscripts 
refers to the first element of the array.  It is illegal to associate two 
or more elements of the same array, directly or indirectly, or do 
anything which conflicts with the storage sequence rules.  
Variables and arrays in common blocks can appear in 
{\tt EQUIVALENCE} statements but this has the effect of bringing all 
the associated items into the block.  They can be used to extend the 
contents of the block upwards, subject to the rules for common 
block length, but not downwards.  

Although the {\tt EQUIVALENCE} statement does have a few 
legitimate uses it is usually encountered in programs where the 
rules of Fortran are broken to obtain some special effect.  Programs 
which do this are rarely portable.   

\subsection{Specific Names of Intrinsic Functions} 

Specific names should be used instead of the generic name of an 
intrinsic function only if the name is to be the actual argument of 
a procedure call; the name then must also be declared in an 
INTRINSIC statement.  The following intrinsic functions cannot 
be used in this way, and their specific names are therefore 
completely obsolete. 

\begin{center} 
\begin{tabular}{ll} 
\hline 
Obsolete specific name & Preferred generic form \\ 
\hline 
IFIX, IDINT & INT\\ 
FLOAT, SNGL & REAL \\ 
MAX0, AMAX1, DMAX1 & MAX \\ 
AMAX0, MAX1        & MAX * \\ 
MIN0, AMIN1, DMIN1 & MIN \\ 
AMIN0, MIN1 & MIN * \\ 
\hline 
\end{tabular} 
\end{center} 
{\tt *} the functions AMAX0, MAX1, AMIN0, and MIN1 which 
have a data type different from that of their arguments can 
only be replaced by appropriate type conversion functions in 
addition to MAX or MIN. 

\subsection{\texttt{PRINT} Statement and simplified \texttt{READ}} 

The {\tt PRINT} statement can produce formatted or list-directed output 
on the standard pre-connected output file.  Thus these two statements 
have exactly the same effect: 
\begin{verbatim}
       PRINT fmt, data-list 
       WRITE(*, fmt) data-list 
\end{verbatim} 

The {\tt PRINT} statement is limited in its functionality and misleading, 
since there is no necessity for its output to appear in printed form. 

In a similar way there is a simplified form of READ statement, so these 
have the same effect: 
\begin{verbatim}
       READ fmt, data-list 
       READ(*, fmt) data-list 
\end{verbatim} 

\subsection{\texttt{END FILE} Statement} 

The {\tt END FILE} statement has the same general forms as {\tt REWIND}
and {\tt BACKSPACE}: 
\begin{verbatim}
       END FILE(UNIT=unit, ERR=label, IOSTAT=int-var) 
       END FILE unit 
\end{verbatim} 

It appends a special ``end-file'' record to a sequential file which is 
designed to trigger the end-of-file detection mechanism on 
subsequent input.  No further records can be written to the file 
after this end-file record, i.e.\ the next operation must be {\tt CLOSE}, 
{\tt REWIND,} or {\tt BACKSPACE}. 

The statement seems to be superfluous on almost all current 
systems since they can detect the end of an input file without its 
aid.  The Fortran Standard requires that the end-file record be 
treated as a physical record, so that after an end-of-file condition 
has been detected an explicit {\tt BACKSPACE} operation is required 
before any new data records are appended.  This notion is 
somewhat artificial and not all systems implement it correctly.  
This is one of the few cases where a deliberate departure from the 
Fortran Standard can enhance portability. 

\subsection{Obsolete Format Descriptors} 

The data descriptor {\tt Dw.d} is exactly equivalent to Ew.d on input; 
on output it is similar except that the exponent will use the letter 
{\tt D} instead of {\tt E}. Real and double precision data items can be
read 
equally well by {\tt D,} {\tt E,} {\tt F,} or {\tt G} descriptors. 

The format descriptor  
\verb? nHstring ? is exactly equivalent to  
\verb? 'string' ? 
(where n is an unsigned integer constant giving the length of the 
string).  When used with a formatted {\tt WRITE} statement the string 
is copied to the output record.  The {\tt nH} form does not require 
apostrophes to be doubled within the string but does require an 
accurate character count. 

\section{Common Extensions to the Fortran Standard}

Almost before the official Standard (ANSI X3.9-1978) for Fortran77 had
been defined, various software producers started to add their own
favourite features. The US Department of Defense issued in 1988 a
supplement called MIL-STD-1753 setting out a list of extensions that it
required Fortran systems to support if they were to tender for DoD
contracts.  This requirement later spread to other areas of Federal
Government procurement, so these extensions are now almost universally
provided and can be used with confidence without reducing portability.

\subsection{MIL-STD-1753 Extensions}

\subsubsection*{\texttt{IMPLICIT NONE} statement}

This statement says that there are no default data types in this program
unit, so that all named items (variables, arrays, constants, and external
functions) must have their data type specified in an explicit {\em type}
statement.  It must appear before all these specification statements and
cannot be used together with any other {\tt IMPLICIT} statement in the
same program unit.  Although novice programmers find it tedious to have
to declare each name before using it, the benefits are considerable
in that mis-spelled names come to light much more easily, and most
professional programmers find it a worth-while investment.

\subsubsection*{\texttt{INCLUDE} statement}

The {\tt INCLUDE} statement specifies the name of another file which
contains some source code.  It is most often used to contain a set of
specification statements which are common to a number of different
program units, for example {\tt COMMON} blocks and their associated {\em
type} statements, or a list of common constants such as $\pi$.  The form
of file-name is, of course, system dependent.  In portable software it is
prudent to choose a simple name which is likely to be acceptable to most
operating systems. For example:

\begin{verbatim}
      INCLUDE 'trig.inc'
\end{verbatim}
where the file {\tt trig.inc} (or maybe {\tt TRIG.INC}) contains:
\begin{verbatim}
      REAL PI, TWOPI, RTOD
      PARAMETER (PI = 3.14159265, TWOPI=2.0*PI, RTOD=PI/180.0)
\end{verbatim}
If such constants are defined only once, it is much easier to ensure that
they are correct.  Similarly the definition of a {\tt COMMON} block in
only one place ensures that its consistency throughout the program.

\subsubsection*{\texttt{DO}-loops with \texttt{END DO}}
The Fortran77 Standard seemed deficient in pairing {\tt IF} with {\tt END
IF} but not {\tt DO} with {\tt END DO}.  This extension is widely
available and helpful in that it avoids the need to use a different 
statement label on each loop.  For
example:
\begin{verbatim}
      DO J = 1,NPTS
         SUM   = SUM + X(I)
         SUMSQ = SUMSQ + X(I)**2
      END DO
\end{verbatim}
It is good practice to indent the lines between the {\tt DO} and {\tt END
DO} statements to make the repeated section obvious.  The appearance of a
statement label in such code usually marks the destination of a {\tt GO
TO} statement, and alerts the programmer to some unusual alteration to
the standard sequence of operations.  Where only labelled {\tt DO}-loops
are used, such exceptions are harder to spot.

\appendix
\renewcommand{\appendixname}{Appendix}
\section{List of Intrinsic Functions} 

This table shows the number of arguments for each function and what data 
types are permitted. The data type codes are: {\tt I} = Integer, {\tt R} 
= Real, {\tt D} = Double precision, {\tt X} = Complex, {\tt C} = 
Character, {\tt L} = Logical, {\tt *} means the result has the same data 
type as the argument(s).  Note that if there is more than one argument in 
such cases they must all have the same data type. 

\begin{tabular}{p{1.9in}p{3.8in}} 
{\tt R = ABS(X) } & Takes the modulus of a complex number 
(i.e.\ the square-root of the sum of the 
squares of the two components).\\ 
{\tt *      =      ACOS(RD)} &  
Arc-cosine; the result is in the range 0 to $+ \pi$ \\ 
{\tt R      =     AIMAG(X) } & Extracts the imaginary component of a 
complex number.  Use REAL to extract the 
real component. \\ 
{\tt *      =     ANINT(RD) } & Rounds to the nearest whole number.\\ 
{\tt *      =     ATAN2(RD,RD)} & Arc-tangent of $arg_{1}$/$arg_{2}$
resolved into the 
correct quadrant, the result is in the range 
$-\pi$ to $+ \pi$.  It is an error to have both arguments zero. \\ 
{\tt C = CHAR(I) } & Returns Nth character in local character 
code table.\\ 
{\tt X = CMPLX(IRDX,IRD) } & Converts to complex, second arg optional. \\ 
{\tt X = CONJG(X)    } & Computes the complex conjugate of a complex
number. \\ 
{\tt * = COS(RDX)    } & Cosine of the angle in radians.\\ 
{\tt D = DBLE(IRDX)  } & Converts to double precision. \\ 
{\tt * = DIM(IRD,IRD)} & Returns the positive difference of $arg_{1}$ and 
$arg_{2}$, i.e.\ if $arg_{1}$ {\tt >} $arg_{2}$ it returns ($arg_{1}$  
- $arg_{2}$), otherwise zero.\\ 
{\tt D = DPROD(R,R)  } & Computes the double precision product of two real 
values. \\ 
{\tt * = EXP(RDX)    } & Returns the exponential, i.e.\ e to the power 
of the argument.  This is the inverse of the natural logarithm.\\ 
{\tt I = ICHAR(C)    } & Returns position of first character of the 
string in the local character code table.\\ 
{\tt I = INDEX(C,C)   } & Searches first string and returns position of 
second string within it, otherwise zero.\\ 
{\tt I = INT(IRDX)   } & Converts to integer by truncation.\\ 
{\tt I = LEN(C)      } & Returns length of the character argument.\\ 
{\tt L = LGE(C,C)    } & Lexical comparison using ASCII character 
code: returns true if $arg_{1} >= arg_{2}$. \\ 
{\tt L = LGT(C,C)    } & Lexical comparison using ASCII character 
code: returns true if $arg_{1} > arg_{2}$. \\ 
{\tt L = LLE(C,C)    } & Lexical comparison using ASCII character 
code: returns true if $arg_{1} <= arg_{2}$. \\ 
{\tt L = LLT(C,C)    } & Lexical comparison using ASCII character 
code: returns true if $arg_{1} < arg_{2}$. \\ 
{\tt * = LOG(RDX)    } & Logarithm to base e (where 
e=2.718...).\\ 
{\tt * = LOG10(RD)   } & Logarithm to base 10.\\ 
{\tt * = MAX(IRD,IRD},...{\tt )} & Returns the largest of its arguments.\\ 
{\tt * = MIN(IRD,IRD},...{\tt )} & Returns the smallest of its
arguments.\\ 
{\tt * = MOD(IRD,IRD)} & Returns $arg_{1}$ modulo $arg_{2}$, i.e.\ the  
remainder after dividing $arg_{1}$ by $arg_{2}$.\\ 
{\tt R = REAL(IRDX)  } & Converts to real.\\ 
{\tt * = SIGN(IRD,IRD)} & Performs sign transfer: if $arg_{2}$ is
negative the  
result is $-arg_{1}$, if $arg_{2}$ is zero or positive the result is  
$arg_{1}$.\\ 
{\tt * = SQRT(RDX)   } & Square root. \\ 
{\tt * = TAN(RD)     } & Tangent of the angle in radians.\\ 
\end{tabular} 

\section{Specific Names of Generic Functions} 

Specific names are still needed when the function name is used as 
the actual argument of another procedure.  The specific name must 
then also be declared in an {\tt INTRINSIC} statement.  This table lists 
all the specific names which are still useful in Fortran77.  The 
other functions either do not have generic names or cannot be 
passed as actual arguments. 
\begin{center} 
\begin{tabular}{|l|l|l|l|l|} 
\hline 
Generic &  \multicolumn{4}{c|}{Specific names}   \\ 
\cline{2-5}
Name    &  INTEGER & REAL & DOUBLE PRECISION & COMPLEX \\ 
\hline 
ABS & IABS & ABS & DABS & CABS \\ 
ACOS &     & ACOS & DACOS & \\ 
AINT &     & AINT & DINT  & \\ 
ANINT &    & ANINT & DNINT & \\ 
ASIN  &    & ASIN  & DASIN & \\ 
ATAN &     & ATAN  & DATAN & \\ 
ATAN2 &    & ATAN2 & DATAN2 & \\ 
COS   &    & COS   & DCOS   & CCOS \\ 
COSH &     & COSH  & DCOSH  & \\ 
DIM  & IDIM & DIM  & DDIM & \\ 
EXP  &     & EXP   & DEXP & CEXP \\ 
LOG  &     & ALOG  & DLOG & CLOG \\ 
LOG10 &    & ALOG10 & DLOG10 & \\ 
MOD  & MOD & AMOD  & DMOD & \\ 
NINT &     & NINT  & IDNINT & \\ 
SIGN & ISIGN & SIGN & DSIGN & \\ 
SIN  &     & SIN   & DSIN  & CSIN \\ 
SINH &     & SINH  & DSINH & \\ 
SQRT &     & SQRT  & DSQRT & CSQRT \\ 
TAN  &     & TAN   & DTAN  & \\ 
TANH &     & TANH  & DTANH & \\ 
\hline 
\end{tabular} 
\end{center}

\section{GNU Free Documentation Licence}

\noindent Version 1.1, March 2000

\noindent Copyright (C) 2000  Free Software Foundation, Inc.
59 Temple Place, Suite 330, Boston, \\MA~\mbox{02111-1307} USA

\noindent Everyone is permitted to copy and distribute verbatim copies
of this license document, but changing it is not allowed.

\setcounter{subsection}{-1}

\subsection{PREAMBLE}

The purpose of this License is to make a manual, textbook, or other
written document ``free'' in the sense of freedom: to assure everyone
the effective freedom to copy and redistribute it, with or without
modifying it, either commercially or noncommercially.  Secondarily,
this License preserves for the author and publisher a way to get
credit for their work, while not being considered responsible for
modifications made by others.

This License is a kind of ``copyleft'', which means that derivative
works of the document must themselves be free in the same sense.  It
complements the GNU General Public License, which is a copyleft
license designed for free software.

We have designed this License in order to use it for manuals for free
software, because free software needs free documentation: a free
program should come with manuals providing the same freedoms that the
software does.  But this License is not limited to software manuals;
it can be used for any textual work, regardless of subject matter or
whether it is published as a printed book.  We recommend this License
principally for works whose purpose is instruction or reference.


\subsection{APPLICABILITY AND DEFINITIONS}

This License applies to any manual or other work that contains a
notice placed by the copyright holder saying it can be distributed
under the terms of this License.  The ``Document'', below, refers to any
such manual or work.  Any member of the public is a licensee, and is
addressed as ``you''.

A ``Modified Version'' of the Document means any work containing the
Document or a portion of it, either copied verbatim, or with
modifications and/or translated into another language.

A ``Secondary Section'' is a named appendix or a front-matter section of
the Document that deals exclusively with the relationship of the
publishers or authors of the Document to the Document's overall subject
(or to related matters) and contains nothing that could fall directly
within that overall subject.  (For example, if the Document is in part a
textbook of mathematics, a Secondary Section may not explain any
mathematics.)  The relationship could be a matter of historical
connection with the subject or with related matters, or of legal,
commercial, philosophical, ethical or political position regarding
them.

The ``Invariant Sections'' are certain Secondary Sections whose titles
are designated, as being those of Invariant Sections, in the notice
that says that the Document is released under this License.

The ``Cover Texts'' are certain short passages of text that are listed,
as Front-Cover Texts or Back-Cover Texts, in the notice that says that
the Document is released under this License.

A ``Transparent'' copy of the Document means a machine-readable copy,
represented in a format whose specification is available to the
general public, whose contents can be viewed and edited directly and
straightforwardly with generic text editors or (for images composed of
pixels) generic paint programs or (for drawings) some widely available
drawing editor, and that is suitable for input to text formatters or
for automatic translation to a variety of formats suitable for input
to text formatters.  A copy made in an otherwise Transparent file
format whose markup has been designed to thwart or discourage
subsequent modification by readers is not Transparent.  A copy that is
not ``Transparent'' is called ``Opaque''.

Examples of suitable formats for Transparent copies include plain
ASCII without markup, Texinfo input format, LaTeX input format, SGML
or XML using a publicly available DTD, and standard-conforming simple
HTML designed for human modification.  Opaque formats include
PostScript, PDF, proprietary formats that can be read and edited only
by proprietary word processors, SGML or XML for which the DTD and/or
processing tools are not generally available, and the
machine-generated HTML produced by some word processors for output
purposes only.

The ``Title Page'' means, for a printed book, the title page itself,
plus such following pages as are needed to hold, legibly, the material
this License requires to appear in the title page.  For works in
formats which do not have any title page as such, ``Title Page'' means
the text near the most prominent appearance of the work's title,
preceding the beginning of the body of the text.


\subsection{VERBATIM COPYING}

You may copy and distribute the Document in any medium, either
commercially or noncommercially, provided that this License, the
copyright notices, and the license notice saying this License applies
to the Document are reproduced in all copies, and that you add no other
conditions whatsoever to those of this License.  You may not use
technical measures to obstruct or control the reading or further
copying of the copies you make or distribute.  However, you may accept
compensation in exchange for copies.  If you distribute a large enough
number of copies you must also follow the conditions in section 3.

You may also lend copies, under the same conditions stated above, and
you may publicly display copies.


\subsection{COPYING IN QUANTITY}

If you publish printed copies of the Document numbering more than 100,
and the Document's license notice requires Cover Texts, you must enclose
the copies in covers that carry, clearly and legibly, all these Cover
Texts: Front-Cover Texts on the front cover, and Back-Cover Texts on
the back cover.  Both covers must also clearly and legibly identify
you as the publisher of these copies.  The front cover must present
the full title with all words of the title equally prominent and
visible.  You may add other material on the covers in addition.
Copying with changes limited to the covers, as long as they preserve
the title of the Document and satisfy these conditions, can be treated
as verbatim copying in other respects.

If the required texts for either cover are too voluminous to fit
legibly, you should put the first ones listed (as many as fit
reasonably) on the actual cover, and continue the rest onto adjacent
pages.

If you publish or distribute Opaque copies of the Document numbering
more than 100, you must either include a machine-readable Transparent
copy along with each Opaque copy, or state in or with each Opaque copy
a publicly-accessible computer-network location containing a complete
Transparent copy of the Document, free of added material, which the
general network-using public has access to download anonymously at no
charge using public-standard network protocols.  If you use the latter
option, you must take reasonably prudent steps, when you begin
distribution of Opaque copies in quantity, to ensure that this
Transparent copy will remain thus accessible at the stated location
until at least one year after the last time you distribute an Opaque
copy (directly or through your agents or retailers) of that edition to
the public.

It is requested, but not required, that you contact the authors of the
Document well before redistributing any large number of copies, to give
them a chance to provide you with an updated version of the Document.


\subsection{MODIFICATIONS}

You may copy and distribute a Modified Version of the Document under
the conditions of sections 2 and 3 above, provided that you release
the Modified Version under precisely this License, with the Modified
Version filling the role of the Document, thus licensing distribution
and modification of the Modified Version to whoever possesses a copy
of it.  In addition, you must do these things in the Modified Version:

\begin{enumerate}
\renewcommand{\labelenumi}{\Alph{enumi}.}
\item Use in the Title Page (and on the covers, if any) a title distinct
   from that of the Document, and from those of previous versions
   (which should, if there were any, be listed in the History section
   of the Document).  You may use the same title as a previous version
   if the original publisher of that version gives permission.
\item List on the Title Page, as authors, one or more persons or entities
   responsible for authorship of the modifications in the Modified
   Version, together with at least five of the principal authors of the
   Document (all of its principal authors, if it has less than five).
\item State on the Title page the name of the publisher of the
   Modified Version, as the publisher.
\item Preserve all the copyright notices of the Document.
\item Add an appropriate copyright notice for your modifications
   adjacent to the other copyright notices.
\item Include, immediately after the copyright notices, a license notice
   giving the public permission to use the Modified Version under the
   terms of this License, in the form shown in the Addendum below.
\item Preserve in that license notice the full lists of Invariant Sections
   and required Cover Texts given in the Document's license notice.
\item Include an unaltered copy of this License.
\item  Preserve the section entitled ``History'', and its title, and add to
   it an item stating at least the title, year, new authors, and
   publisher of the Modified Version as given on the Title Page.  If
   there is no section entitled ``History'' in the Document, create one
   stating the title, year, authors, and publisher of the Document as
   given on its Title Page, then add an item describing the Modified
   Version as stated in the previous sentence.
\item  Preserve the network location, if any, given in the Document for
   public access to a Transparent copy of the Document, and likewise
   the network locations given in the Document for previous versions
   it was based on.  These may be placed in the ``History'' section.
   You may omit a network location for a work that was published at
   least four years before the Document itself, or if the original
   publisher of the version it refers to gives permission.
\item In any section entitled ``Acknowledgements'' or ``Dedications'',
   preserve the section's title, and preserve in the section all the
   substance and tone of each of the contributor acknowledgements
   and/or dedications given therein.
\item Preserve all the Invariant Sections of the Document,
   unaltered in their text and in their titles.  Section numbers
   or the equivalent are not considered part of the section titles.
\item Delete any section entitled ``Endorsements''.  Such a section
   may not be included in the Modified Version.
\item Do not retitle any existing section as ``Endorsements''
   or to conflict in title with any Invariant Section.
\end{enumerate}

If the Modified Version includes new front-matter sections or
appendices that qualify as Secondary Sections and contain no material
copied from the Document, you may at your option designate some or all
of these sections as invariant.  To do this, add their titles to the
list of Invariant Sections in the Modified Version's license notice.
These titles must be distinct from any other section titles.

You may add a section entitled ``Endorsements'', provided it contains
nothing but endorsements of your Modified Version by various
parties--for example, statements of peer review or that the text has
been approved by an organization as the authoritative definition of a
standard.

You may add a passage of up to five words as a Front-Cover Text, and a
passage of up to 25 words as a Back-Cover Text, to the end of the list
of Cover Texts in the Modified Version.  Only one passage of
Front-Cover Text and one of Back-Cover Text may be added by (or
through arrangements made by) any one entity.  If the Document already
includes a cover text for the same cover, previously added by you or
by arrangement made by the same entity you are acting on behalf of,
you may not add another; but you may replace the old one, on explicit
permission from the previous publisher that added the old one.

The author(s) and publisher(s) of the Document do not by this License
give permission to use their names for publicity for or to assert or
imply endorsement of any Modified Version.


\subsection{COMBINING DOCUMENTS}

You may combine the Document with other documents released under this
License, under the terms defined in section 4 above for modified
versions, provided that you include in the combination all of the
Invariant Sections of all of the original documents, unmodified, and
list them all as Invariant Sections of your combined work in its
license notice.

The combined work need only contain one copy of this License, and
multiple identical Invariant Sections may be replaced with a single
copy.  If there are multiple Invariant Sections with the same name but
different contents, make the title of each such section unique by
adding at the end of it, in parentheses, the name of the original
author or publisher of that section if known, or else a unique number.
Make the same adjustment to the section titles in the list of
Invariant Sections in the license notice of the combined work.

In the combination, you must combine any sections entitled ``History''
in the various original documents, forming one section entitled
``History''; likewise combine any sections entitled ``Acknowledgements'',
and any sections entitled ``Dedications''.  You must delete all sections
entitled ``Endorsements.''


\subsection{COLLECTIONS OF DOCUMENTS}

You may make a collection consisting of the Document and other documents
released under this License, and replace the individual copies of this
License in the various documents with a single copy that is included in
the collection, provided that you follow the rules of this License for
verbatim copying of each of the documents in all other respects.

You may extract a single document from such a collection, and distribute
it individually under this License, provided you insert a copy of this
License into the extracted document, and follow this License in all
other respects regarding verbatim copying of that document.


\subsection{AGGREGATION WITH INDEPENDENT WORKS}

A compilation of the Document or its derivatives with other separate
and independent documents or works, in or on a volume of a storage or
distribution medium, does not as a whole count as a Modified Version
of the Document, provided no compilation copyright is claimed for the
compilation.  Such a compilation is called an ``aggregate'', and this
License does not apply to the other self-contained works thus compiled
with the Document, on account of their being thus compiled, if they
are not themselves derivative works of the Document.

If the Cover Text requirement of section 3 is applicable to these
copies of the Document, then if the Document is less than one quarter
of the entire aggregate, the Document's Cover Texts may be placed on
covers that surround only the Document within the aggregate.
Otherwise they must appear on covers around the whole aggregate.


\subsection{TRANSLATION}

Translation is considered a kind of modification, so you may
distribute translations of the Document under the terms of section 4.
Replacing Invariant Sections with translations requires special
permission from their copyright holders, but you may include
translations of some or all Invariant Sections in addition to the
original versions of these Invariant Sections.  You may include a
translation of this License provided that you also include the
original English version of this License.  In case of a disagreement
between the translation and the original English version of this
License, the original English version will prevail.


\subsection{TERMINATION}

You may not copy, modify, sublicense, or distribute the Document except
as expressly provided for under this License.  Any other attempt to
copy, modify, sublicense or distribute the Document is void, and will
automatically terminate your rights under this License.  However,
parties who have received copies, or rights, from you under this
License will not have their licenses terminated so long as such
parties remain in full compliance.

\subsection{FUTURE REVISIONS OF THIS LICENSE}

The Free Software Foundation may publish new, revised versions
of the GNU Free Documentation License from time to time.  Such new
versions will be similar in spirit to the present version, but may
differ in detail to address new problems or concerns.  See
\verb+http://www.gnu.org/copyleft/+.

Each version of the License is given a distinguishing version number.
If the Document specifies that a particular numbered version of this
License ``or any later version'' applies to it, you have the option of
following the terms and conditions either of that specified version or
of any later version that has been published (not as a draft) by the
Free Software Foundation.  If the Document does not specify a version
number of this License, you may choose any version ever published (not
as a draft) by the Free Software Foundation.

\section{Acknowkedgements}

I am grateful to many people who have let me know of mistakes in my text.
These include: Paul Youngman, Mikhail Titov, Myron Calhoun, David Simpson,
Rory Yorke, Alexandr Lomovtsev, John Girash, Rob Scott, and Jan
Wennstrom.  Apologies to those whose contributions I have forgotten to
record. 

\end{document} 
 
